\documentclass[a4paper, 12pt]{article}

\usepackage{graphicx}
\usepackage{longtable}
\graphicspath{ {images/} }

\newcommand{\templates}{../../template}
\usepackage[a4paper, margin=2.5cm]{geometry}

\usepackage{enumitem}
\setlist[itemize]{noitemsep}
\setlist[enumerate]{noitemsep}

\let\oldpar\paragraph
\renewcommand{\paragraph}[1]{\oldpar{#1\\}\noindent}
\input{\templates/front_page}
\usepackage{hyperref}
\usepackage{array}
\usepackage{tabularx}

\def\vers#1-#2-#3-#4-#5\\{#1&#2&#3&#4&#5\\\hline}

\newcommand{\addversione}[5]{
	\ifdefined\versioni
		\let\old\versioni
		\renewcommand{\versioni}{#1&#2&#3&#4&#5\\\hline\old}
	\else
		\newcommand{\versioni}{#1&#2&#3&#4&#5\\\hline}
	\fi
}

\newcommand{\setversioni}[1]{\newcommand{\versioni}{#1}}

\newcommand{\makeversioni}{
	\begin{center}
		\begin{tabularx}{\textwidth}{|c|c|c|c|X|}
		\hline
		\textbf{Versione} & \textbf{Data} & \textbf{Persona} & \textbf{Attivtà} & \textbf{Descrizione} \\
		\hline
		\versioni
		\end{tabularx}
	\end{center}
	\clearpage
}

\settitolo{Specifiche Architetturali}
\setprogetto{ShopChain}
\setcommittenti{SyncLab}
\setredattori{Giovanni Cocco}
\setrevisori{Federico Marchi}
\setresponsabili{Giovanni Cocco}
\setdestuso{esterno}
\setdescrizione{
Architettura del progetto
}

\addversione{1.0.0}{23/2/2022}{Giovanni Cocco}{Redazione}{Creazione del documento}

\begin{document}

\makefrontpage

\makeversioni

\tableofcontents
\clearpage

\section{Introduzone}
\subsection{Scopo del documento}
Il documento illustra le scelte architetturali e illustra l'architettura.

\section{Tecnologie/Linguaggi/Librerie}

\subsection{Solidity}
\begin{itemize}
\item \textbf{Versione:} 0.8.13
\item \textbf{Documentazione:} \href{https://docs.soliditylang.org/en/v0.8.13/}{https://docs.soliditylang.org/en/v0.8.13/}
\end{itemize}

\subsection{Typescript}
\begin{itemize}
\item \textbf{Versione:} 4.6.3
\item \textbf{Documentazione:} \href{https://www.typescriptlang.org/docs/}{https://www.typescriptlang.org/docs/}
\end{itemize}

\subsection{Express}
\begin{itemize}
\item \textbf{Versione:} 4.17.2
\item \textbf{Documentazione:} \href{https://devdocs.io/express/}{https://devdocs.io/express/}
\end{itemize}

\subsection{MariaDB}
\begin{itemize}
\item \textbf{Versione:} 10.7.3
\item \textbf{Documentazione:} \href{https://mariadb.com/kb/en/documentation/}{https://mariadb.com/kb/en/documentation/}
\end{itemize}

\subsection{Python}
\begin{itemize}
\item \textbf{Versione:} 3.8
\item \textbf{Documentazione:} \href{https://docs.python.org/3.8/}{https://docs.python.org/3.8/}
\end{itemize}

\subsection{Web3}
\begin{itemize}
\item \textbf{Versione:} 1.7.1
\item \textbf{Documentazione:} \href{https://web3js.readthedocs.io/en/v1.7.1/}{https://web3js.readthedocs.io/en/v1.7.1/}
\end{itemize}

\subsection{MetaMask}
\begin{itemize}
\item \textbf{Versione:} 10.11.3
\item \textbf{Documentazione:} \href{https://docs.metamask.io/guide/}{https://docs.metamask.io/guide/}
\end{itemize}
\clearpage
\section{Architettura}
Il progetto si compone di 4 macro parti:
\begin{itemize}
\item Server
\item Smart contract
\item Web app
\item Script di messa in vendita
\end{itemize}
\subsection{Server}
\subsubsection{Diagramma delle classi}
\includegraphics[width=1.0\textwidth]{server_class}
\paragraph{Commenti}
La classe ShopContract andrà a interfacciarsi con il contratto in blockchain.\\
La classe PageCreator andrà a interfacciarsi con la WebApp.
\subsubsection{Design pattern: Constructor injection}
\paragraph{Descrizione}
Le dipendenze sono tracciate e passate agli oggetti tramite il construttore.
\paragraph{Motivazioni}
Facilità il tracciamento delle dipendenze e agevola il mocking in fase di test.

\subsection{Smart contract}
\subsubsection{Diagramma delle classi}
\includegraphics[width=1.0\textwidth]{contract}
\paragraph{Commenti}
\subsubsection{Pattern architetturale adottato}
\paragraph{Descrizione}
\paragraph{Motivazioni}

\subsection{Web app}
\subsubsection{Diagramma delle classi}
[grafico]
\paragraph{Commenti}
\subsubsection{Pattern architetturale adottato}
\paragraph{Descrizione}
\paragraph{Motivazioni}

\subsection{Script di messa in vendita}
\subsubsection{sell\_item}
\textbf{Parametri}\\
price: float - il prezzo in dollari della entry di pagamento.\\\\
\textbf{Valore Restituito}\\
entry id: int - l'id dell'entry inserita in blockchain (-1 in caso di errore).\\\\
\textbf{Comportamento}\\
Il metodo inserisce una nuova entry di pagamento in blockchain con il prezzo specificato.\\
Sia in caso si successo che di errore tutte le informazione vengono salvate nel file \textit{sell.log}.

\section{Diagrammi di sequenza}
\subsection{Main}
\includegraphics[width=1.0\textwidth]{main}
\paragraph{Commenti}
Mostra cosa avviene all'avvio del server.

\subsection{Ascolto eventi del contratto}
\includegraphics[width=1.0\textwidth]{shopContractEventManager}
\paragraph{Commenti}
Mostra la sottoscrizione degli eventi del contratto.

\subsection{Inizializzazione server web}
\includegraphics[width=1.0\textwidth]{server}
\paragraph{Commenti}
Mostra l'inizializzazione delle routes per express.

\subsection{Nuovo oggetto in vendita}
\includegraphics[width=1.0\textwidth]{addedPaymentEntry}
\paragraph{Commenti}
Mostra cosa succede quando viene inserito una nuova entry di pagamento da parte di un e-commerce.

\subsection{Nuova transazione}
\includegraphics[width=1.0\textwidth]{settledPayment}
\paragraph{Commenti}
Mostra cosa succede quando viene create una nuova transazione.

\subsection{Cambio di stato di una transazione}
\includegraphics[width=1.0\textwidth]{statusChanged}
\paragraph{Commenti}
Mostra cosa succede quando una transazione cambia di stato.

\subsection{Pagina transazioni in entrata}
\includegraphics[width=1.0\textwidth]{paymentBySellerPage}
\paragraph{Commenti}
Mostra cosa succede quando un utente richiede la pagina delle transazione in entrata.\\
Le altre pagine utilizzano lo stesso modello e dunque si preferisce evitare diagrammi di sequenza ridondanti.

\end{document}