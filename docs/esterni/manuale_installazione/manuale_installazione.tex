\documentclass[a4paper, 12pt]{article}

\newcommand{\templates}{../../template}
\usepackage[a4paper, margin=2.5cm]{geometry}

\usepackage{enumitem}
\setlist[itemize]{noitemsep}
\setlist[enumerate]{noitemsep}

\let\oldpar\paragraph
\renewcommand{\paragraph}[1]{\oldpar{#1\\}\noindent}
\input{\templates/front_page}
\usepackage{hyperref}
\usepackage{array}
\usepackage{tabularx}

\def\vers#1-#2-#3-#4-#5\\{#1&#2&#3&#4&#5\\\hline}

\newcommand{\addversione}[5]{
	\ifdefined\versioni
		\let\old\versioni
		\renewcommand{\versioni}{#1&#2&#3&#4&#5\\\hline\old}
	\else
		\newcommand{\versioni}{#1&#2&#3&#4&#5\\\hline}
	\fi
}

\newcommand{\setversioni}[1]{\newcommand{\versioni}{#1}}

\newcommand{\makeversioni}{
	\begin{center}
		\begin{tabularx}{\textwidth}{|c|c|c|c|X|}
		\hline
		\textbf{Versione} & \textbf{Data} & \textbf{Persona} & \textbf{Attivtà} & \textbf{Descrizione} \\
		\hline
		\versioni
		\end{tabularx}
	\end{center}
	\clearpage
}
\usepackage{hyperref}
\usepackage{graphicx}
\usepackage{placeins}
\usepackage{listings}
\settitolo{Manuale Installazione Shop Chain}
\setredattori{WinningSoftwareSolution}
\setdestuso{esterno}
\setdescrizione{
Manuale installazione.
}

\begin{document}

\makefrontpage
\tableofcontents
\newpage

\section{Introduzione}
\label{sec:intro}
Il presente documento ha la funzione di descrivere in dettaglio la procedura da seguire per l'installazione di tutte le componenti necessarie per il corretto funzionamento di Shop Chain.
\\È necessario clonare la repository GitHub al seguente indirizzo: \url{https://github.com/iota97/WinningSoftwareSolution}.
\section{Server}
\subsection{Requisiti}
\subsubsection{Node Js}
Installare \href{https://nodejs.org/en/download/}{Node Js} se non è già stato installato precedentemente. Una volta effettuata correttamente l'installazione, è necessario eseguire il seguente comando direttamente nella cartella "Server" per l'installazione delle dipendenze:
\begin{lstlisting}[language=bash]
  $ npm install
\end{lstlisting}
\subsubsection{Provider}
\label{sec:provider}
Per ottenere il provider per la testnet di Polygon Mumbai è necessario registrarsi su \href{https://www.moralis.io}{moralis.io}. Una volta effettuata la registrazione:
\begin{enumerate}
  \item selezionare la sezione "SpeedyNodes" dal menù;
  \item selezionare l'endpoint per Polygon Network;
  \item selezionare il protocollo WS;
  \item copiare l'url dell'endpoint per la testnet Mumbai (Fig. 1).
\end{enumerate}
Nell'url copiato è possibile visualizzare le api key, si tratta infatti del codice evidenziato in grassetto nel seguente esempio:\\\\
wss://speedy-nodes-nyc.moralis.io/\textbf{a8d734a415dd368a3498db63}/polygon/mainnet/ws\\\\
È necessario salvare l'url e le api key per i passi successivi.
\FloatBarrier
\begin{figure}[!h]
\centering
\includegraphics[width=0.6\linewidth]{img/moralis.png}
\caption{Provider su moralis.io.}
\end{figure}
\FloatBarrier
\subsubsection{MariaDB}
\label{sec:db}
Installare il database \href{https://mariadb.org/download/}{MariaDB} se non è gia stato installato precedentemente. Una volta eseguita l'installazione del database, collegarsi con il comando:
\begin{lstlisting}[language=bash]
  $ sudo mysql -u <user> -p <password>
\end{lstlisting}
e sostituire:
\begin{itemize}
  \item \verb|<user>|: con l'user con il quale si desidera creare il database;
  \item \verb|<password>|: con la password dell'user selezionato.
\end{itemize}
Effettuato il collegamento eseguire la seguente query per la creazione del database:
\begin{verbatim}
CREATE DATABASE OnlineStore;
\end{verbatim}
Successivamente creare le tabelle con le seguenti query:
\begin{verbatim}
USE OnlineStore;
DROP TABLE PaymentEntries;
DROP TABLE SettledPayments;
DROP TABLE LastBlockSynced;
CREATE TABLE PaymentEntries (id bigint, ecommerce varchar(255) not null,
price bigint not null, primary key(id));
CREATE TABLE SettledPayments (id bigint, item_id bigint not null,
buyer varchar(255) not null, status int not null, created bigint not null,
confirmed bigint, primary key(id));
CREATE TABLE LastBlockSynced (id int(1), value bigint not null, primary key(id));
INSERT INTO LastBlockSynced (id, value) VALUES (0, 0);
\end{verbatim}
\subsubsection{Metamask}
Per poter utilizzare la webbapp è necessario metamask. Visitare \href{https://www.metamask.io}{metamask.io} per maggiori informazioni.\\\\
Durante la creazione del wallet ricordare di salvare la mnemonic phrase poiché necessaria nei passi successivi.
\paragraph{Configurazione network}\\
Per l'inserimento e la configurazione del network Mumbai nel proprio wallet Metamask è necessario utilizzare i seguenti dati:
\begin{itemize}
\item Network Name: \textbf{Mumbai};
\item New RPC Url: \textbf{https://rpc-mumbai.maticvigil.com/};
\item ChainID: \textbf{80001};
\item Currency Simbol: \textbf{MATIC};
\item Block Explorer URL: \textbf{https://polygonscan.com/}.
\end{itemize}
Una volta inseriti i dati e aggiunto il network, ricordare di selezionare il network.
\subsection{Configurazione}
\label{sec:config_server}
Una volta clonata la repository come spiegato nell'\hyperref[sec:intro]{introduzione}, è necessaria la creazione di un file `.env` dentro la cartella "Server". Il file deve essere strutturato come segue:
\begin{verbatim}
  PORT=8080
  DB_HOST="localhost"
  DB_USER="username"
  DB_PWD="password"
  DB_NAME="OnlineStore"
  API_KEY=xxxxxxxxxxxxxxxxxxxxxxxx
  SERVER_URL="tinyurl.com/xxxxxxxx"
\end{verbatim}
dove ciascun campo indica:
\begin{itemize}
  \item \textbf{PORT}: la porta che si desidera utilizzare;
  \item \textbf{DB\_HOST}: l'host selezionato;
  \item \textbf{DB\_USER}: l'user associato al database;
  \item \textbf{DB\_PWD}: password dell' user utilizzato;
  \item \textbf{DB\_NAME}: il nome del database da utilizzare, in questo caso "OnlineStore";
  \item \textbf{API\_KEY}: le api key, ottenibili come spiegato al punto \hyperref[sec:provider]{2.1.2}.
  \mbox{}
  \item \textbf{SERVER\_URL}: strutturare il server url come segue:
  \begin{verbatim}
    http://<ip_locale>.sslip.io:<port>/
  \end{verbatim}
  dove:
  \begin{itemize}
    \item \verb|<ip_locale>| va sostituito con il proprio IP locale;
    \item \verb|<port>| va sostituito con la porta utilizzata
  \end{itemize}
   Successivamente inserire il server url su \href{https://tinyurl.com/app}{tinyurl} e utilizzare l'url ritornato.
\end{itemize}
\subsection{Avvio server e test}
\label{sec:avvio}
Conclusa la configurazione è possibile far partire il server con il comando:
\begin{lstlisting}[language=bash]
  $ npm start
\end{lstlisting}
Mentre per i test è sufficiente la creazione del database "OnlineStoreTest" inserendo le seguenti query come spiegato al punto \hyperref[sec:db]{2.1.3}:
\begin{verbatim}
CREATE DATABASE OnlineStoreTest;
USE OnlineStoreTest;
DROP TABLE PaymentEntries;
DROP TABLE SettledPayments;
DROP TABLE LastBlockSynced;
CREATE TABLE PaymentEntries (id bigint, ecommerce varchar(255) not null,
price bigint not null, primary key(id));
CREATE TABLE SettledPayments (id bigint, item_id bigint not null,
buyer varchar(255) not null, status int not null, created bigint not null,
confirmed bigint, primary key(id));
CREATE TABLE LastBlockSynced (id int(1), value bigint not null, primary key(id));
INSERT INTO LastBlockSynced (id, value) VALUES (0, 0);
\end{verbatim}
e successivamente eseguire il comando:
\begin{lstlisting}[language=bash]
  $ npm test
\end{lstlisting}
\subsection{Test sulla Web App}
Per eseguire i test sulla web app si utilizza JSCover in modalità proxy. I file Javascript della web app passano attraverso un server proxy che li trasforma e tiene traccia della loro copertura. Per eseguire correttamente il test seguire la seguente procedura:
\begin{itemize}
  \item avviare il server di Shop Chain come spiegato al punto \hyperref[sec:avvio]{2.3};
  \item avviare il server proxy con il comando:
  \begin{lstlisting}[language=bash]
    $ java -jar JSCover-all.jar -ws --proxy --port=3128
    --report-dir=jscoverage --local-storage
  \end{lstlisting}
  è possibile cambiare porta nel caso la porta \verb|3128| sia già occupata;
  \item configurare il proxy aggiungendo come host \verb|localhost-proxy| e impostare come proxy \verb|localhost-proxy:3128|;
  \item per eseguire i test è necessario connettersi a \url{http://localhost-proxy:8080}.
\end{itemize}
\paragraph{Salvataggio dati}\\
JScover tiene traccia di ogni linea di codice eseguita e salva i dati nel localStorage di HTML5. Per salvare i dati nella cartella "jscoverage" è necessario:
\begin{itemize}
  \item caricare la pagina \url{http://localhost-proxy:8080/jscoverage.html};
  \item aprire la tab "Store";
  \item selezionare "Store Report".
\end{itemize}
\paragraph{Cancellazione dati}\\
Essendo i dati dei test cumulativi, quando si desidera effettuare il test da zero è necessario cancellare i dati salvati. Per la cancellazione dei dati è sufficiente caricare la pagina \url{http://localhost-proxy:8080/jscoverage-clear-local-storage.html}.
\paragraph{Visualizzazione report}\\
Per la visualizzazione del report è sufficiente selezionare la tab "Summary" nella seguente pagina:  \url{http://localhost-proxy:8080/jscoverage-clear-local-storage.html}.
\section{Script}
\subsection{Requisiti}
\subsubsection{Python e dipendenze}
Al fine di utilizzare correttamente lo script è necessario:
\begin{itemize}
  \item \href{https://www.python.org/downloads/}{python3.8};
  \item pip3.8;
  \item installare le seguenti dipendenze:
  \begin{itemize}
    \item web3;
    \item python-dotenv;
    \item pytest.
  \end{itemize}
\end{itemize}
\subsection{Configurazione}
È necessario creare un file chiamato `.env` all'interno della stessa cartella contenente lo script \textit{sell.py}. Il file dovrà contenere:
\begin{itemize}
  \item la mnemonic phrase del proprio wallet;
  \item il provider ottenuto al punto \hyperref[sec:provider]{2.1.2}.
\end{itemize}
Di seguito un esempio del formato corretto da utilizzare:
\begin{verbatim}
MNEMONIC=mnemonic phrase del proprio wallet
PROVIDER=provider url
\end{verbatim}
\subsection{Eseguire lo script}
Una volta soddisfatti i requisiti e ultimata la configurazione è possibile procedere con l'esecuzione dello script python. Verranno illustrati due modi per la corretta esecuzione dello script \textit{sell.py}.
\subsubsection{Sell.py da terminale}
È possibile eseguire lo script \textit{sell.py} direttamente da terminale, attraverso il comando:
\begin{lstlisting}[language=bash]
  $ python3 sell.py [item price]
\end{lstlisting}
Sostituire \verb|[item price]| con il prezzo con il quale si desidera vendere il proprio prodotto.
Una volta eseguito lo script, verrà ritornato e stampato l'id e il prezzo del prodotto messo in vendita.
\subsubsection{Script python}
È possibile realizzare un proprio script in python per la vendita di prodotti.\\
La funzione per mettere in vendita il proprio prodotto è \textbf{sell\_item} di \textit{sell.py}. La funzione richiede in input un solo parametro, ovvero il prezzo in dollari del prodotto che si desidera vendere.\\
È dunque necessario includere nel proprio script la funzione come segue:
\begin{verbatim}
from sell import sell_item
\end{verbatim}
\subsection{Test}
Per eseguire il test sullo script è sufficiente eseguire il seguente comando:
\begin{lstlisting}[language=bash]
$ pytest
\end{lstlisting}
\section{Contract}
\subsection{Requisiti}
\subsubsection{Truffle, plugin e librerie}
Per compilare lo smart contract e deployarlo è necessario installare Truffle e alcuni plugin e librerie all'interno della cartella "Contract". Eseguire i seguenti comandi per una corretta installazione:
\begin{lstlisting}[language=bash]
  $ npm install
  $ npm install -g truffle
  $ npm install @truffle/hdwallet-provider
\end{lstlisting}
per le librerie eseguire:
\begin{lstlisting}[language=bash]
  $ npm install @openzeppelin/contracts
  $ npm install @chainlink/contracts
\end{lstlisting}
infine per la verifica e test installare:
\begin{lstlisting}[language=bash]
  $ npm install truffle-plugin-verify
  $ npm install solidity-coverage
\end{lstlisting}
\subsection{Configurazione}
È necessaria la creazione di tre file nella cartella "Contract" per il deployment del contratto:
\begin{itemize}
  \item \textbf{mnemonic.secret}: deve contenere la mnemonic del proprio wallet;
  \item \textbf{providerlink.secret}: deve contenere l'url del provider ottenibile come spiegato al punto \hyperref[sec:provider]{2.1.2}, può essere utilizzato anche l'url in http;
  \item \textbf{apikey.secret}: deve contenere le api key ottenibili come spiegato al punto \hyperref[sec:provider]{2.1.2}.
\end{itemize}
\subsection{Compilazione, deployment e verifica}
\paragraph{Compilazione}\\
Per la compilazione del contract è sufficiente eseguire il seguente comando:
\begin{lstlisting}[language=bash]
  $ truffle compile
\end{lstlisting}
\paragraph{Deployment}\\
Per il deployment del contratto eseguire il seguente comando:
\begin{lstlisting}[language=bash]
  $ truffle deploy --netowrk <YOUR_NETWORK>
\end{lstlisting}
dove \verb|<YOUR_NETWORK>| rappresenta il network che si desidera utilizzare, in questo caso viene utilizzato \verb|polygon_mumbai|. Il network è definito nel file \textit{truffle-config.js}.\\ \\
In caso di errori di timeout cambiare l'url nel file \textit{providerlink.secret} da http a wss e/o eseguire il comando aggiungendo:
\begin{lstlisting}[language=bash]
  $ truffle deploy --netowrk <YOUR_NETWORK> --reset --compile-none
\end{lstlisting}
\paragraph{Verifica}\\
Per la verifica del contratto eseguire il comando:
\begin{lstlisting}[language=bash]
  $ truffle run verify ShopContract --network <YOUR-NETWORK>
\end{lstlisting}
questo comando ritorna un link a PolygonScan nel quale è possibile visualizzare il codice verificato e le ABI del contratto.
\subsection{Test}
Per testare senza copertura del codice eseguire il comando:
\begin{lstlisting}[language=bash]
  $ truffle test --network <YOUR-NETWORK> ./test/shopcontract.js
\end{lstlisting}
mentre per testare con copertura del codice eseguire il comando:
\begin{lstlisting}[language=bash]
  $ sudo truffle run coverage --file="./test/shopcontract.js"
    --solcoverjs ./.solcover.js
\end{lstlisting}
\end{document}
