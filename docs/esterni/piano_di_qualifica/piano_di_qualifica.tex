\documentclass[a4paper, 12pt]{article}
\usepackage{amsmath}
\usepackage{pbox}

\newcommand{\templates}{../../template}
\usepackage[a4paper, margin=2.5cm]{geometry}

\usepackage{enumitem}
\setlist[itemize]{noitemsep}
\setlist[enumerate]{noitemsep}

\let\oldpar\paragraph
\renewcommand{\paragraph}[1]{\oldpar{#1\\}\noindent}
\input{\templates/front_page}
\usepackage{hyperref}
\usepackage{array}
\usepackage{tabularx}

\def\vers#1-#2-#3-#4-#5\\{#1&#2&#3&#4&#5\\\hline}

\newcommand{\addversione}[5]{
	\ifdefined\versioni
		\let\old\versioni
		\renewcommand{\versioni}{#1&#2&#3&#4&#5\\\hline\old}
	\else
		\newcommand{\versioni}{#1&#2&#3&#4&#5\\\hline}
	\fi
}

\newcommand{\setversioni}[1]{\newcommand{\versioni}{#1}}

\newcommand{\makeversioni}{
	\begin{center}
		\begin{tabularx}{\textwidth}{|c|c|c|c|X|}
		\hline
		\textbf{Versione} & \textbf{Data} & \textbf{Persona} & \textbf{Attivtà} & \textbf{Descrizione} \\
		\hline
		\versioni
		\end{tabularx}
	\end{center}
	\clearpage
}

\settitolo{Piano di Qualifica}
\setredattori{Raffaele Oliviero \\ Elia Scandaletti \\ Giovanni Cocco}
\setdestuso{esterno}
\setdescrizione{
Questo documento serve a definire le metriche e i criteri di accettazione dei prodotti.
}

\addversione{0.0.0}{09/01/2021}{Raffaele Oliviero}{Redazione}{Stesura iniziale.}
\addversione{0.0.1}{16/01/2021}{Elia Scandaletti}{Redazione}{Correzione indice di Gulpease.}
\addversione{0.0.2}{16/01/2021}{Giovanni Cocco}{Redazione}{Migliorata la leggibilità.}
\addversione{0.0.3}{04/02/2021}{Giovanni Cocco}{Redazione}{Stesura iniziale sezione software.}
\addversione{0.0.4}{21/02/2021}{Giovanni Cocco}{Redazione}{Aggiunta dashboard al documento.}

\begin{document}

\makefrontpage

\makeversioni

\section{Premessa}
Al fine di garantire la qualità del prodotto saranno fatte due attività: verifica e validazione.
\subsection{Verifica}
Ogni PR verrà verificata con attenzione al rispetto delle metriche definite da questo documento.
\subsection{Validazione}
Il prodotto finale verrà validato confrontandolo con le specifiche definite dall'analisi dei requisiti.
\subsection{Metriche}
Il seguente documento definisce metriche e requisiti minimi per l'accettazione di una PR.\\
Il documento è suddiviso per categoria di prodotti.\\\\
Queste metriche dovranno essere calcolabili in maniera automatica.
\section{Documenti}
\subsection{Metriche}
\subsubsection{Indice di Gulpease}
L'indice Gulpease è un indice di leggibilità di un testo tarato sulla lingua italiana.\\\\
Ha il vantaggio di utilizzare la lunghezza delle parole in lettere anziché in sillabe.\\
Questo semplifica il calcolo automatico.\\
I risultati sono compresi tra 0 e 100. Il valore "100" indica la leggibilità più alta.\\\\
In generale risulta che testi coi seguenti indici sono difficili da leggere.
\begin{itemize}
    \item < 80 per chi ha la licenza elementare;
    \item < 60 per chi ha la licenza media;
    \item < 40 per chi ha un diploma superiore.
\end{itemize}

\[ \text{Indice di Gulpease} = 89 + \frac{300*\text{numero di frasi} - 10*\text{numero di lettere}}{\text{numero di parole}} \]

I parametri saranno calcolati:
\begin{itemize}
	\item frasi: i punti presenti nel testo;
	\item lettere: i caratteri alfanumerici;
	\item parole: le parole.
\end{itemize}

\subsection{Requisiti minimi}
\subsubsection{Indice di Gulpease}
\begin{itemize}
	\item \textbf{Documenti interni}: minimo 40;
	\item \textbf{Documenti esterni}: minimo 50.
\end{itemize}

\subsection{Stato attuale}
\begin{tabular}{|l|l|}\hline
	Data & Stato \\\hline
	12/12/2021 & N/A \\\hline
	22/1/2022 & Creati script per automatizzare il processo\\\hline
	9/2/2022 & Raggiunta copertura 100\%\\\hline
\end{tabular}

\subsection{Aspettative di miglioramento}
\begin{tabular}{|l|l|l|}\hline
	Data & Obiettivo & Data realizzazione attesa \\\hline
	30/11/2121 & Creare script per automatizzare il processo & 30/12/2021\\\hline
	12/1/2022 & Copertura al 100\% & 30/1/2022\\\hline
\end{tabular}



\section{Software}
\subsection{Requisiti minimi}
Il sofware dovrà essere coperto almeno per l'80\% da test come richiesto dal proponente.\\\\
A questo scopo saranno creati appositi test automatici per verificare la correttezza del prodotto al momento della PR.\\
L'insieme dei test verrà usato per assicurarsi di non introdurre regression durante lo sviluppo.

\subsection{Stato attuale}
\begin{tabular}{|l|l|}\hline
	Data & Stato \\\hline
	4/2/2022 & Creato PoC di test per il server (jtest)\\\hline
\end{tabular}

\subsection{Aspettative di miglioramento}
\begin{tabular}{|l|l|l|}\hline
	Data & Obiettivo & Data realizzazione attesa \\\hline
	20/1/2022 & Verificare la possibilità di test automatici del server & 7/2/2022\\\hline
	13/2/2022 & \pbox{20cm}{Verificare la possibilità di test automatici del contratto\\ tramite truffle} & 5/3/2022\\\hline
	26/2/2022 & \pbox{20cm}{Verificare la possibilità di test automatici del JS \\tramite jest} & ?\\\hline
	26/2/2022 & \pbox{20cm}{Verificare la possibilità di test automatici dello script \\python tramite pytest} & ?\\\hline
	26/2/2022 & \pbox{20cm}{Verificare la possibilità di test automatici della WebApp \\tramite ?} & 5/3/2022\\\hline
\end{tabular}


\end{document}