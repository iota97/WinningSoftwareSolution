\documentclass[a4paper, 12pt]{article}
\usepackage{amsmath}

\newcommand{\templates}{../../template}
\usepackage[a4paper, margin=2.5cm]{geometry}

\usepackage{enumitem}
\setlist[itemize]{noitemsep}
\setlist[enumerate]{noitemsep}

\let\oldpar\paragraph
\renewcommand{\paragraph}[1]{\oldpar{#1\\}\noindent}
\input{\templates/front_page}
\usepackage{hyperref}
\usepackage{array}
\usepackage{tabularx}

\def\vers#1-#2-#3-#4-#5\\{#1&#2&#3&#4&#5\\\hline}

\newcommand{\addversione}[5]{
	\ifdefined\versioni
		\let\old\versioni
		\renewcommand{\versioni}{#1&#2&#3&#4&#5\\\hline\old}
	\else
		\newcommand{\versioni}{#1&#2&#3&#4&#5\\\hline}
	\fi
}

\newcommand{\setversioni}[1]{\newcommand{\versioni}{#1}}

\newcommand{\makeversioni}{
	\begin{center}
		\begin{tabularx}{\textwidth}{|c|c|c|c|X|}
		\hline
		\textbf{Versione} & \textbf{Data} & \textbf{Persona} & \textbf{Attivtà} & \textbf{Descrizione} \\
		\hline
		\versioni
		\end{tabularx}
	\end{center}
	\clearpage
}

\settitolo{Piano di Qualifica}
\setredattori{Raffaele Oliviero \\ Elia Scandaletti}
\setdestuso{esterno}
\setdescrizione{
Questo documento serve a definire le metriche e i criteri di accettazione dei prodotti.
}

\addversione{0.0.0}{09/01/2021}{Raffaele Oliviero}{Redazione}{Stesura iniziale}
\addversione{0.0.1}{16/01/2021}{Elia Scandaletti}{Redazione}{Correzione indice di Gulpease}

\begin{document}

\makefrontpage

\makeversioni

\section{Premessa}
Il seguente documento definisce le metriche e i requisiti minimi per l'accettazione di una pull request sulla repository del progetto. 

Ogni sezione definisce le metriche e i requisiti minimi di una categoria di prodotto.

\section{Documenti}
\subsection{Metriche}
\subsubsection{Indice di Gulpease}
Per il calcolo della leggibilità con:
\[ \text{Indice di Gulpease} = 89 + \frac{300*\text{numero di frasi} - 10*\text{numero di lettere}}{\text{numero di parole}} \]


\subsection{Requisiti minimi}
\subsubsection{Indice di Gulpease}
\begin{itemize}
	\item \textbf{Documenti interni}: minimo 40
	\item \textbf{Documenti esterni}: minimo 60
\end{itemize}

\section{Software}
Per ogni metrica vengono definiti requisiti minimi per accettare un prodotto nella repository e i requisiti minimi per rilasciare una versione
	


\end{document}