\documentclass[a4paper, 12pt]{article}

\usepackage{graphicx}
\usepackage{longtable}
\graphicspath{ {images/} }

\newcommand{\templates}{../../template}
\usepackage[a4paper, margin=2.5cm]{geometry}

\usepackage{enumitem}
\setlist[itemize]{noitemsep}
\setlist[enumerate]{noitemsep}

\let\oldpar\paragraph
\renewcommand{\paragraph}[1]{\oldpar{#1\\}\noindent}
\input{\templates/front_page}
\usepackage{hyperref}
\usepackage{array}
\usepackage{tabularx}

\def\vers#1-#2-#3-#4-#5\\{#1&#2&#3&#4&#5\\\hline}

\newcommand{\addversione}[5]{
	\ifdefined\versioni
		\let\old\versioni
		\renewcommand{\versioni}{#1&#2&#3&#4&#5\\\hline\old}
	\else
		\newcommand{\versioni}{#1&#2&#3&#4&#5\\\hline}
	\fi
}

\newcommand{\setversioni}[1]{\newcommand{\versioni}{#1}}

\newcommand{\makeversioni}{
	\begin{center}
		\begin{tabularx}{\textwidth}{|c|c|c|c|X|}
		\hline
		\textbf{Versione} & \textbf{Data} & \textbf{Persona} & \textbf{Attivtà} & \textbf{Descrizione} \\
		\hline
		\versioni
		\end{tabularx}
	\end{center}
	\clearpage
}

\settitolo{Analisi dei requisiti}
\setprogetto{ShopChain}
\setcommittenti{SyncLab}
\setredattori{Raffaele Oliviero}
\setdestuso{esterno}
\setdescrizione{
Lista Metodi
}

\addversione{0.0.0}{23/2/2022}{Raffaele Oliviero}{Redazione}{Strutturazione del documento}

\begin{document}

\makefrontpage

\makeversioni

\section{Introduzione}
\subsection{Scopo del documento}
Il documento elenca i metodi pubblici di ogni classe, descrivendone i loro parametri di invocazione, i loro return e il comportamento.

\section{Lista dei metodi pubblici}
\subsection{Persistence}

\subsubsection{getPaymentByBuyer(buyer)}
Restituisce la lista delle transazioni il cui acquirente corrisponde a \texttt{buyer}. \\
\texttt{buyer} è una stringa.

\subsubsection{getPaymentBySeller()}
\textbf{getPaymentBySeller(seller)} \\
Restituisce la lista delle transazioni il cui venditore corrisponde a \texttt{seller}. \\
\texttt{seller} è una stringa.

\subsubsection{getPaymentByID()}
\textbf{getPaymentByID(id)} \\
Restituisce la transazione il cui ID corrisponde a \texttt{id} o, se esso non esiste, restituisce un messaggio di transazione inesistente \\
\texttt{id} è un intero.

\subsubsection{getPaymentEntryByID()}
\textbf{getPaymentEntryByID(id)} \\
Restituisce ID, venditore e prezzo dell'articolo il cui ID corrisponde a \texttt{id} o, se esso non esiste, restituisce un messaggio di articolo inesistente \\
\texttt{buyer} è un intero.

\subsection{Server}
\subsubsection{close()}
Interrompe la connessione al server

\subsection{PageCreator}

\end{document}