\documentclass[a4paper, 12pt]{article}

\usepackage{graphicx}
\usepackage{longtable}
\graphicspath{ {images/} }

\newcommand{\templates}{../../template}
\usepackage[a4paper, margin=2.5cm]{geometry}

\usepackage{enumitem}
\setlist[itemize]{noitemsep}
\setlist[enumerate]{noitemsep}

\let\oldpar\paragraph
\renewcommand{\paragraph}[1]{\oldpar{#1\\}\noindent}
\input{\templates/front_page}
\usepackage{hyperref}
\usepackage{array}
\usepackage{tabularx}

\def\vers#1-#2-#3-#4-#5\\{#1&#2&#3&#4&#5\\\hline}

\newcommand{\addversione}[5]{
	\ifdefined\versioni
		\let\old\versioni
		\renewcommand{\versioni}{#1&#2&#3&#4&#5\\\hline\old}
	\else
		\newcommand{\versioni}{#1&#2&#3&#4&#5\\\hline}
	\fi
}

\newcommand{\setversioni}[1]{\newcommand{\versioni}{#1}}

\newcommand{\makeversioni}{
	\begin{center}
		\begin{tabularx}{\textwidth}{|c|c|c|c|X|}
		\hline
		\textbf{Versione} & \textbf{Data} & \textbf{Persona} & \textbf{Attivtà} & \textbf{Descrizione} \\
		\hline
		\versioni
		\end{tabularx}
	\end{center}
	\clearpage
}

\settitolo{Analisi dei requisiti}
\setprogetto{ShopChain}
\setcommittenti{SyncLab}
\setredattori{Raffaele Oliviero}
\setdestuso{esterno}
\setdescrizione{
Lista Metodi
}

\addversione{0.0.0}{23/2/2022}{Raffaele Oliviero}{Redazione}{Strutturazione del documento}
\addversione{0.0.1}{04/3/2022}{Andrea Volpe}{Redazione}{Inserimento metodi pubblici}

\begin{document}

\makefrontpage

\makeversioni

\section{Introduzione}
\subsection{Scopo del documento}
Il documento elenca i metodi pubblici di ogni classe, descrivendone i loro parametri di invocazione, i loro return e il comportamento.

\section{Lista dei metodi pubblici}
\subsection{Persistence}

%variabili
%\newcommand\linkServerManager{https://github.com/iota97/WinningSoftwareSolution/blob/main/server/Server/ServerManager.ts?}

\subsubsection{getPaymentByBuyer(buyer)}
Restituisce la lista delle transazioni il cui acquirente corrisponde a \texttt{buyer}. \\
\texttt{buyer} è una stringa.

\subsubsection{getPaymentBySeller()}
\textbf{getPaymentBySeller(seller)} \\
Restituisce la lista delle transazioni il cui venditore corrisponde a \texttt{seller}. \\
\texttt{seller} è una stringa.

\subsubsection{getPaymentByID()}
\textbf{getPaymentByID(id)} \\
Restituisce la transazione il cui ID corrisponde a \texttt{id} o, se esso non esiste, restituisce un messaggio di transazione inesistente \\
\texttt{id} è un intero.

\subsubsection{getPaymentEntryByID()}
\textbf{getPaymentEntryByID(id)} \\
Restituisce ID, venditore e prezzo dell'articolo il cui ID corrisponde a \texttt{id} o, se esso non esiste, restituisce un messaggio di articolo inesistente \\
\texttt{buyer} è un intero.

\subsection{ServerManager}
class ServerManager
\subsubsection{setContract(shopContract) : ServerManager}
Imposta un contratto\\
\textbf{Parametri}\\
shopContract : ShopContract\_Interface\\
\textbf{Valore Restituito}\\
%\underline{\href{\linkServerManager}{ServerManager}}\\
ServerManager\\
\subsubsection{setSQL(sql) : ServerManager}
Imposta una connessione SQL\\
\textbf{Parametri}\\
sql : SQL\_Interface\\
\textbf{Valore Restituito}\\
ServerManager
\subsubsection{setPageCreator(page): ServerManager}
Imposta un PageCreator\\
\textbf{Parametri}\\
page : PageCreator\\
\textbf{Valore Restituito}\\
ServerManager
\subsubsection{start() : ServerManager}
Crea e avvia un nuovo server\\
\textbf{Parametri}\\
\textbf{Valore Restituito}\\
ServerManager\\
\textbf{Eccezioni}\\
AssertionError?
\subsubsection{closeServer() : void}
Chiude la connessione con il server\\
\textbf{Parametri}\\
\textbf{Valore Restituito}\\
void

\subsection{SQL}
class SQL implements SQL\_Interface
\subsubsection{closeConnection(): void}
Chiude la connessione con il database\\
\textbf{Parametri}\\
\textbf{Valore Restituito}\\
void
\subsubsection{insertPaymentEntry(entry): Promise$<$void$>$}
Inserisce un nuovo prodotto nel database\\
\textbf{Parametri}\\
entry: paymentEntry\\
\textbf{Valore Restituito}\\
Promise$<$void$>$
\subsubsection{insertSettledPayment(entry): Promise$<$void$>$}
Inserisce un nuovo pagamento nel database\\
\textbf{Parametri}\\
entry: settledPayment\\
\textbf{Valore Restituito}\\
Promise$<$void$>$
\subsubsection{updateSettledPayment(id, status, timestamp): Promise$<$void$>$}
Aggiorna un pagamento nel database\\
\textbf{Parametri}\\
id: bigint,\\ 
status: number,\\
timestamp: bigint\\
\textbf{Valore Restituito}\\
Promise$<$void$>$
\subsubsection{getPaymentByBuyer(buyer): Promise$<$payment[ ]$>$}
Restituisce i pagamenti di uno specifico acquirente\\
\textbf{Parametri}\\
buyer: string\\
\textbf{Valore Restituito}\\
Promise$<$payment[ ]$>$
\subsubsection{getPaymentBySeller(seller): Promise$<$payment[ ]$>$}
Restituisce i pagamenti di uno specifico venditore\\
\textbf{Parametri}\\
seller: string\\
\textbf{Valore Restituito}\\
Promise$<$payment[ ]$>$
\subsubsection{getPaymentEntryByID(id): Promise$<$paymentEntry$>$}
Restituisce uno specifico prodotto\\
\textbf{Parametri}\\
id: bigint\\
\textbf{Valore Restituito}\\
Promise$<$paymentEntry$>$
\subsubsection{getPaymentByID(id): Promise$<$payment$>$}
Restituisce uno specifico pagamento\\
\textbf{Parametri}\\
id: bigint\\
\textbf{Valore Restituito}\\
Promise$<$payment$>$
\subsubsection{getLastSyncBlock(): Promise$<$number$>$}
Restituisce l'ultimo blocco sincronizzato\\
\textbf{Parametri}\\
\textbf{Valore Restituito}\\
Promise$<$number$>$
\subsubsection{setLastSyncBlock(block): Promise $<$void$>$}
Imposta l'ultimo blocco sincronizzato al valore passato come parametro\\
\textbf{Parametri}\\
block: number\\
\textbf{Valore Restituito}\\
Promise$<$void$>$

\subsection{Server}
class Server
\subsubsection{close(): void}
Chiude la connessione con il server\\
\textbf{Parametri}\\
\textbf{Valore Restituito}\\
void

\subsection{PageCreator}
class PageCreator
\subsubsection{landPage(req, res, db): void} 
Renderizza e invia la pagina \texttt{landing page} al client\\
\textbf{Parametri}\\
req: Request,\\
res: Response,\\
db: Persistence\\
\textbf{Valore Restituito}\\
void
\subsubsection{helpPage(req, res): void}
Renderizza e invia la pagina \texttt{help} al client\\
\textbf{Parametri}\\
req: Request,\\
res: Response\\
\textbf{Valore Restituito}\\
void
\subsubsection{mainPage(req, res): void}
Renderizza e invia la pagina \texttt{principale} al client\\
\textbf{Parametri}\\
req: Request,\\
res: Response\\
\textbf{Valore Restituito}\\
void
\subsubsection{confirmPage(req, res, db): void}
Renderizza e invia la pagina \texttt{?} al client\\
\textbf{Parametri}\\
req: Request,\\
res: Response,\\
db: Persistence\\
\textbf{Valore Restituito}\\
void
\subsubsection{paymentByBuyerPage(req, res, db): void}
Renderizza e invia la pagina delle \texttt{transazioni effettuate} al client\\
\textbf{Parametri}\\
req: Request,\\
res: Response,\\
db: Persistence\\
\textbf{Valore Restituito}\\
void
\subsubsection{detailPage(req, res, db): void}
Renderizza e invia la pagina dei \texttt{dettagli transazione?} al client\\
\textbf{Parametri}\\
req: Request,\\
res: Response,\\
db: Persistence\\
\textbf{Valore Restituito}\\
void
\subsubsection{paymentBySellerPage(req, res, db): void}
Renderizza e invia la pagina delle \texttt{transazioni in ingresso} al client\\
\textbf{Parametri}\\
req: Request,\\
res: Response,\\
db: Persistence\\
\textbf{Valore Restituito}\\
void

\end{document}