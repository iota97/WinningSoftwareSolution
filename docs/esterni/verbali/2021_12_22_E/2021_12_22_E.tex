\documentclass[a4paper, 12pt]{article}

\newcommand{\templates}{../../../template}
\usepackage[a4paper, margin=2.5cm]{geometry}

\usepackage{enumitem}
\setlist[itemize]{noitemsep}
\setlist[enumerate]{noitemsep}

\let\oldpar\paragraph
\renewcommand{\paragraph}[1]{\oldpar{#1\\}\noindent}
\input{\templates/front_page}
\usepackage{hyperref}
\usepackage{array}
\usepackage{tabularx}

\def\vers#1-#2-#3-#4-#5\\{#1&#2&#3&#4&#5\\\hline}

\newcommand{\addversione}[5]{
	\ifdefined\versioni
		\let\old\versioni
		\renewcommand{\versioni}{#1&#2&#3&#4&#5\\\hline\old}
	\else
		\newcommand{\versioni}{#1&#2&#3&#4&#5\\\hline}
	\fi
}

\newcommand{\setversioni}[1]{\newcommand{\versioni}{#1}}

\newcommand{\makeversioni}{
	\begin{center}
		\begin{tabularx}{\textwidth}{|c|c|c|c|X|}
		\hline
		\textbf{Versione} & \textbf{Data} & \textbf{Persona} & \textbf{Attivtà} & \textbf{Descrizione} \\
		\hline
		\versioni
		\end{tabularx}
	\end{center}
	\clearpage
}

\settitolo{Verbale del 22/12/2021 con SyncLab}
\setredattori{Raffaele Oliviero}
\setrevisori{Giovanni Cocco}
\setdestuso{interno}
\setdescrizione{
Verbale dell'incontro del 22/12/2021 con l'azienda SyncLab. 
}

\begin{document}

\makefrontpage

\section{Informazioni}
\textbf{Data}: 22/12/2021\\
\textbf{Durata}: 1 ora\\
\textbf{Luogo}: Aula Virtuale Google Meet\\

\textbf{Partecipanti}
\begin{itemize}
	\item Andrea Volpe
	\item Giovanni Cocco
	\item Alberto Nicoletti
	\item Matteo Galvagni
	\item Raffaele Oliviero
	\item Fabio Pallaro
\end{itemize}

\section{Ordine del giorno}
\begin{enumerate}
	\item Varie ed eventuali;
	\item Domande da parte del gruppo al proponente.
\end{enumerate}


\section{Svolgimento}
\subsection{Utilizzo di Discord}
Si è deciso di spostare le call sul server Discord dell'azienda e di pubblicare i verbali all'interno del canale testuale dedicato utilizzando la funzione thread di Discord

\subsection{Validazione della transazione}
Si è discusso su due possibili opzioni per validare e tenere traccia delle transazioni
	\begin{itemize}
		\item \textbf{Opzione 1}: L'intero processo viene gestito da Shopchain, quindi tutto dipende dal funzionamento dei suoi server, inoltre le operazioni non sono trasparenti finchè non vengono immesse in blockchain e questo potrebbe scoraggiare potenziali acquirenti;
		\item \textbf{Opzione 2}: Il database è gestito dall'e-commerce che effettuerà delle chiamate alla blockchain, questo richiederà modificare il backend del venditore.
	\end{itemize}
Durante la discussione sono anche stati trattati i seguenti argomenti: 
\subsubsection{Landing page}
Quando l'utente raggiunge la landing page, Shopchain creerà lo smart contract inserendo l'id del prodotto e l'address del venditore nella blockchain, il cliente inserirà l'importo e firmerà il contratto con Metamask.  L'importo dovrà essere superiore a 5€, altrimenti la transazione non proseguirà.
\subsubsection{Log transazioni}
Il venditore deve essere in grado di visualizzare un resoconto di tutte le transazioni in corso e di annularle su richiesta
\subsection{PoC}
Il proponente ha richiesto una Proof of Concept che contenga:
	\begin{itemize}
		\item Una landing page in grado di ricevere i dati e creare uno smart contract;
		\item Un simulatore dello sblocco dei contratti.
	\end{itemize}  
\subsection{Pagamenti in altre valute}
La conversione in valuta stabile, anche se non strettamente obbligatorio, è un requisito ritenuto importante, inoltre si è deciso di non considerare l'utilizzo di ogni possibile criptovaluta.

\section{Conclusioni}
Sono state dettagliate delle soluzioni al problema della gestione delle transazioni e ad altri problemi sorti durante la discussione fornendo una visione più delineata del prodotto che si andrà a creare.

\end{document}