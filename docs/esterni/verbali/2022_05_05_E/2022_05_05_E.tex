\documentclass[a4paper, 12pt]{article}

\newcommand{\templates}{../../../template}
\usepackage[a4paper, margin=2.5cm]{geometry}

\usepackage{enumitem}
\setlist[itemize]{noitemsep}
\setlist[enumerate]{noitemsep}

\let\oldpar\paragraph
\renewcommand{\paragraph}[1]{\oldpar{#1\\}\noindent}
\input{\templates/front_page}
\usepackage{hyperref}
\usepackage{array}
\usepackage{tabularx}

\def\vers#1-#2-#3-#4-#5\\{#1&#2&#3&#4&#5\\\hline}

\newcommand{\addversione}[5]{
	\ifdefined\versioni
		\let\old\versioni
		\renewcommand{\versioni}{#1&#2&#3&#4&#5\\\hline\old}
	\else
		\newcommand{\versioni}{#1&#2&#3&#4&#5\\\hline}
	\fi
}

\newcommand{\setversioni}[1]{\newcommand{\versioni}{#1}}

\newcommand{\makeversioni}{
	\begin{center}
		\begin{tabularx}{\textwidth}{|c|c|c|c|X|}
		\hline
		\textbf{Versione} & \textbf{Data} & \textbf{Persona} & \textbf{Attivtà} & \textbf{Descrizione} \\
		\hline
		\versioni
		\end{tabularx}
	\end{center}
	\clearpage
}

\settitolo{Verbale del 05/05/2022 con SyncLab}
\setredattori{Alberto Nicoletti}
\setrevisori{Elia Scandaletti}
\setdestuso{interno}
\setdescrizione{
Verbale dell'incontro del 05/05/2022 con l'azienda SyncLab.
}

\begin{document}

\makefrontpage

\section{Informazioni}
\textbf{Data}: 05/05/2022\\
\textbf{Durata}: 45 minuti\\
\textbf{Luogo}: Call Google Meet\\

\textbf{Partecipanti}
\begin{itemize}
	\item Giovanni Cocco;
	\item Matteo Galvagni;
	\item Federico Marchi;
	\item Alberto Nicoletti;
	\item Elia Scandaletti;
	\item Andrea Volpe;
	\item Fabio Pallaro.
\end{itemize}


\section{Ordine del giorno}
\begin{enumerate}
	\item Illustrazione del prodotto completo;
	\item prova d'esposizione per il colloquio CA.
\end{enumerate}

\section{Svolgimento}
\subsection{Illustrazione del prodotto completo}
Il proponente si è mostrato soddisfatto del risultato ottenuto dal gruppo. Il prodotto rispecchia le aspettative.
\\Riguardo l'interfaccia grafica è stata sollevata una considerazione da Fabio Pallaro. E' spesso preferibile differenziare l'interfaccia delle varie categorie d'utente. Nel nostro caso le grafiche della parte acquirente e della parte venditore sono molto simili.

\subsection{Prova d'esposizione per il colloquio CA}
Nella prova d'esposizione si è mostrato il funzionamento del prodotto in tutte le sue parti. Durante la dimostrazione il pagamento non è andato a buon fine. Questo problema è dovuto all'RPC settato nella network di Metamask. La soluzione è utilizzare un RPC privato.
\\Il server è in locale sulla macchina dei componenti. Ciò significa che durante la dimostrazione l'acquirente ed il venditore devono essere simulati dalla stessa persona. Così è stato fatto anche durante il colloquio PB con il docente Cardin. Il proponente ha sottolineato che sarebbe preferibile trovare delle alternative in modo da differenziare i due utenti di Shop Chain. Le soluzioni discusse sono: 
\begin{itemize}
	\item server in hosting da parte di un componente del gruppo;
	\item un'unica persona può usare più account Metamask.
\end{itemize}

\subsection{Altro}
È stato chiesto al proponente se tenere come timer per la scadenza della transazione 5 minuti o 14 giorni per il colloquio CA.
\begin{itemize}
	\item Mettere la scadenza a 14 giorni significa mostrare il prodotto esattamente come verrà consegnato. Non ci saranno modifiche tra la dimostrazione in CA e la consegna. Durante l'esposizione non si potrà però mostrare il caso di transazione scaduta.
	\item Tenere la scadenza a 5 minuti permette di dimostrare tutte le funzionalità del prodotto. La differenza tra prodotto mostrato e prodotto consegnato è minima. Un solo parametro.
\end{itemize}
Il proponente ha chiesto e consigliato che venga tenuta la scadenza a 5 minuti. In questo modo si potrà mostrare il prodotto nella sua completezza.

\section{Impegni presi}
\begin{itemize}
\item Mantenere la scadenza transazione a 5 minuti.
\item Utilizzare un RPC privato.
\end{itemize}

\end{document}
