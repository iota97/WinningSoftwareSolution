\documentclass[a4paper, 12pt]{article}

\newcommand{\templates}{../../template}
\usepackage[a4paper, margin=2.5cm]{geometry}

\usepackage{enumitem}
\setlist[itemize]{noitemsep}
\setlist[enumerate]{noitemsep}

\let\oldpar\paragraph
\renewcommand{\paragraph}[1]{\oldpar{#1\\}\noindent}
\input{\templates/front_page}
\usepackage{hyperref}
\usepackage{array}
\usepackage{tabularx}

\def\vers#1-#2-#3-#4-#5\\{#1&#2&#3&#4&#5\\\hline}

\newcommand{\addversione}[5]{
	\ifdefined\versioni
		\let\old\versioni
		\renewcommand{\versioni}{#1&#2&#3&#4&#5\\\hline\old}
	\else
		\newcommand{\versioni}{#1&#2&#3&#4&#5\\\hline}
	\fi
}

\newcommand{\setversioni}[1]{\newcommand{\versioni}{#1}}

\newcommand{\makeversioni}{
	\begin{center}
		\begin{tabularx}{\textwidth}{|c|c|c|c|X|}
		\hline
		\textbf{Versione} & \textbf{Data} & \textbf{Persona} & \textbf{Attivtà} & \textbf{Descrizione} \\
		\hline
		\versioni
		\end{tabularx}
	\end{center}
	\clearpage
}
\usepackage{hyperref}
\usepackage{graphicx}
\usepackage{placeins}
\settitolo{Manuale E-commerce Shop Chain}
\setredattori{WinningSoftwareSolution}
\setdestuso{esterno}
\setdescrizione{
Manuale e-commerce.
}

\begin{document}

\makefrontpage
\tableofcontents
\newpage

\section{Introduzione}
Attraverso l'utilizzo della blockchain \textit{Polygon}, Shop Chain fornisce un sistema decentralizzato per il pagamento di prodotti in MATIC. Utilizzando Shop Chain è possibile effettuare una qualsiasi attività di compravendita in maniera sicura per entrambi gli attori dello scambio. L'acquirente che desidera acquistare un prodotto da un venditore in un e-commerce potrà effettuare il suo pagamento in MATIC, ovvero il token nativo della blockchain \textit{Polygon}. Il venditore non può disporre immediatamente del saldo ricevuto poiché sarà bloccato fino alla ricezione del pacco, momento nel quale l'acquirente può sbloccare i fondi attraverso la scannerizzazione di un codice QR presente sulla scatola.
Il presente documento ha la funzione di descrivere in dettaglio la procedura da seguire per l'e-commerce al fine di vendere correttamente i propri prodotti.

\section{Vendita di prodotti}
\subsection{Wallet Metamask}
Per poter vendere i propri prodotti sarà necessario utilizzare un wallet Metamask. Visitare \href{https://www.metamask.io}{metamask.io} per maggiori informazioni.
\subsubsection{Configurazione}
Se il network \textit{Polygon} non è gia stato inserito nel proprio wallet Metamask, procedere con la lettura del seguente paragrafo.
Per l'inserimento e la configurazione del network \textit{Polygon} nel proprio wallet Metamask è necessario utilizzare i seguenti dati:
\begin{itemize}
\item Network Name: \textbf{Matic Mainnet};
\item New RPC Url: \textbf{https://rpc-mainnet.maticvigil.com/};
\item ChainID: \textbf{137};
\item Currency Simbol: \textbf{MATIC};
\item Block Explorer URL: \textbf{https://explorer.matic.network/}.
\end{itemize}
Una volta inseriti i dati e aggiunto il network, ricordare di selezionare il network.

\subsubsection{Collegamento del wallet a Shop Chain}
Una volta inserito correttamente il network è necessario collegare il proprio wallet metamask a Shop Chain. Nella webapp di Shop Chain, se non si ha già collegato in precedenza il proprio wallet:
\begin {itemize}
\item selezionare "Connect Wallet" all'apertura del pop-up (Fig. 1);
\item approvare il collegamento dal proprio wallet Metamask.
\end{itemize}

\FloatBarrier
\begin{figure}
\centering
\includegraphics[width=0.5\linewidth]{img/connessione_wallet.png}
\caption{Interfaccia per la connessione del wallet Metamask.}
\end{figure}
\FloatBarrier

\section{Script per aggiunta nuovi prodotti}
Per l'aggiunta di un nuovo prodotto in blockchain si rende disponibile uno script python.
\subsection{Configurazione}
Creare un file chiamato `.env` con:
\begin{enumerate}
    \item 
MNEMONIC=il memonic del proprio wallet
\item 
PROVIDER=wss://speedy-nodes-nyc.moralis.io/5c29520422f9528344aa64a1/polygon/mumbai/ws
\end{enumerate}

\subsection{Utilizzo}
Lo script richiede solamente di inserire il prezzo in dollari del prodotto che si vuole aggiungere.\\
Lo script ritorna l'id che rappresenta il nuovo prodotto. Questo id andrà utilizzato ogni volta che si va a creare una nuova transazione di pagamento del relativo prodotto.

\section{Operazioni sulle transazioni}
Se si è connessi alla webapp ShopChain con il proprio wallet Metamask, è possibile visualizzare la lista delle transazioni.
\\Nella pagina outgoing sono presenti le transazioni d'acquisto(vedi manuale utente acquirente).
\\Nella pagina incoming sono presenti le transazioni in entrata.
\\\\Cliccando su ogni transazione è possibile visualizzare le informazioni:
\begin{itemize}
\item Wallet acquirente
\item id prodotto
\item prezzo
\end{itemize}
Per ogni transazione viene visualizzato il suo stato:
\begin{itemize}
    \item 0=cancellata.
    \item 1=in attesa. L'acquirente non ha ancora scansionato il QR Code
    \item 2=completata. L'acquirente ha confermato la ricezione del prodotto scansionando il QR Code.
\end{itemize}
Per ogni transazione in attesa è possibile eseguire due operazioni:
\begin{itemize}
    \item Cancella: si annulla la transazione ed il denaro viene restituito all'acquirente.
    \item Visualizza Qr Code: per stampare e applicare sul pacco del prodotto il Qr Code che verrà scansionato dall'acquirente.
\end{itemize}

\end{document}
