\documentclass[a4paper, 12pt]{article}

\usepackage{graphicx}
\graphicspath{ {images/} }

\newcommand{\templates}{../../template}
\usepackage[a4paper, margin=2.5cm]{geometry}

\usepackage{enumitem}
\setlist[itemize]{noitemsep}
\setlist[enumerate]{noitemsep}

\let\oldpar\paragraph
\renewcommand{\paragraph}[1]{\oldpar{#1\\}\noindent}
\input{\templates/front_page}
\usepackage{hyperref}
\usepackage{array}
\usepackage{tabularx}

\def\vers#1-#2-#3-#4-#5\\{#1&#2&#3&#4&#5\\\hline}

\newcommand{\addversione}[5]{
	\ifdefined\versioni
		\let\old\versioni
		\renewcommand{\versioni}{#1&#2&#3&#4&#5\\\hline\old}
	\else
		\newcommand{\versioni}{#1&#2&#3&#4&#5\\\hline}
	\fi
}

\newcommand{\setversioni}[1]{\newcommand{\versioni}{#1}}

\newcommand{\makeversioni}{
	\begin{center}
		\begin{tabularx}{\textwidth}{|c|c|c|c|X|}
		\hline
		\textbf{Versione} & \textbf{Data} & \textbf{Persona} & \textbf{Attivtà} & \textbf{Descrizione} \\
		\hline
		\versioni
		\end{tabularx}
	\end{center}
	\clearpage
}

\settitolo{Analisi dei requisiti}
\setprogetto{ShopChain}
\setcommittenti{SyncLab}
\setredattori{Alberto Nicoletti, Andrea Volpe}
\setdestuso{esterno}
\setdescrizione{
Questo documento contiene..
}

\addversione{0.0.0}{11/12/2021}{Alberto Nicoletti}{Redazione}{Struttura del documento, stesura sezioni 1 e 2}
\addversione{0.0.1}{30/12/2021}{Alberto Nicoletti}{Redazione}{Stesura casi d'uso da UC1 a UC5.4}

\begin{document}

\makefrontpage

\makeversioni

\section{Introduzone}
\subsection{Scopo del documento}
Lo scopo del documento è raccogliere i risultati dell'attività di analisi dei requisiti. Contiene quindi la descrizione dei casi d'uso del prodotto software da sviluppare, ed i requisiti suddivisi per tipologia. Si vuole così dimostrare una completa comprensione del problema e delle aspettative  nella soluzione. Casi d'uso e requisiti saranno tenunuti in considerazione nella fase di progettazione, assicurandosi la soddisfazione dei requisiti. 
\subsection{Glossario}
Di seguito un elenco di parole usate in questo documento, la cui definizione può essere trovata nel Glossario.
parola, parola, parola.

\section{Descrizione del prodotto}
L'azieda \textit{SyncLab} propone, attraverso il capitolato C2: \textit{ShopChain - Exchange Platform on
BlockChain}. Consiste nel realizzare un prototipo di una piattaforma in grado di ‘affiancare’ un crypto-e-commerce nella fasi di pagamento fino alla consegna.
\subsection{Scopo del prodotto}
Il progetto consiste nello sviluppo di una piattaforma su blockchain con lo scopo di rendere possibile e in sicurezza l'acquisto di prodotti tramite criptovalute. Il processo di trasferimento del denaro avviene seguendo queste fasi:
\begin{enumerate}
\item Caricamento dei dati dell'ordine di acquisto nella blockchain e generazione dello smart contract.
\item Trasferimento del denaro dal wallet dell'acquirente in blockchain.
\item Notifica al venditore dell'avvenuto pagamento e blocco del denaro in blockchain.
\item Conferma di ricezione del pacco da parte dell'acquirente tramite scannerizzazione di un Qr Code nel prodotto acquistato.
\item Sblocco del denaro e trasferimento nel wallet del venditore.
\end{enumerate}
\subsection{Parti del prodotto}
Il prodotto software è composta dalle seguenti parti:
\begin{itemize}
\item Smart contracts nella blockchain per la gestione di tutte le fasi del processo di trasferimento del denaro.
\item Piattaforma web per la gestione dei pagamenti da parte del venditore.
\item App o web app per la scannerizzazione del Qr code .
\end{itemize}
\subsection{Caratteristiche utenti}
Gli utenti di \textit{ShopChain} possono essere suddivisi in due categorie:
\begin{itemize}
\item Venditore: gli amministratori di un sito di e-commerce che vogliono aggiungere le criptovalute come metodo di pagamento.
\item Acquirente: I clienti di un sito di e-commerce che scelgono di utilizzare le criptovalute come pagamento per i prodotti da acquistare.
\end{itemize}
Tutti gli utenti sono in possesso di un wallet per criptovalute. 
Non potendo prevedere con accuratezza quanti e quali e-commerce decideranno di utilizzare \textit{ShopChain} altre considerazioni sulle caratteristiche di utenza sono irrilevanti.
\subsection{Vincoli e preferenze}
La proponente non impone vincoli nella scelta della tipologia di tecnologie, va ci sono comunque delle scelte preferenziali da considerare:
\begin{itemize}
\item Utilizzo di blockchain pubblica
\item utilizzo di Java e Angular per lo sviluppo delle parti di Back-end e di Front-end della componente web application del sistema
\item utilizzo di database Postgres
\end{itemize}

Per il completamento del progetto la proponente richiede che siano realizzati i seguenti risultati:
\begin{itemize}
\item server, completo di UI
\item test che dimostrino il corretto funzionamento dei servizi e delle funzionalità previste, con una copertura minima dell' 80\% correlata di report.
\item documentazione su: scelte implementative e progettuali effettuate e relative motivazioni, problemi aperti e eventuali soluzioni proposte da esplorare.
\end{itemize}

\section{Casi d'uso}

\includegraphics[width=0.9\textwidth]{uc01}

\paragraph{UC1 - Installazione ShopChain}
\textbf{Attore primario}: Venditore\\
\textbf{Precondizioni}: Il venditore ha stipulato un accordo con WinningSoftwareSolution per aggiungere ShopChain come metodo di pagamento sul propiro e-commerce.\\
\textbf{Postcondizioni}: ShopChain è inserito come metodo di pagamento sull'e-commerce del venditore.\\
\textbf{Scenario principale}:
Il venditore proprietario di un e-commerce aggiunge ShopChain come metodo di pagamento.

\paragraph{UC1.1 - Selezione Wallet}
\textbf{Attore primario}: Venditore\\
\textbf{Attore secondario}: Metamask\\
\textbf{Precondizioni}: Il venditore sta aggiungendo ShopChain come metodo di pagamento al suo e-commerce.\\
\textbf{Postcondizioni}: Il wallet del venditore è configurato come destinazione delle transazioni con ShopChain.\\
\textbf{Scenario principale}: Il venditore collega il proprio wallet ai pagamenti che avvengono con ShopChain.

\paragraph{UC2 - Consultazione manuale ShopChain}
\textbf{Attore primario}: Venditore\\
\textbf{Precondizioni}: Il venditore usa il servizio ShopChain e vorrebbe avere più informazioni sul suo utilizzo.\\
\textbf{Postcondizioni}: Il venditore consulta il manuale utente di ShopChain.\\
\textbf{Scenario principale}:\\
\begin{enumerate}
\item Al venditore viene fornito il manuale utente già dall'acquisizione del prodotto ShopChain.
\item Il venditore è libero di consultare il manuale in ogni momento.
\end{enumerate}


\paragraph{UC3 - Accesso al sistema}
\textbf{Attore primario}: Utente non riconosciuto\\
\textbf{Attore secondario}: Metamask\\
\textbf{Precondizioni}:  Un utente non riconosciuto vuole accedere al sistema.\\
\textbf{Postcondizioni}: L'utente viene riconosciuto come venditore nella webapp.\\
\textbf{Scenario principale}:
\begin{enumerate}
\item Viene chiesto all'utente di accedere a Metamask.
\item L'utente accede a Metamask.
\item L'utente viene riconosciuto nella webapp.
\end{enumerate}
Estensione:\\
\textbf{UC3.1 Visualizzazione errore venditore}
\textbf{Attore primario}: Utente non riconosciuto\\
\textbf{Precondizioni}: Un utente non riconosciuto vuole accedere al sistema, non avendo eseguito correttamente l'accesso a Metamask.\\
\textbf{Postcondizioni}: L'utente non riconosciuto viene informato del non avvenuto accesso al sistema.\\
\textbf{Scenario principale}:
\begin{enumerate}
\item L'utente che non ha fatto l'accesso a Metamask in modo corretto prova ad accedere alla webapp venditore di ShopChain.
\item All'utente compare un messaggio di errore nell'accesso.
\end{enumerate}

\paragraph{UC3.2 - Accesso al sistema con autologin}
\textbf{Attore primario}: Venditore\\
\textbf{Precondizioni}: Un venditore già con accesso a Metamask vuole entrare nella webapp.\\
\textbf{Postcondizioni}: Il venditore accede direttamente alla sua pagina nella webapp.\\
\textbf{Scenario principale}:
\begin{enumerate}
\item Un venditore già con accesso a Metamask prova accede alla webapp.
\item Il sistema riconosce l'accesso già effettuato e porta il venditore alla propria pagina.
\end{enumerate}

\paragraph{UC4 visualizzazione lista transazioni}
\textbf{Attore primario}: Venditore \\
\textbf{Precondizioni}: Il venditore ha effettuato l'accesso al sistema.\\
\textbf{Postcondizioni}:  Il venditore vede la lista delle transazioni
Scenario principale.\\
\textbf{Scenario principale}:
\begin{enumerate}
\item Il venditore tramite un menu arriva alla lista delle transazioni.
\item Il venditore può scegliere che tipo di transazioni vedere (tutte, completate, in attesa).
\item Il sistema mostra al venditore l'elenco delle transizioni richieste, ordinate per data e per ognuna vengono mostrati i dati più importanti.
\end{enumerate}

\paragraph{UC5 - visualizzazione dettaglio transazione}
\textbf{Attore primario}: Venditore\\
\textbf{Precondizioni}: Il venditore ha acceduto alla funzionalità di visualizzazione della lista delle transazioni.\\
\textbf{Postcondizioni}: Il venditore vede i dettagli di una singola transizione.\\
\begin{enumerate}
\item Il venditore seleziona una delle transizioni dall'elenco.
\item Il venditore visualizza i dettagli della singola transizione.
\end{enumerate}

\paragraph{UC5.1 - Visualizzazione stato transizione}
\textbf{Attore primario}: Venditore\\
\textbf{Precondizioni}: Il venditore vuole visualizzare lo stato di una transizione.\\
\textbf{Postcondizioni}: Il venditore visualizza lo stato di una transizione.\\
\textbf{Scenario principale}: Il venditore vede se la transizione è in attesa, oppure è stata completata.\\

\paragraph{UC5.2 - Generazione QR Code}
\textbf{Attore primario}: Venditore\\
\textbf{Precondizioni}: Il venditore vuole generare il QR Code di una transizione in attesa.\\
\textbf{Postcondizioni}: Il venditore è in possesso del QR code da applicare sul pacco del prodotto.\\
\begin{enumerate}
\item Il venditore visualizza il Qr code della transizione in attesa selezionata.
\item Il venditore stampa il Qr code e lo appplica sul pacco del prodotto.
\end{enumerate}

\paragraph{UC5.3 - Visualizzazione importo transazione}
\textbf{Attore primario}: Venditore\\
\textbf{Precondizioni}: Il venditore vuole visualizzare l'importo di una transizione.\\
\textbf{Postcondizioni}: Il venditore visualizza l'importo di una transizione.\\
\textbf{Scenario principale}: Il venditore vede l'importo della transazione.\\

\paragraph{UC5.4 - Visualizzazione scadenza transazione}
\textbf{Attore primario}: Venditore\\
\textbf{Precondizioni}: Il venditore vuole visualizzare la scadenza di una transazione in attesa.\\
\textbf{Postcondizioni}: Il venditore visualizza la scadenza di una transazione in attesa.\\
\textbf{Scenario principale}: Il venditore vede la data della scadenza di una transazione in attesa, oltre la quale riceverà l'importo anche se non è stato scannerizzato il QR Code.\\


%\paragraph{UCx - }
%\textbf{Attore primario}: \\
%\textbf{Precondizioni}: \\
%\textbf{Postcondizioni}: \\
%\textbf{Scenario principale}: \\


\section{Requisiti}
elenco

\section{Riferimenti}



\end{document}