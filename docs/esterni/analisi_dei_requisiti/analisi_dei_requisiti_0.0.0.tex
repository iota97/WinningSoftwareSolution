\documentclass[a4paper, 12pt]{article}

\newcommand{\templates}{../../template}
\usepackage[a4paper, margin=2.5cm]{geometry}

\usepackage{enumitem}
\setlist[itemize]{noitemsep}
\setlist[enumerate]{noitemsep}

\let\oldpar\paragraph
\renewcommand{\paragraph}[1]{\oldpar{#1\\}\noindent}
\input{\templates/front_page}
\usepackage{hyperref}
\usepackage{array}
\usepackage{tabularx}

\def\vers#1-#2-#3-#4-#5\\{#1&#2&#3&#4&#5\\\hline}

\newcommand{\addversione}[5]{
	\ifdefined\versioni
		\let\old\versioni
		\renewcommand{\versioni}{#1&#2&#3&#4&#5\\\hline\old}
	\else
		\newcommand{\versioni}{#1&#2&#3&#4&#5\\\hline}
	\fi
}

\newcommand{\setversioni}[1]{\newcommand{\versioni}{#1}}

\newcommand{\makeversioni}{
	\begin{center}
		\begin{tabularx}{\textwidth}{|c|c|c|c|X|}
		\hline
		\textbf{Versione} & \textbf{Data} & \textbf{Persona} & \textbf{Attivtà} & \textbf{Descrizione} \\
		\hline
		\versioni
		\end{tabularx}
	\end{center}
	\clearpage
}

\settitolo{Analisi dei requisiti}
\setprogetto{ShopChain}
\setcommittenti{SyncLab}
\setredattori{Alberto Nicoletti, Andrea Volpe}
\setdestuso{esterno}
\setdescrizione{
Questo documento contiene..
}

\addversione{0.0.0}{11/12/2021}{Alberto Nicoletti}{Redazione}{Struttura del documento, stesura sezioni 1 e 2}

\begin{document}

\makefrontpage

\makeversioni

\section{Introduzone}
\subsection{Scopo del documento}
Lo scopo del documento è raccogliere i risultati dell'attività di analisi dei requisiti. Contiene quindi la descrizione dei casi d'uso del prodotto software da sviluppare, ed i requisiti suddivisi per tipologia. Si vuole così dimostrare una completa comprensione del problema e delle aspettative  nella soluzione. Casi d'uso e requisiti saranno tenunuti in considerazione nella fase di progettazione, assicurandosi la soddisfazione dei requisiti. 
\subsection{Glossario}
Di seguito un elenco di parole usate in questo documento, la cui definizione può essere trovata nel Glossario.
parola, parola, parola.

\section{Descrizione del prodotto}
L'azieda \textit{SyncLab} propone, attraverso il capitolato C2: \textit{ShopChain - Exchange Platform on
BlockChain}. Consiste nel realizzare un prototipo di una piattaforma in grado di ‘affiancare’ un crypto-e-commerce nella fasi di pagamento fino alla consegna.
\subsection{Scopo del prodotto}
Il progetto consiste nello sviluppo di una piattaforma su blockchain con lo scopo di rendere possibile e in sicurezza l'acquisto di prodotti tramite criptovalute. Il processo di trasferimento del denaro avviene seguendo queste fasi:
\begin{enumerate}
\item Caricamento dei dati dell'ordine di acquisto nella blockchain e generazione dello smart contract.
\item Trasferimento del denaro dal wallet dell'acquirente in blockchain.
\item Notifica al venditore dell'avvenuto pagamento e blocco del denaro in blockchain.
\item Conferma di ricezione del pacco da parte dell'acquirente tramite scannerizzazione di un Qr Code nel prodotto acquistato.
\item Sblocco del denaro e trasferimento nel wallet del venditore.
\end{enumerate}
\subsection{Parti del prodotto}
Il prodotto software è composta dalle seguenti parti:
\begin{itemize}
\item Smart contracts nella blockchain per la gestione di tutte le fasi del processo di trasferimento del denaro.
\item Piattaforma web per la gestione dei pagamenti da parte del venditore.
\item App o web app per la scannerizzazione del Qr code .
\end{itemize}
\subsection{Caratteristiche utenti}
Gli utenti di \textit{ShopChain} possono essere suddivisi in due categorie:
\begin{itemize}
\item Venditore: gli amministratori di un sito di e-commerce che vogliono aggiungere le criptovalute come metodo di pagamento.
\item Acquirente: I clienti di un sito di e-commerce che scelgono di utilizzare le criptovalute come pagamento per i prodotti da acquistare.
\end{itemize}
Tutti gli utenti sono in possesso di un wallet per criptovalute. 
Non potendo prevedere con accuratezza quanti e quali e-commerce decideranno di utilizzare \textit{ShopChain} altre considerazioni sulle caratteristiche di utenza sono irrilevanti.
\subsection{Vincoli e preferenze}
La proponente non impone vincoli nella scelta della tipologia di tecnologie, va ci sono comunque delle scelte preferenziali da considerare:
\begin{itemize}
\item Utilizzo di blockchain pubblica
\item utilizzo di Java e Angular per lo sviluppo delle parti di Back-end e di Front-end della componente web application del sistema
\item utilizzo di database Postgres
\end{itemize}

Per il completamento del progetto la proponente richiede che siano realizzati i seguenti risultati:
\begin{itemize}
\item server, completo di UI
\item test che dimostrino il corretto funzionamento dei servizi e delle funzionalità previste, con una copertura minima dell' 80\% correlata di report.
\item documentazione su: scelte implementative e progettuali effettuate e relative motivazioni, problemi aperti e eventuali soluzioni proposte da esplorare.
\end{itemize}

\section{Casi d'uso}
elenco

\section{Requisiti}
elenco

\section{Riferimenti}



\end{document}