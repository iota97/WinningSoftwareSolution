\documentclass[a4paper, 12pt]{article}

\usepackage{graphicx}
\usepackage{longtable}
\usepackage[font=small,labelfont=bf]{caption}
\graphicspath{ {images/} }

\newcommand{\templates}{../../template}
\usepackage[a4paper, margin=2.5cm]{geometry}

\usepackage{enumitem}
\setlist[itemize]{noitemsep}
\setlist[enumerate]{noitemsep}

\let\oldpar\paragraph
\renewcommand{\paragraph}[1]{\oldpar{#1\\}\noindent}
\input{\templates/front_page}
\usepackage{hyperref}
\usepackage{array}
\usepackage{tabularx}

\def\vers#1-#2-#3-#4-#5\\{#1&#2&#3&#4&#5\\\hline}

\newcommand{\addversione}[5]{
	\ifdefined\versioni
		\let\old\versioni
		\renewcommand{\versioni}{#1&#2&#3&#4&#5\\\hline\old}
	\else
		\newcommand{\versioni}{#1&#2&#3&#4&#5\\\hline}
	\fi
}

\newcommand{\setversioni}[1]{\newcommand{\versioni}{#1}}

\newcommand{\makeversioni}{
	\begin{center}
		\begin{tabularx}{\textwidth}{|c|c|c|c|X|}
		\hline
		\textbf{Versione} & \textbf{Data} & \textbf{Persona} & \textbf{Attivtà} & \textbf{Descrizione} \\
		\hline
		\versioni
		\end{tabularx}
	\end{center}
	\clearpage
}

\settitolo{Analisi dei requisiti - v1.0.1} 
\setprogetto{ShopChain}
\setcommittenti{SyncLab}
\setredattori{Alberto Nicoletti\\Andrea Volpe}
\setrevisori{Giovanni Cocco}
\setresponsabili{Elia Scandaletti}
\setdestuso{esterno}
\setdescrizione{
Analisi dei requisiti del progetto ShopChain con casi d'uso e requisiti.
}

\addversione{0.0.0}{11/12/2021}{Alberto Nicoletti}{Redazione}{Strutturazione del documento}
\addversione{0.0.1}{03/01/2021}{Andrea Volpe}{Redazione}{Stesura casi d'uso}
\addversione{0.0.2}{21/01/2022}{Alberto Nicoletti}{Redazione}{Stesura requisiti}
\addversione{0.0.3}{21/01/2022}{Andrea Volpe}{Redazione}{Riorganizzazione requisiti}
\addversione{0.1.0}{04/02/2022}{Giovanni Cocco}{Revisione}{Correzioni varie}
\addversione{0.1.1}{05/02/2022}{Alberto Nicoletti}{Redazione}{Adeguamento casi d'uso a tecnologie scelte}
\addversione{0.1.2}{07/02/2022}{Andrea Volpe}{Redazione}{Modifica dei requisiti come concordato nell'incontro del 04/02/2022}
\addversione{1.0.0}{09/02/2022}{Elia Scandaletti}{Approvazione}{Approvazione per RTB}
\addversione{1.0.1}{01/03/2022}{Giovanni Cocco}{Redazione}{Approfondimento requisiti funzionali}
\addversione{1.0.2}{01/03/2022}{Elia Scandaletti}{Redazione}{Applicazione indicazioni del prof. Cardin}


\begin{document}

\makefrontpage

\makeversioni

\section{Introduzione}
\subsection{Scopo del documento}
Lo scopo del documento è raccogliere i risultati dell'attività di analisi dei requisiti. Contiene quindi la descrizione dei casi d'uso del prodotto software da sviluppare, ed i requisiti suddivisi per tipologia. Si vuole così dimostrare una completa comprensione del problema e delle aspettative della soluzione. I casi d'uso, ma soprattutto i requisiti saranno tenuti in considerazione nelle fasi di progettazione, di verifica e di validazione.

\section{Descrizione del prodotto}
L'azienda \textit{SyncLab} propone, attraverso il capitolato C2: \textit{ShopChain - Exchange Platform on
BlockChain}. L'obiettivo è sviluppare un sistema di pagamento sicuro e \textit{super partes} per E-commerce che trattenga i fondi durante la spedizione e che li sblocchi all'arrivo del pacco. Ciò consiste nella realizzazione su blockchain di un contratto digitale che si incarichi di ricevere l’ammontare in criptovaluta, lo trattenga, e lo consegni al venditore solo quando il pacco viene recapitato all’acquirente.
\subsection{Scopo del prodotto}
Il progetto consiste nello sviluppo di una piattaforma su blockchain con lo scopo di rendere automatizzato e sicuro lo smistamento dei fondi da clienti a E-commerce. Il processo di trasferimento del denaro avviene seguendo queste fasi:
\begin{enumerate}
\item caricamento dei dati dell'ordine di acquisto su contratto digitale;
\item trasferimento del denaro dal wallet dell'acquirente al contratto;
\item notifica al venditore dell'avvenuto pagamento;
\item conferma di ricezione del pacco da parte dell'acquirente tramite scannerizzazione di un QR Code sul pacco del prodotto acquistato;
\item invio del denaro dal contratto al wallet del venditore.
\end{enumerate}
\subsection{Parti del prodotto}
Il prodotto software è composto dalle seguenti parti:
\begin{itemize}
\item smart contract nella blockchain per la gestione di tutte le fasi del processo di trasferimento del denaro;
\item script per la messa in vendita automatizzata su contratto da integrare nel backend del venditore;
\item landing page per il pagamento da parte dell'acquirente;
\item piattaforma web per la visualizzazione e gestione delle transazioni da parte di sia venditore sia acquirente;
\item webapp che consente lo sblocco dei fondi dal contratto.
\end{itemize}
\subsection{Caratteristiche utenti}
Gli utenti di \textit{ShopChain} possono essere suddivisi in due categorie:
\begin{itemize}
\item venditore: gli amministratori di un sito di e-commerce che vogliono aggiungere \textit{ShopChain} come metodo di pagamento;
\item acquirente: I clienti di un sito di e-commerce che scelgono di utilizzare \textit{ShopChain} come metodo di pagamento per i prodotti da acquistare.
\end{itemize}
Tutti gli utenti devono essere in possesso di un wallet compatibile con la blockchain scelta per questo Progetto.
Non potendo prevedere con accuratezza quanti e quali e-commerce decideranno di utilizzare \textit{ShopChain} altre considerazioni sulle caratteristiche di utenza sono superflue. Il prodotto deve essere facilmente integrabile in quante più tipologie di e-commerce possibili.

\subsection{Vincoli e preferenze}
Il proponente non impone vincoli nella scelta delle tecnologie, ma ci sono comunque dei suggerimenti da considerare:
\begin{itemize}
\item utilizzo di blockchain pubblica;
\item utilizzo di Java e Angular per lo sviluppo delle parti di Back-end e di Front-end della componente web application del sistema;
\item utilizzo di database PostgreSQL.
\end{itemize}

Per il completamento del progetto il proponente richiede che siano ottenuti i seguenti risultati:
\begin{itemize}
\item server, completo di UI;
\item test che dimostrino il corretto funzionamento dei servizi e delle funzionalità previste, con una copertura minima dell'80\% correlata di report;
\item documentazione su scelte implementative e progettuali effettuate, le relative motivazioni, i problemi aperti e le eventuali soluzioni proposte da esplorare.
\end{itemize}

\section{Casi d'uso}
\subsection{Attori}
\begin{itemize}
\item \textbf{Venditore con wallet connesso}: venditore che ha già connesso il proprio wallet con Metamask;
\item \textbf{venditore con wallet non connesso}: venditore che non ha ancora connesso il proprio wallet con Metamask;
\item \textbf{acquirente con wallet connesso}: acquirente che ha già connesso il proprio wallet con Metamask;
\item \textbf{acquirente con wallet non connesso}: acquirente che non ha ancora connesso il proprio wallet con Metamask;
\item \textbf{e-commerce}: backend dell'e-commerce;
\item \textbf{metamask}: applicativo di terze parti per la gestione dei wallet;
\item \textbf{database}: persistenze SQL interna al nostro sistema.
\end{itemize}

\includegraphics[width=0.7\textwidth]{UC_WAV1}
\captionof{figure}{Caso d'uso UC1}

\paragraph{UC1 - Consultazione manuale ShopChain}\\
\textbf{Attori primari}: Venditore con wallet connesso.\\
\textbf{Precondizioni}: Il venditore usa il servizio ShopChain e vorrebbe avere più informazioni sul suo utilizzo.\\
\textbf{Postcondizioni}: Il venditore consulta il manuale utente di ShopChain.\\
\textbf{Scenario principale}:
\begin{enumerate}
\item Al venditore viene fornito il manuale utente già dall'acquisizione del prodotto ShopChain.
\item Il venditore è libero di consultare il manuale in ogni momento.
\end{enumerate}
\textbf{Requisiti collegati}: RNO03.

\includegraphics[width=0.9\textwidth]{UC_WAV2}
\captionof{figure}{Caso d'uso UC2}

\paragraph{UC2 - Accesso al sistema (webapp venditore)}\\
\textbf{Attore primario}: Venditore con wallet non connesso.\\
\textbf{Attore secondario}: Metamask.\\
\textbf{Precondizioni}: Il venditore vuole accedere alla webapp.\\
\textbf{Postcondizioni}: Il venditore si è connesso alla webapp con Metamask.\\
\textbf{Scenario principale}:
Il venditore usando Metamask, si connette alla webapp con il proprio wallet.\\
\textbf{Requisiti collegati}: RFO01, RFO34.

\includegraphics[width=0.9\textwidth]{UC_WAV3}
\captionof{figure}{Casi d'uso UC3 e UC4}

\paragraph{UC3 - visualizzazione lista transazioni}\\
\textbf{Attore primario}: Venditore con wallet connesso. \\
\textbf{Attore secondario}: Database. \\
\textbf{Precondizioni}: Il venditore ha effettuato l'accesso al sistema.\\
\textbf{Postcondizioni}:  Il venditore vede la lista delle transazioni.\\
\textbf{Scenario principale}:
\begin{enumerate}
\item Il venditore vede la lista delle transazioni;
\item Il venditore può scegliere che tipo di transazioni vedere (tutte, completate, in attesa);
\item Il sistema mostra al venditore l'elenco delle transazioni richieste.
\end{enumerate}
\textbf{Requisiti collegati}: RFO03, RFO35, RFO36.

\paragraph{UC4 - visualizzazione dettaglio transazione}\\
\textbf{Attore primario}: Venditore con wallet connesso.\\
\textbf{Attore secondario}: Database\\
\textbf{Precondizioni}: Il venditore ha acceduto alla funzionalità di visualizzazione della lista delle transazioni.\\
\textbf{Postcondizioni}: Il venditore vede i dettagli di una singola transazione.\\
\textbf{Scenario principale}:
\begin{enumerate}
\item Il venditore seleziona una delle transazioni dall'elenco;
\item Il venditore visualizza i dettagli della singola transazione.
\end{enumerate}
\textbf{Requisiti collegati}: RFO37, RFO38, RFO40.

\includegraphics[width=0.9\textwidth]{UC3D}
\captionof{figure}{Casi d'uso UC3.1, UC3.2 ed UC3.3}

\paragraph{UC3.1 - Visualizzazione stato transazioni}\\
\textbf{Attore primario}: Venditore con wallet connesso.\\
\textbf{Precondizioni}: Il venditore vuole visualizzare lo stato delle transazioni.\\
\textbf{Postcondizioni}: Il venditore visualizza lo stato delle transazioni.\\
\textbf{Scenario principale}: Viene visualizzato lo stato delle transazioni.\\

\paragraph{UC3.2 - Visualizzazione data transazioni}\\
\textbf{Attore primario}: Venditore con wallet connesso.\\
\textbf{Precondizioni}: Il venditore vuole visualizzare la data delle transazioni.\\
\textbf{Postcondizioni}: Il venditore visualizza la data delle transazioni.\\
\textbf{Scenario principale}: Viene visualizzato la data delle transazioni.\\

\paragraph{UC3.3 - Visualizzazione prezzo transazioni}\\
\textbf{Attore primario}: Venditore con wallet connesso.\\
\textbf{Precondizioni}: Il venditore vuole visualizzare il prezzo delle transazioni.\\
\textbf{Postcondizioni}: Il venditore visualizza il prezzo delle transazioni.\\
\textbf{Scenario principale}: Viene visualizzato il prezzo delle transazioni.\\



\includegraphics[width=0.9\textwidth]{UC_WAV4}
\captionof{figure}{Casi d'uso UC4.1, UC4.2, UC4.3 ed UC4.4}

\paragraph{UC4.1 - Visualizzazione stato transazione}\\
\textbf{Attore primario}: Venditore con wallet connesso.\\
\textbf{Precondizioni}: Il venditore vuole visualizzare lo stato di una transazione.\\
\textbf{Postcondizioni}: Il venditore visualizza lo stato di una transazione.\\
\textbf{Scenario principale}: Viene visualizzato lo stato della transazione.\\
\textbf{Requisiti collegati}: RFO04.

\paragraph{UC4.2 - Visualizzazione QR Code}\\
\textbf{Attore primario}: Venditore con wallet connesso.\\
\textbf{Precondizioni}: Il venditore vuole ottenere il QR Code di una transazione in attesa.\\
\textbf{Postcondizioni}: Il venditore è in possesso del QR Code da applicare sul pacco del prodotto.\\
\textbf{Scenario principale}:
\begin{enumerate}
\item Viene visualizzato il QR Code della transazione in attesa selezionata;
\item Il venditore può stampare il QR Code applicarlo sul pacco del prodotto.
\end{enumerate}
\textbf{Requisiti collegati}: RFO05, RFO51.

\paragraph{UC4.3 - Visualizzazione importo transazione}\\
\textbf{Attore primario}: Venditore con wallet connesso.\\
\textbf{Precondizioni}: Il venditore vuole visualizzare l'importo di una transazione.\\
\textbf{Postcondizioni}: Il venditore visualizza l'importo di una transazione.\\
\textbf{Scenario principale}: Viene visualizzato l'importo della transazione in dollari.\\
\textbf{Requisiti collegati}: RFO06.

\paragraph{UC4.4 - Visualizzazione scadenza transazione}\\
\textbf{Attore primario}: Venditore  con wallet connesso.\\
\textbf{Precondizioni}: Il venditore vuole visualizzare la scadenza di una transazione in attesa.\\
\textbf{Postcondizioni}: Il venditore visualizza la scadenza di una transazione in attesa.\\
\textbf{Scenario principale}: Il venditore vede la data della scadenza di una transazione in attesa, oltre la quale l'acquirente riceverà indietro la valuta bloccata.\\
\textbf{Requisiti collegati}: RFO07, RFO39, RFO41.

\includegraphics[width=0.9\textwidth]{UC_script}
\captionof{figure}{Casi d'uso UC5 e UC5.1}


\paragraph{UC5 - Pubblicazione prodotto in blockchain}\\
\textbf{Attore primario}: E-commerce.\\
\textbf{Precondizioni}: L'e-commerce ha integrato lo script nel proprio backend.\\
\textbf{Postcondizioni}: Lo script nell'e-commerce ha pubblicato il nuovo prodotto in blockchain.\\
\textbf{Scenario principale}:
\begin{enumerate}
    \item Il venditore inserisce un nuovo prodotto nell'e-commerce;
    \item Tramite lo script il prodotto viene pubblicato in blockchain;
    \item Il prodotto è pubblico in blockchain e utilizzabile per fare nuove transazioni.
\end{enumerate}
\textbf{Requisiti collegati}: RFO09, RFO43, RFD44, RNO06.\\
\textbf{Estensione}:
\begin{enumerate}
    \item UC5.1 Errore fallimento pubblicazione prodotto in blockchain:\\
\end{enumerate}
lo script salva in un file di log il messaggio di errore.\\
\textbf{Requisiti collegati all'estensione UC5.1}: RFO10.

\includegraphics[width=0.9\textwidth]{UC_LP2}
\captionof{figure}{Casi d'uso UC7, UC8, UC8.1, UC9 e UC9.1}

\paragraph{UC7 - Visualizzazione riepilogo transazione}\\
\textbf{Attore primario}: Acquirente con wallet connesso.\\
\textbf{Precondizioni}: L'acquirente con wallet connesso vuole effettuare un acquisto usando ShopChain.\\
\textbf{Postcondizioni}: Il sistema mostra il riepilogo della transazione.\\
\textbf{Scenario principale}:
Viene mostrato all'acquirente il riepilogo della transazione con il prezzo in dollari da pagare.\\
\textbf{Requisiti collegati}: RFO13.

\paragraph{UC8 - Conferma pagamento}\\
\textbf{Attore primario}: Acquirente con wallet connesso.\\
\textbf{Precondizioni}: L'acquirente sta effettuando un acquisto e il sistema mostra il riepilogo della transazione.\\
\textbf{Postcondizioni}: L'acquirente ha confermato il pagamento.\\
\textbf{Scenario principale}:
\begin{enumerate}
    \item Il sistema mostra il riepilogo della transazione;
    \item L'acquirente conferma l'ordine.
\end{enumerate}
\textbf{Requisiti collegati}: RFO14.\\
\textbf{Estensione}:
UC8.1 Visualizzazione messaggio procedura di pagamento annullata:
\begin{enumerate}
    \item visualizzazione messaggio procedura di pagamento annullata;
    \item cancellazione sessione di pagamento.
\end{enumerate}
\textbf{Requisiti collegati all'estensione UC8.1}: RFO15, RFO16.

\paragraph{UC9 - Pagamento}\\
\textbf{Attore primario}: Acquirente con wallet connesso.\\
\textbf{Attore secondario}: Metamask.\\
\textbf{Precondizioni}: L'acquirente ha confermato il pagamento.\\
\textbf{Postcondizioni}: Il pagamento è stato completato con successo.\\
\textbf{Scenario principale}:
\begin{enumerate}
    \item L'acquirente avvia la procedura di pagamento;
    \item L'acquirente effettua il pagamento tramite Metamask;
    \item L'ordine è stato pagato.
\end{enumerate}
\textbf{Requisiti collegati}: RFO17, RFO47.\\
\textbf{Estensione}:
UC9.1 Visualizzazione errore pagamento non riuscito:
\begin{enumerate}
    \item Il pagamento non è stato completato con successo;
    \item Cancellazione sessione di pagamento.
\end{enumerate}
\textbf{Requisiti collegati all'estensione UC9.1}: RFO18.

\includegraphics[width=0.9\textwidth]{UC7D}
\captionof{figure}{Casi d'uso UC7.1 e UC7.2}

\paragraph{UC7.1 - Visualizzazione prezzo oggetto}\\
\textbf{Attore primario}: Acquirente con wallet connesso.\\
\textbf{Precondizioni}: L'Acquirente vuole visualizzare il prezzo dell'oggetto.\\
\textbf{Postcondizioni}: L'Acquirente visualizza il prezzo dell'oggetto.\\
\textbf{Scenario principale}: Viene visualizzato il prezzo dell'oggetto.\\

\paragraph{UC7.2 - Visualizzazione il venditore dell'oggetto}\\
\textbf{Attore primario}: Acquirente con wallet connesso.\\
\textbf{Precondizioni}: L'Acquirente vuole visualizzare il venditore dell'oggetto.\\
\textbf{Postcondizioni}: L'Acquirente visualizza il venditore dell'oggetto.\\
\textbf{Scenario principale}: Viene visualizzato il venditore dell'oggetto.\\

\includegraphics[width=0.9\textwidth]{UC_WAA1}
\captionof{figure}{Caso d'uso UC10}


\paragraph{UC10 - Accesso al sistema (WebApp Acquirente)}\\
\textbf{Attore primario}: Acquirente con wallet non connesso.\\
\textbf{Attore secondario}: Metamask.\\
\textbf{Precondizioni}: L'acquirente nella webapp acquirente non è connesso con Metamask.\\
\textbf{Postcondizioni}: L'acquirente accede al sistema.\\
\textbf{Scenario principale}:
L'acquirente usando Metamask, si connette con il proprio wallet.
\textbf{Requisiti collegati}: RFO19.

\includegraphics[width=0.9\textwidth]{UC_WAA2}
\captionof{figure}{Casi d'uso UC11, UC12 e UC12.1}

\paragraph{UC11 - Visualizzazione conferma ricezione}\\
\textbf{Attore primario}: Acquirente con wallet connesso.\\
\textbf{Precondizioni}: L'acquirente è connesso con Metamask ed è arrivato alla pagina di conferma ricezione scansionando il QR Code sul pacco del prodotto.\\
\textbf{Postcondizioni}: L'acquirente visualizza la pagina della transazione relativa al pacco appena ricevuto.\\
\textbf{Scenario principale}:
Vengono mostrati i dati relativi alla transazione e un pulsante per confermare la ricezione del pacco.\\
\textbf{Requisiti collegati}: RFO21.

\paragraph{UC12 - Conferma ricezione}\\
\textbf{Attore primario}: Acquirente con wallet connesso.\\
\textbf{Precondizioni}: L'acquirente è connesso con Metamask alla pagina di conferma ricezione e conferma la ricezione del pacco.\\
\textbf{Postcondizioni}: L'acquirente ha confermato la ricezione del pacco, i fondi sono stati sbloccati.\\
\textbf{Scenario principale}:
\begin{enumerate}
    \item L'acquirente controlla sia tutto corretto e conferma la ricezione del pacco;
    \item Il denaro viene sbloccato e mandato nel wallet del venditore.
\end{enumerate}
\textbf{Requisiti collegati}: RFO22, RFO45, RFO46.\\
\textbf{Estensione}:
UC12.1 Errore wallet non corretto:
\begin{enumerate}
    \item Viene mostrato un messaggio di errore spiegando che il wallet con cui è connesso l'acquirente non è corretto.
    \item Si invita a cambiare wallet e riprovare.
\end{enumerate}
\textbf{Requisiti collegati all'estensione UC12.1}: RFO23.

\includegraphics[width=0.9\textwidth]{UC_ECA1}
\captionof{figure}{Caso d'uso UC13}

\paragraph{UC13 - Accesso al sistema (pagina lista transazioni acquirente in e-commerce)}\\
\textbf{Attore primario}: Acquirente con wallet non connesso.\\
\textbf{Attore secondario}: Metamask.\\
\textbf{Precondizioni}: L'acquirente, nella pagina con la sua lista delle transazioni in e-commerce, non è connesso con Metamask.\\
\textbf{Postcondizioni}: L'acquirente accede alla pagina con la sua lista delle transazioni in e-commerce.\\
\textbf{Scenario principale}:
L'acquirente usando Metamask, si connette con il proprio wallet.\\
\textbf{Requisiti collegati}: RFO24, RFO48.

\includegraphics[width=0.9\textwidth]{UC_ECA2}
\captionof{figure}{Casi d'uso UC14 e UC15}

\paragraph{UC14 visualizzazione lista transazioni}\\
\textbf{Attore primario}: Acquirente con wallet connesso. \\
\textbf{Attore secondario}: Database. \\
\textbf{Precondizioni}: L'acquirente ha effettuato l'accesso al sistema.\\
\textbf{Postcondizioni}:  L'acquirente vede la lista delle transazioni.\\
\textbf{Scenario principale}:
\begin{enumerate}
\item L'acquirente vede la lista delle transazioni;
\item L'acquirente può scegliere che tipo di transazioni vedere (tutte, completate, in attesa);
\item Il sistema mostra all'acquirente l'elenco delle transazioni richieste.
\end{enumerate}
\textbf{Requisiti collegati}: RFO26, RFO49.

\paragraph{UC15 - visualizzazione dettaglio transazione}\\
\textbf{Attore primario}: Acquirente con wallet connesso.\\
\textbf{Attore secondario}: Database.\\
\textbf{Precondizioni}: L'acquirente ha acceduto alla funzionalità di visualizzazione della lista delle transazioni.\\
\textbf{Postcondizioni}: L'acquirente vede i dettagli di una singola transazione.\\
\textbf{Scenario principale}:
\begin{enumerate}
\item L'acquirente seleziona una delle transazioni dall'elenco;
\item L'acquirente visualizza i dettagli della singola transazione.
\end{enumerate}
\textbf{Requisiti collegati}: RFO11, .

\includegraphics[width=0.9\textwidth]{UC14D}
\captionof{figure}{Casi d'uso UC14.1, UC14.2 e UC14.3}

\paragraph{UC14.1 - Visualizzazione stato transazioni}\\
\textbf{Attore primario}: Acquirente con wallet connesso.\\
\textbf{Precondizioni}: L'Acquirente vuole visualizzare lo stato delle transazioni.\\
\textbf{Postcondizioni}: L'Acquirente visualizza lo stato delle transazioni.\\
\textbf{Scenario principale}: Viene visualizzato lo stato delle transazioni.\\

\paragraph{UC14.2 - Visualizzazione data transazioni}\\
\textbf{Attore primario}: Acquirente con wallet connesso.\\
\textbf{Precondizioni}: L'Acquirente vuole visualizzare la data delle transazioni.\\
\textbf{Postcondizioni}: L'Acquirente visualizza la data delle transazioni.\\
\textbf{Scenario principale}: Viene visualizzato la data delle transazioni.\\

\paragraph{UC14.3 - Visualizzazione prezzo transazioni}\\
\textbf{Attore primario}: Acquirente con wallet connesso.\\
\textbf{Precondizioni}: L'Acquirente vuole visualizzare il prezzo delle transazioni.\\
\textbf{Postcondizioni}: L'Acquirente visualizza il prezzo delle transazioni.\\
\textbf{Scenario principale}: Viene visualizzato il prezzo delle transazioni.\\


\includegraphics[width=0.9\textwidth]{UC_ECA3}
\captionof{figure}{Casi d'uso UC15.1, UC15.2 e UC15.3}

\paragraph{UC15.1 - Visualizzazione stato transazione}\\
\textbf{Attore primario}: Acquirente con wallet connesso.\\
\textbf{Precondizioni}: L'acquirente vuole visualizzare lo stato di una transazione.\\
\textbf{Postcondizioni}: L'acquirente visualizza lo stato di una transazione.\\
\textbf{Scenario principale}: Viene mostrato lo stato della transazione: in attesa o completata.\\
\textbf{Requisiti collegati}: RFO27.

\paragraph{UC15.2 - Visualizzazione importo transazione}\\
\textbf{Attore primario}: Acquirente con wallet connesso.\\
\textbf{Precondizioni}: L'acquirente vuole visualizzare l'importo di una transazione.\\
\textbf{Postcondizioni}: L'acquirente visualizza l'importo di una transazione.\\
\textbf{Scenario principale}: Viene mostrato l'importo della transazione in dollari.\\
\textbf{Requisiti collegati}: RFO28, RFO29.

\paragraph{UC15.3 - Visualizzazione scadenza transazione}\\
\textbf{Attore primario}: Acquirente  con wallet connesso.\\
\textbf{Precondizioni}: L'acquirente vuole visualizzare la scadenza di una transazione in attesa.\\
\textbf{Postcondizioni}: L'acquirente visualizza la scadenza di una transazione in attesa.\\
\textbf{Scenario principale}: Viene visualizzata la data della scadenza di una transazione in attesa, oltre la quale l'acquirente riceverà indietro la valuta bloccata.\\
\textbf{Requisiti collegati}: RFO30.

\includegraphics[width=0.9\textwidth]{UC_controlli}
\captionof{figure}{Casi d'uso UCa, UCa.1, UCa.2 e UCa.3}

\paragraph{UCa Controllo connessione}\\
\textbf{Attori primari}: Venditore con wallet connesso, acquirente con wallet connesso. \\
\textbf{Precondizioni}: Il sistema vuole interagire con la blockchain.\\
\textbf{Postcondizioni}:  Il sistema può interagire con la blockchain senza problemi.\\
\textbf{Scenario principale}:
Vengono fatti una serie di controlli per la corretta interazione con la blockchain.\\
\textbf{Estensione}: UCa.1 Errore nella connessione alla rete:
\begin{enumerate}
\item Il venditore o l'acquirente è connesso alla rete sbagliata
\end{enumerate}
\textbf{Requisiti collegati all'estensione UCa.1}: RFO31, RFO50.\\
\textbf{Estensione}: UCa.2 Errore transazione non inviata:
\begin{enumerate}
\item La transazione non è stata inviata correttamente.
\end{enumerate}
\textbf{Requisiti collegati all'estensione UCa.2}: RFO32.\\
\textbf{Estensione}: UCa.3 Errore nessun risultato ottenibile:
\begin{enumerate}
\item La transazione è stata correttamente inviata, ma non è stato ottenuto nessun risultato.
\end{enumerate}
\textbf{Requisiti collegati all'estensione UCa.3}: RFO33.

\section{Requisiti}
Ogni requisito è identificato da un codice univoco composto da: R(per requisito)+F/N(per la tipologia: funzionale, non funzionale)+O/D/F(per la rilevanza: obbligatorio, desiderabile, facoltativo)+x(un numero univoco a due cifre).
\subsection{Requisiti funzionali}

 \setlength\tabcolsep{4pt}
\begin{longtable}{|c|p{7cm}|c|p{4cm}|}
\hline
 \multicolumn{4}{| c |}{Requisiti funzionali}\\
 \hline
 Codice & Descrizione & Rilevanza & Fonti\\
 \hline
 \endfirsthead

 \hline
 \multicolumn{4}{| c |}{Requisiti funzionali}\\
 \hline
 Codice & Descrizione & Rilevanza & Fonti\\
 \hline
 \endhead
% WEBAPP VENDITORE
\hline
RFO01 & Per accedere alla webapp venditore il venditore con wallet non connesso deve connettere il proprio wallet tramite Metamask. & Obbligatorio & UC2 \\
\hline
RFO02 & Un venditore con wallet non connesso non può accedere alla webapp venditore. & Obbligatorio & \underline{\href{https://www.math.unipd.it/~tullio/IS-1/2021/Progetto/C2.pdf}{Capitolato}} \\
\hline
RFO34 & La webapp deve comunicare il wallet del venditore al server. & Obbligatorio & UC2\\
\hline
RFO03 & Il venditore deve poter visualizzare la lista delle transazioni nella webapp venditore. & Obbligatorio & UC3 \\
\hline
RFO35 & Il server deve fornire una pagina con la lista delle transazioni. & Obbligatorio & UC3 \\
\hline
RFO36 & La persistenza deve permettere di filtrare le transazioni per venditore. & Obbligatorio & UC3 \\
\hline
RFO04 & Il venditore deve poter visualizzare nella webapp venditore, per ogni transazione, il suo stato (attesa o completata). & Obbligatorio & UC4.1 \\
\hline
RFO05 & Il venditore deve poter visualizzare nella webapp venditore, per ogni transazione, il QR Code da applicare sul pacco. & Obbligatorio & UC4.2 \\
\hline
RFO51 & La webapp venditore deve poter generare data una stringa un QR Code. & Obbligatorio & UC4.2 \\
\hline
RFO06 & Il venditore deve poter visualizzare nella webapp venditore, per ogni transazione, l'importo in dollari. & Obbligatorio & UC4.3 \\
\hline
RFO07 & Il venditore deve poter visualizzare nella webapp venditore, per ogni transazione, la scadenza della transazione. & Obbligatorio & UC4.4 \\
\hline
RFO37 & Il server deve fornire una pagina con i dettagli di un transazione (prezzo, stato, data di creazione, data di chiusura, acquirente e venditore). & Obbligatorio & UC4 \\
\hline
RFO38 & La persistenza deve tener traccia dei dettagli di un transazione (prezzo, stato, data di creazione, data di chiusura, acquirente e venditore). & Obbligatorio & UC4 \\
\hline
RFO39 & La persistenza deve poter associare ad ogni transazione il proprio timestamp. & Obbligatorio & UC4.4 \\
\hline
RFO40 & La persistenza deve poter fornire i dati di una particolare transazione su richiesta del server. & Obbligatorio & UC4 \\
\hline
RFO41 & Lo smart contract deve rimborsare la valuta al compratore se sono passati 14gg dall'apertura della transazione & Obbligatorio & \underline{\href{https://github.com/iota97/WinningSoftwareSolution/blob/main/docs/interni/verbali/2021_11_12_E/2021_11_12_E.pdf}{Verbale del 12/11/2021}}, UC4.4 \\
\hline
% SCRIPT
RFO09 & Il prodotto deve essere pubblicato automaticamente in blockchain, quando il venditore inserisce un nuovo prodotto nell'e-commerce, grazie al nostro script incluso dal venditore nel suo e-commerce. & Obbligatorio & UC5 \\
\hline
RFO10 & Il nostro script deve salvare un messaggio di errore in un file di log, se la pubblicazione del prodotto in blockchain dall'e-commerce (grazie allo script) fallisce. & Obbligatorio & UC5.1 \\
\hline
RFO43 & Lo script deve ritornare l'id del prodotto inserito in blockchain. & Obbligatorio & UC5 \\
\hline
RFD44 & Lo script essere usabile sia come modulo python che come script python standalone. & Desiderabile & UC5 \\
\hline
% LANDING PAGE
RFO11 & Per accedere alla landing page l'acquirente con wallet non connesso deve connettere il proprio wallet tramite Metamask. & Obbligatorio & UC6, UC15 \\
\hline
RFO12 & Un acquirente con wallet non connesso non può accedere alla landing page. & Obbligatorio & \underline{\href{https://www.math.unipd.it/~tullio/IS-1/2021/Progetto/C2.pdf}{Capitolato}}, \underline{\href{https://github.com/iota97/WinningSoftwareSolution/blob/main/docs/interni/verbali/2021_12_22_E/2021_12_22_E.pdf}{verbale del 22/12/2021}} \\
\hline
RFO48 & La landing page deve tracciare l'id dell'oggetto venduto tramite metodo GET. & Obbligatorio & \underline{\href{https://www.math.unipd.it/~tullio/IS-1/2021/Progetto/C2.pdf}{Capitolato}}, \underline{\href{https://github.com/iota97/WinningSoftwareSolution/blob/main/docs/interni/verbali/2021_12_22_E/2021_12_22_E.pdf}{verbale del 22/12/2021}}, \underline{\href{https://github.com/iota97/WinningSoftwareSolution/blob/main/docs/interni/verbali/2021_12_23_I/2021_12_23_I.pdf}{verbale del 23/12/2021}} \\
\hline
RFO13 & L'acquirente con wallet connesso deve vedere il prezzo da pagare in dollari nella landing page. & Obbligatorio & UC7 \\
\hline
RFO14 & L'acquirente con wallet connesso poter confermare il pagamento nella landing page. & Obbligatorio & UC8 \\
\hline
RFO15 & Il sistema deve mostrare un messaggio "ordine annullato" se l'acquirente con wallet connesso annulla il pagamento sulla landing page. & Obbligatorio & UC8.1 \\
\hline
RFO16 & Il sistema deve cancellare la sessione di pagamento se l'acquirente con wallet connesso annulla il pagamento sulla landing page. & Obbligatorio & UC8.1 \\
\hline
RFO17 & L'acquirente con wallet connesso deve essere indirizzato automaticamente su Metamask per completare il pagamento, quando conferma il pagamento sulla landing page. & Obbligatorio & UC9 \\
\hline
RFO47 & La landing page deve, attraverso Metamask, chiamare il metodo di pagamento dello smart contract. & Obbligatorio & UC9 \\
\hline
RFO18 & Il sistema deve cancellare la sessione di pagamento, se il pagamento non va a buon fine quando l'acquirente con wallet connesso effettua il pagamento tramite Metamask. & Obbligatorio & UC9.1 \\
\hline
% WEBAPP ACQUIRENTE
RFO19 & Per accedere alla webapp acquirente, l'acquirente con wallet non connesso deve connettere il proprio wallet tramite Metamask. & Obbligatorio & UC10 \\
\hline
RFO20 & Un acquirente con wallet non connesso non può accedere alla webapp acquirente. & Obbligatorio & \underline{\href{https://www.math.unipd.it/~tullio/IS-1/2021/Progetto/C2.pdf}{Capitolato}}, \underline{\href{https://github.com/iota97/WinningSoftwareSolution/blob/main/docs/interni/verbali/2021_11_12_E/2021_11_12_E.pdf}{verbale del 12/11/2021}} \\
\hline
RFO21 & L'acquirente con wallet connesso deve poter vedere il prezzo in dollari nella webapp acquirente, dopo essere atterrato nella webapp acquirente scansionando il QR Code sul pacco. & Obbligatorio & UC11 \\
\hline
RFO22 & L'acquirente con wallet connesso deve poter confermare la ricezione del pacco nella webapp acquirente, dopo essere atterrato nella webapp acquirente scansionando il QR Code sul pacco. & Obbligatorio & UC12 \\
\hline
RFO45 & La landing page deve attraverso Metamask chiamare il metodo di conferma dello smart contract. & Obbligatorio & UC12 \\
\hline
RFO46 & Lo smart contract deve permettere la chiamata del metodo di conferma solo all'acquirente (identificato dal suo wallet). & Obbligatorio & UC12 \\
\hline
RFO23 & Il sistema deve mostrare un messaggio di errore "wallet non corretto" nella webapp acquirente, nel caso in cui dopo aver confermato la ricezione del pacco nella webapp acquirente, il sistema rileva che il wallet non corrisponde a quello che ha effettuato la transazione. & Obbligatorio & UC12.1 \\
\hline
% E-COMMERCE
RFO24 & Per accedere alla pagina con la sua lista delle transazioni in e-commerce, l'acquirente con wallet non connesso deve connettere il proprio wallet tramite Metamask. & Obbligatorio & UC13 \\
\hline
RFO25 & Un acquirente con wallet non connesso non può accedere alla pagina con la sua lista delle transazioni in e-commerce. & Obbligatorio & \underline{\href{https://www.math.unipd.it/~tullio/IS-1/2021/Progetto/C2.pdf}{Capitolato}}, \underline{\href{https://github.com/iota97/WinningSoftwareSolution/blob/main/docs/interni/verbali/2021_11_12_E/2021_11_12_E.pdf}{verbale del 12/11/2021}} \\
\hline
RFO48 & La pagina delle transazioni deve comunicare il wallet del venditore al server. & Obbligatorio & UC13\\
\hline
RFO26 & L'acquirente con wallet connesso deve poter visualizzare la lista delle sue transazioni in una pagina apposita, dedicata all'e-commerce. & Obbligatorio & UC14 \\
\hline
RFO49 & La persistenza deve permettere di filtrare le transazioni per acquirente. & Obbligatorio & UC14 \\
\hline
RFO27 & L'acquirente con wallet connesso deve poter visualizzare lo stato di ogni sua transazione (attesa o completata) in una pagina apposita, dedicata all'e-commerce. & Obbligatorio & UC15.1 \\
\hline
RFO28 & L'acquirente con wallet connesso deve poter visualizzare l'importo di ogni sua transazione in una pagina apposita, dedicata all'e-commerce. & Obbligatorio & UC15.2 \\
\hline
RFO29 & L'acquirente con wallet connesso deve poter visualizzare l'importo in dollari di ogni sua transazione in una pagina apposita, dedicata all'e-commerce. & Obbligatorio & UC15.2 \\
\hline
RFO30 & L'acquirente con wallet connesso deve poter visualizzare la scadenza di ogni sua transazione in una pagina apposita, dedicata all'e-commerce. & Obbligatorio & UC15.3 \\
\hline
% CONTROLLI
RFO31 & Il sistema deve mostrare un messaggio di errore "Errore nella connessione alla rete" se l'utente (acquirente o venditore) è connesso alla rete sbagliata. & Obbligatorio & UCa.1 \\
\hline
RFO50 & Il sistema deve poter interrogare Metamask su quale sia la rete attualmente usata dall'utente. & Obbligatorio & UCa.1 \\
\hline
RFO32 & Il sistema deve mostrare un messaggio di errore "Errore transazione non inviata" se la transazione non è stata inviata in maniera corretta. & Obbligatorio & UCa.2 \\
\hline
RFO33 & Il sistema deve mostrare un messaggio di errore "Errore nessun risultato ottenibile" se la transazione è stata inviata in maniera corretta ma non è stata ottenuto alcun risultato di risposta. & Obbligatorio & UCa.3 \\
\hline

\end{longtable}
\pagebreak

\setlength\tabcolsep{4pt}
\begin{longtable}{|c|p{5cm}|c|p{4cm}|c|}
\hline
 \multicolumn{5}{| c |}{Requisiti non funzionali}\\
 \hline
 Codice & Descrizione & Rilevanza & Fonti & Tipologia\\
 \hline
 \endfirsthead

 \hline
 \multicolumn{5}{| c |}{Requisiti non funzionali}\\
 \hline
 Codice & Descrizione & Rilevanza & Fonti & Tipologia\\
 \hline
 \endhead

\hline
RNO01 & Utilizzo di blockchain pubblica. & Obbligatorio & \underline{\href{https://www.math.unipd.it/~tullio/IS-1/2021/Progetto/C2.pdf}{Capitolato}}, \underline{\href{https://github.com/iota97/WinningSoftwareSolution/blob/main/docs/esterni/analisi_delle_tecnologie/analisi_delle_tecnologie.pdf}{analisi delle tecnologie}} & Vincolo\\
\hline
RND02 & Utilizzo di database PostgreSQL. & Desiderabile & \underline{\href{https://www.math.unipd.it/~tullio/IS-1/2021/Progetto/C2.pdf}{Capitolato}} & Vincolo\\
\hline
RNO03 & Il venditore deve disporre di un manuale utente per l'utilizzo del sistema. & Obbligatorio &  UC1, \underline{\href{https://www.math.unipd.it/~tullio/IS-1/2021/Progetto/C2.pdf}{capitolato}} & Vincolo\\
\hline
RNO04 & Test che dimostrino il corretto funzionamento dei servizi e delle funzionalità previste,
con una copertura minima dell’ 80\% correlata di report. & Obbligatorio & \underline{\href{https://www.math.unipd.it/~tullio/IS-1/2021/Progetto/C2.pdf}{Capitolato}} & Qualità\\
\hline
RNO05 & Documentazione su scelte implementative e progettuali effettuate, le relative motivazioni, i problemi aperti e le eventuali soluzioni proposte da esplorare. & Obbligatorio & \underline{\href{https://www.math.unipd.it/~tullio/IS-1/2021/Progetto/C2.pdf}{Capitolato}} & Qualità\\
\hline
RNO06 & Deve essere reso disponibile uno script python scaricabile dai venditori da integrare nel loro e-commerce per la pubblicazione automatica dei nuovi prodotti in blockchain. & Obbligatorio & UC5, \underline{\href{https://github.com/iota97/WinningSoftwareSolution/blob/main/docs/interni/verbali/2021_12_23_I/2021_12_23_I.pdf}{verbale del 23/12/2021}}  & Vincolo\\
\hline

\end{longtable}



\end{document}