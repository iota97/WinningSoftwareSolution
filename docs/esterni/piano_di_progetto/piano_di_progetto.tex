\documentclass[a4paper, 12pt]{article}
\usepackage{eurosym}
\usepackage{pdflscape}
\usepackage{pgfgantt}
\usepackage{pgfplots}


\newcommand{\templates}{../../template}
\usepackage[a4paper, margin=2.5cm]{geometry}

\usepackage{enumitem}
\setlist[itemize]{noitemsep}
\setlist[enumerate]{noitemsep}

\let\oldpar\paragraph
\renewcommand{\paragraph}[1]{\oldpar{#1\\}\noindent}
\input{\templates/front_page}
\usepackage{hyperref}
\usepackage{array}
\usepackage{tabularx}

\def\vers#1-#2-#3-#4-#5\\{#1&#2&#3&#4&#5\\\hline}

\newcommand{\addversione}[5]{
	\ifdefined\versioni
		\let\old\versioni
		\renewcommand{\versioni}{#1&#2&#3&#4&#5\\\hline\old}
	\else
		\newcommand{\versioni}{#1&#2&#3&#4&#5\\\hline}
	\fi
}

\newcommand{\setversioni}[1]{\newcommand{\versioni}{#1}}

\newcommand{\makeversioni}{
	\begin{center}
		\begin{tabularx}{\textwidth}{|c|c|c|c|X|}
		\hline
		\textbf{Versione} & \textbf{Data} & \textbf{Persona} & \textbf{Attivtà} & \textbf{Descrizione} \\
		\hline
		\versioni
		\end{tabularx}
	\end{center}
	\clearpage
}

\settitolo{Piano di Progetto}
\setredattori{Elia Scandaletti}
\setresponsabili{Giovanni Cocco}
\setdestuso{esterno}
\setdescrizione{
Questo documento serve a tracciare l'efficienza del progetto. Tiene traccia dei costi sostenuti fino ad oggi e li confronta con i costi preventivati, in relazione agli obiettivi fissati.
}

\addversione{0.0.0}{13/11/2021}{Elia Scandaletti}{Redazione}{Stesura iniziale}
\addversione{0.0.1}{07/02/2022}{Elia Scandaletti}{Redazione}{Aggiornamento impegni secondo Verbale del 04/02/2022}
\addversione{0.0.2}{11/02/2022}{Elia Scandaletti}{Redazione}{Aggiornamento per RTB}
\addversione{1.0.0}{12/02/2022}{Giovanni Cocco}{Approvazione}{Approvazione per RTB}
\addversione{2.0.0}{19/03/2022}{Giovanni Cocco}{Redazione}{Rifacimento documento}

\def\pgfcalendarmonthitname#1{%
\ifcase#1 \or Gennaio\or Febbraio\or Marzo\or Aprile\or Maggio\or Giugno\or Luglio\or Agosto\or Settembre\or Ottobre\or Novembre\or Dicembre\fi%
}
\usepgfplotslibrary{dateplot}

\begin{document}

\makefrontpage

\makeversioni

\tableofcontents
\clearpage

\section{Pianificazione generale}
\subsection{Diagramma di Gannt}
\subsection{Consultivo attuale}
\subsection{Preventivo a finire}

\section{Dettagli di periodo}
\subsection{Metodo di lavoro}
Il metodo di lavoro fa uso del framework SCRUM. Si farà inoltre uso della tecnica di Continuos Integration.\\
Questo metodo si presta meglio di un modello a cascata data la difficoltà di realizzare una pianificazione accurata a monte con tecnologie poco conosciute.\\\\
Tuttavia si terrà che:
\begin{itemize} 
\item dei cicli non vadano a buon fine portanto a iterazioni (ove incrementi sono sempre preferibili);
\item delle attività critiche blocchino le altre portanto a una perdita di parallelismo all'interno del gruppo.
\end{itemize}
I cicli di SCRUM avverranno ad intervalli settimanali, ogni ciclo inizia con una riunione che comprende:
\begin{itemize}
\item resoconto della settimana precedente;
\item focus su problematiche riscontrate e soluzioni adottate;
\item aggiornamento della pianificazione della settimana successiva;
\item assegnazioni degli item ai componenti del gruppo da parte del responsabile.
\end{itemize}
I cicli di SCRUM verranno elencati nella sezione seguente adottando una approccio schematico e disciplinato.\\
Il primo ciclo sarà riassuntivo del lavoro fino all'RTB.
\subsection{Cicli di SCRUM}
\subsubsection{Cliclo 0}
\textbf{Data di inizio:} 9-11-2021\\
\textbf{Data di fine pianficata:} 23-03-2022\\
\textbf{Data di fine effettiva:} N/A
\paragraph{Pianificazione}\\
\textbf{Item da realizzare:}
\begin{itemize}
\item piano di progetto;
\item piano di qualifica;
\item norme di progetto;
\item PoC.
\end{itemize}
\textbf{Ore stimate:}
\begin{itemize}
\item analista: 60 ore (1500€);
\item progettista: 80 ore (2000€);
\item programmatore: 45 ore (675€);
\item responsabile: 40 ore (1200€);
\item verificatore: 20 ore (300€).
\end{itemize}
\textbf{Possibili problematiche:}\\(\textit{problema: mitigazione prevista})
\begin{itemize}
\item difficoltà con le tecnologie: si realizzeranno nel PoC prima le parti più critiche;
\end{itemize}
\paragraph{Resoconto}\\
\textbf{Ore effettive:}
\begin{itemize}
\item analista: 57 ore (1400€);
\item progettista: 80 ore (2000€);
\item programmatore: 46 ore (690€);
\item responsabile: 40 ore (1200€);
\item verificatore: 19 ore (285€).
\end{itemize}
\textbf{Problematiche riscontrate:}\\(\textit{problema: soluzione adottata})
\begin{itemize}
\item difficoltà di tracciamento del lavoro: utilizzo issues GitHub;
\item difficoltà di coordinamento: riunioni settimanali;
\end{itemize}
\textbf{Nozioni apprese:}
\begin{itemize}
\item abbiamo capito la necessità di meglio organizzare i documenti;
\item abbiamo meglio compreso le tecnologie;
\item abbiamo meglio compreso il capitolato.
\end{itemize}
\end{document}