\documentclass[a4paper, 12pt]{article}
\usepackage{eurosym}
\usepackage{pdflscape}
\usepackage{pgfgantt}
\usepackage{pgfplots}
\usepackage{tabularx}
\usepackage[font=small,labelfont=bf]{caption}

\newcommand{\templates}{../../template}
\usepackage[a4paper, margin=2.5cm]{geometry}

\usepackage{enumitem}
\setlist[itemize]{noitemsep}
\setlist[enumerate]{noitemsep}

\let\oldpar\paragraph
\renewcommand{\paragraph}[1]{\oldpar{#1\\}\noindent}
\input{\templates/front_page}
\usepackage{hyperref}
\usepackage{array}
\usepackage{tabularx}

\def\vers#1-#2-#3-#4-#5\\{#1&#2&#3&#4&#5\\\hline}

\newcommand{\addversione}[5]{
	\ifdefined\versioni
		\let\old\versioni
		\renewcommand{\versioni}{#1&#2&#3&#4&#5\\\hline\old}
	\else
		\newcommand{\versioni}{#1&#2&#3&#4&#5\\\hline}
	\fi
}

\newcommand{\setversioni}[1]{\newcommand{\versioni}{#1}}

\newcommand{\makeversioni}{
	\begin{center}
		\begin{tabularx}{\textwidth}{|c|c|c|c|X|}
		\hline
		\textbf{Versione} & \textbf{Data} & \textbf{Persona} & \textbf{Attivtà} & \textbf{Descrizione} \\
		\hline
		\versioni
		\end{tabularx}
	\end{center}
	\clearpage
}

\settitolo{Piano di Progetto}
\setredattori{Elia Scandaletti}
\setresponsabili{Giovanni Cocco}
\setdestuso{esterno}
\setdescrizione{
Questo documento serve a tracciare l'efficienza del progetto. Tiene traccia dei costi sostenuti fino ad oggi e li confronta con i costi preventivati, in relazione agli obiettivi fissati.
}

\addversione{0.0.0}{13/11/2021}{Elia Scandaletti}{Redazione}{Stesura iniziale}
\addversione{0.0.1}{07/02/2022}{Elia Scandaletti}{Redazione}{Aggiornamento impegni secondo Verbale del 04/02/2022}
\addversione{0.0.2}{11/02/2022}{Elia Scandaletti}{Redazione}{Aggiornamento per RTB}
\addversione{1.0.0}{12/02/2022}{Giovanni Cocco}{Approvazione}{Approvazione per RTB}
\addversione{2.0.0}{19/03/2022}{Giovanni Cocco}{Redazione}{Rifacimento documento}
\addversione{2.1.0}{27/03/2022}{Giovanni Cocco}{Redazione}{Ciclo 1}
\addversione{2.2.0}{02/04/2022}{Matteo Galvagni}{Redazione}{Ciclo 2}


\def\pgfcalendarmonthitname#1{%
\ifcase#1 \or Gennaio\or Febbraio\or Marzo\or Aprile\or Maggio\or Giugno\or Luglio\or Agosto\or Settembre\or Ottobre\or Novembre\or Dicembre\fi%
}
\usepgfplotslibrary{dateplot}

\begin{document}

\makefrontpage

\makeversioni

\tableofcontents
\clearpage

\section{Pianificazione generale}

\subsection{Diagramma di Gannt}
\begin{ganttchart}[
    expand chart = \textwidth,
    time slot format = isodate,
    hgrid,
    vgrid = {*6{draw=none}, dotted},
    y unit title = .8cm,
    y unit chart = .7cm,
    group peaks width = 2,
]{2021-12-01}{2022-05-29}
\gantttitlecalendar{year, month=itname} \\
\ganttgroup[progress=100]{RTB}{2021-12-01}{2022-02-08} \\
\ganttbar[progress=100]{Analisi dei requisiti}{2021-12-01}{2022-02-08} \\
\ganttbar[progress=100]{Analisi delle tecnologie}{2021-12-01}{2022-02-08} \\
\ganttbar[progress=100]{Sviluppo PoC}{2022-01-02}{2022-02-08} \\
\ganttgroup[progress=75]{PB}{2022-02-26}{2022-04-7} \\
\ganttbar[progress=50]{Specifica architetturale}{2022-02-26}{2022-04-07} \\
\ganttbar[progress=65]{Manuale Utente}{2022-02-26}{2022-04-07} \\
\ganttbar[progress=100]{Sviluppo Contratto}{2022-02-26}{2022-03-29} \\
\ganttbar[progress=100]{Sviluppo Server}{2022-02-26}{2022-03-29} \\
\ganttbar[progress=100]{Sviluppo Script Python}{2022-02-26}{2022-03-29} \\
\ganttbar[progress=100]{Sviluppo WebApp}{2022-02-26}{2022-03-29} \\
\ganttgroup[progress=0]{CA}{2022-04-07}{2022-04-29} \\

\end{ganttchart}
\captionof{figure}{Diagramma di Gannt del progetto}


\subsection{Consuntivo attuale}
\begin{center}
    \begin{tabularx}{\textwidth}{|X|X|X|X|}
        \hline
        \multicolumn{4}{|c|}{\textbf{Ore effettive}}\\
        \hline
        \hline
        \textbf{Ruolo} & \textbf{Costo orario (\euro)} & \textbf{Ore} & \textbf{Prezzo (\euro)}\\
        \hline
        Responsabile    & 30 & 53  & 1590\\
        \hline
        Amministratore  & 20 & 62  & 1240\\
        \hline
        Analista        & 25 & 67  & 1675 \\
        \hline
        Progettista     & 25 & 100  & 2500\\
        \hline
        Programmatore   & 15 & 106  & 1590\\
        \hline
        Verificatore    & 15 & 29  & 435 \\
        \hline
        \hline
        \textbf{Totale} &    & 437 & 9030\\
        \hline
    \end{tabularx}\\[8pt]
    \mbox{}\\
\end{center}

\subsection{Preventivo a finire}
\begin{center}
    \begin{tabularx}{\textwidth}{|X|X|X|X|}
        \hline
        \multicolumn{4}{|c|}{\textbf{Preventivo ore/costi}}\\
        \hline
        \hline
        \textbf{Ruolo} & \textbf{Costo orario (\euro)} & \textbf{Ore} & \textbf{Prezzo (\euro)}\\
        \hline
        Responsabile    & 30 & 50  & 1500\\
        \hline
        Amministratore  & 20 & 20  & 400\\
        \hline
        Analista        & 25 & 10  & 2500\\
        \hline
        Progettista     & 25 & 20  & 5000\\
        \hline
        Programmatore   & 15 & 50  & 750\\
        \hline
        Verificatore    & 15 & 100  & 1500\\
        \hline
        \hline
        \textbf{Totale} &    & 250 & 11650\\
        \hline
    \end{tabularx}\\[8pt]
    \mbox{}\\
\end{center}

\section{Metodo di lavoro}
Il metodo di lavoro fa uso del framework SCRUM. Si farà inoltre uso della tecnica di Continuos Integration.\\
Questo metodo si presta meglio di un modello a cascata data la difficoltà di realizzare una pianificazione accurata a monte con tecnologie poco conosciute.\\\\
Tuttavia si terrà che:
\begin{itemize}
    \item dei cicli non vadano a buon fine portando a iterazioni (ove incrementi sono sempre preferibili);
    \item delle attività critiche blocchino le altre portando a una perdita di parallelismo all'interno del gruppo.
\end{itemize}
I cicli di SCRUM avverranno ad intervalli settimanali, ogni ciclo inizia con una riunione che comprende:
\begin{itemize}
    \item resoconto della settimana precedente;
    \item focus su problematiche riscontrate e soluzioni adottate;
    \item aggiornamento della pianificazione della settimana successiva;
    \item assegnazioni degli item ai componenti del gruppo da parte del responsabile.
\end{itemize}
I cicli di SCRUM verranno elencati nella sezione seguente adottando una approccio schematico e disciplinato.\\
Il primo ciclo sarà riassuntivo del lavoro fino alla conclusione dell'RTB. Include anche il lavoro svolto tra la prima richiesta di RTB e il superamento effettivo.

\section{Sprint di SCRUM}

\subsection{Sprint 0}
\textbf{Data di inizio:} 9-11-2021\\
\textbf{Data di fine pianificata:} 23-03-2022\\
\textbf{Data di fine effettiva:} 23-30-2022

\subsubsection{Pianificazione}\mbox{}

\begin{center}
    \begin{tabularx}{\textwidth}{|X|X|}
        \hline
        \multicolumn{2}{|c|}{\textbf{Item da realizzare}}\\
        \hline
        \hline
        \textbf{Item} & \textbf{Persone richieste}\\
        \hline
        Piano di progetto & 2\\
        \hline
        Piano di qualifica & 2\\
        \hline
        Norme di progetto & 2\\
        \hline
        Analisi delle tecnologie & 2\\
        \hline
        Analisi dei requisiti & 4\\
        \hline
        PoC & 3\\
        \hline
        Progettazione architetturale & 3\\
        \hline
        Sviluppo server & 1\\
        \hline
        Sviluppo web app & 2\\
        \hline
        Sviluppo script  & 1\\
        \hline
        Sviluppo contratto & 1\\
        \hline
    \end{tabularx}\\[8pt]
    \mbox{}\\
\end{center}

\begin{center}
    \begin{tabularx}{\textwidth}{|X|X|X|X|}
        \hline
        \multicolumn{4}{|c|}{\textbf{Preventivo ore/costi}}\\
        \hline
        \hline
        \textbf{Ruolo} & \textbf{Costo orario (\euro)} & \textbf{Ore} & \textbf{Prezzo (\euro)}\\
        \hline
        Responsabile    & 30 & 40  & 1200\\
        \hline
        Amministratore  & 20 & 60  & 1200\\
        \hline
        Analista        & 25 & 60  & 1500\\
        \hline
        Progettista     & 25 & 80  & 2000\\
        \hline
        Programmatore   & 15 & 45  & 675\\
        \hline
        Verificatore    & 15 & 20  & 300\\
        \hline
        \hline
        \textbf{Totale} &    & 305 & 6875\\
        \hline
    \end{tabularx}\\[8pt]
    \mbox{}\\
\end{center}

\begin{center}
    \begin{tabularx}{\textwidth}{|X|X|}
        \hline
        \multicolumn{2}{|c|}{\textbf{Possibili problematiche}}\\
        \hline
        \hline
        \textbf{Problema} & \textbf{Misure di mitigazione}\\
        \hline
        Difficoltà con le tecnologie & Si realizzeranno nel PoC prima le parti più critiche\\
        \hline
    \end{tabularx}\\[8pt]
    \mbox{}\\
\end{center}

\subsubsection{Resoconto}\mbox{}

\begin{center}
    \begin{tabularx}{\textwidth}{|X|X|X|X|}
        \hline
        \multicolumn{4}{|c|}{\textbf{Ore effettive}}\\
        \hline
        \hline
        \textbf{Ruolo} & \textbf{Costo orario (\euro)} & \textbf{Ore} & \textbf{Prezzo (\euro)}\\
        \hline
        Responsabile    & 30 & 50 (+10)  & 1500\\
        \hline
        Amministratore  & 20 & 60  & 1200\\
        \hline
        Analista        & 25 & 57 (-3)  & 1425 (-75)\\
        \hline
        Progettista     & 25 & 80  & 2000\\
        \hline
        Programmatore   & 15 & 66 (+21)  & 990 (+315)\\
        \hline
        Verificatore    & 15 & 19 (-1)  & 285 (-15)\\
        \hline
        \hline
        \textbf{Totale} &    & 332 (+27) & 7400 (+525)\\
        \hline
    \end{tabularx}\\[8pt]
    \mbox{}\\
\end{center}

\begin{center}
    \begin{tabularx}{\textwidth}{|X|X|}
        \hline
        \multicolumn{2}{|c|}{\textbf{Problematiche riscontrate}}\\
        \hline
        \hline
        \textbf{Problema} & \textbf{Soluzione adottata}\\
        \hline
        Difficoltà di tracciamento del lavoro & Utilizzo issues GitHub\\
        \hline
        Difficoltà di coordinamento & Riunioni settimanali\\
        \hline
    \end{tabularx}\\[8pt]
    \mbox{}\\
\end{center}

\begin{center}
    \begin{tabularx}{\textwidth}{|X|X|}
        \hline
        \multicolumn{2}{|c|}{\textbf{Nozioni apprese}}\\
        \hline
        \hline
        \textbf{Nozione} & \textbf{Conseguenza sulla pianificazione}\\
        \hline
        Necessità di meglio organizzare i documenti & Aumento dei tempi e costi inizialmente, poi la miglior documentazione porterà a un più celere sviluppo\\
        \hline
        Miglior comprensione delle tecnologie & Diminuzione delle ore previste di programmazione\\
        \hline
        Miglior compresone del capitolato & Pianificazione più accurata e minor incertezza futura\\
        \hline
    \end{tabularx}\\[8pt]
    \mbox{}\\
\end{center}

\subsection{Sprint 1}
\textbf{Data di inizio:} 23-03-2022\\
\textbf{Data di fine pianificata:} 30-03-2022\\
\textbf{Data di fine effettiva:} 30-03-2022

\subsubsection{Pianificazione}\mbox{}

\begin{center}
    \begin{tabularx}{\textwidth}{|X|X|}
        \hline
        \multicolumn{2}{|c|}{\textbf{Item da realizzare}}\\
        \hline
        \hline
        \textbf{Item} & \textbf{Persone richieste}\\
        \hline
        Eliminazione pdf dalla parte privata del repo & 1\\
        \hline
        Creazione e popolamento cartella pubblica & 1\\
        \hline
        Aggiornamento documento \textit{Analisi dei Requisiti} & 1\\
        \hline
        Contattare prof. Cardin riguardo documento \textit{Analisi dei Requisiti} & 1\\
        \hline
        Sviluppo server & 2\\
        \hline
        Sviluppo WebApp & 2\\
        \hline
        Sviluppo script per messa in vendita prodotto & 1\\
        \hline
        Completamento contratto & 2\\
        \hline
        Inizio stesura documento \textit{Specifiche Architetturali} & 1\\
        \hline
        Stesura dei documenti \textit{Manuale Acquirente} & 1\\
        \hline
        Stesura dei documenti \textit{Manuale E-Commerce} & 1\\
        \hline
    \end{tabularx}\\[8pt]
    \mbox{}\\
\end{center}

\begin{center}
    \begin{tabularx}{\textwidth}{|X|X|X|X|}
        \hline
        \multicolumn{4}{|c|}{\textbf{Preventivo ore/costi}}\\
        \hline
        \hline
        \textbf{Ruolo} & \textbf{Costo orario (\euro)} & \textbf{Ore} & \textbf{Prezzo (\euro)}\\
        \hline
        Responsabile    & 30 & 3  & 90\\
        \hline
        Amministratore  & 20 & 2  & 40\\
        \hline
        Analista        & 25 & 10  & 250\\
        \hline
        Progettista     & 25 & 20  & 500\\
        \hline
        Programmatore   & 15 & 60  & 900\\
        \hline
        Verificatore    & 15 & 10  & 150\\
        \hline
        \hline
        \textbf{Totale} &    & 105 & 1930\\
        \hline
    \end{tabularx}\\[8pt]
    \mbox{}\\
\end{center}

\begin{center}
    \begin{tabularx}{\textwidth}{|X|X|}
        \hline
        \multicolumn{2}{|c|}{\textbf{Possibili problematiche}}\\
        \hline
        \hline
        \textbf{Problema} & \textbf{Misure di mitigazione}\\
        \hline
        Eventuali dubbi sui documenti & Si chiederà un colloquio al docente Cardin per eventuali chiarimenti\\
        \hline
    \end{tabularx}\\[8pt]
    \mbox{}\\
\end{center}

\subsubsection{Resoconto}\mbox{}

\begin{center}
    \begin{tabularx}{\textwidth}{|X|X|X|X|}
        \hline
        \multicolumn{4}{|c|}{\textbf{Ore effettive}}\\
        \hline
        \hline
        \textbf{Ruolo} & \textbf{Costo orario (\euro)} & \textbf{Ore} & \textbf{Prezzo (\euro)}\\
        \hline
        Responsabile    & 30 & 3  & 90\\
        \hline
        Amministratore  & 20 & 2  & 40\\
        \hline
        Analista        & 25 & 10  & 250\\
        \hline
        Progettista     & 25 & 20  & 500\\
        \hline
        Programmatore   & 15 & 40 (-20) & 600 (-300)\\
        \hline
        Verificatore    & 15 & 10  & 150\\
        \hline
        \hline
        \textbf{Totale} &    & 105 & 1630 (-300)\\
        \hline
    \end{tabularx}\\[8pt]
    \mbox{}\\
\end{center}

\begin{center}
    \begin{tabularx}{\textwidth}{|X|X|}
        \hline
        \multicolumn{2}{|c|}{\textbf{Problematiche riscontrate}}\\
        \hline
        \hline
        \textbf{Problema} & \textbf{Soluzione adottata}\\
        \hline
        Necessarie correzione dei manuali & Si proseguirà la stesure nel prossimo ciclo di sprint\\
        \hline
        Dubbi sulle specifiche architetturali & Richiesta chiarimenti a Cardin\\
        \hline
    \end{tabularx}\\[8pt]
    \mbox{}\\
\end{center}

\begin{center}
    \begin{tabularx}{\textwidth}{|X|X|}
        \hline
        \multicolumn{2}{|c|}{\textbf{Nozioni apprese}}\\
        \hline
        \hline
        \textbf{Nozione} & \textbf{Conseguenza sulla pianificazione}\\
        \hline
        Necessità di chiarimenti sui documenti & Aumento dei tempi\\
        \hline
    \end{tabularx}\\[8pt]
    \mbox{}\\
\end{center}

\subsection{Sprint 2}
\textbf{Data di inizio:} 30-03-2022\\
\textbf{Data di fine pianificata:} 07-04-2022\\
\textbf{Data di fine effettiva:} TODO

\subsubsection{Pianificazione}\mbox{}

\begin{center}
    \begin{tabularx}{\textwidth}{|X|X|}
        \hline
        \multicolumn{2}{|c|}{\textbf{Item da realizzare}}\\
        \hline
        \hline
        \textbf{Item} & \textbf{Persone richieste}\\
        \hline
        Stesura documento \textit{Specifiche architetturali} & 2\\
        \hline
        Aggiornamento documento \textit{Analisi dei Requisiti} & 3\\
        \hline
        Stesura manuale e-commerce & 2\\
    \end{tabularx}\\[8pt]
    \mbox{}\\
\end{center}

\begin{center}
    \begin{tabularx}{\textwidth}{|X|X|X|X|}
        \hline
        \multicolumn{4}{|c|}{\textbf{Preventivo ore/costi}}\\
        \hline
        \hline
        \textbf{Ruolo} & \textbf{Costo orario (\euro)} & \textbf{Ore} & \textbf{Prezzo (\euro)}\\
        \hline
        Responsabile    & 30 & 3  & 90\\
        \hline
        Amministratore  & 20 & 0  & 0\\
        \hline
        Analista        & 25 & 5  & 125\\
        \hline
        Progettista     & 25 & 4  & 100\\
        \hline
        Programmatore   & 15 & 4  & 60\\
        \hline
        Verificatore    & 15 & 2  & 30\\
        \hline
        \hline
        \textbf{Totale} &    & 18 & 405\\
        \hline
    \end{tabularx}\\[8pt]
    \mbox{}\\
\end{center}

\begin{center}
    \begin{tabularx}{\textwidth}{|X|X|}
        \hline
        \multicolumn{2}{|c|}{\textbf{Possibili problematiche}}\\
        \hline
        \hline
        \textbf{Problema} & \textbf{Misure di mitigazione}\\
        \hline
        Eventuali dubbi sui documenti & Si chiederà un colloquio al docente Cardin per eventuali chiarimenti\\
        \hline
    \end{tabularx}\\[8pt]
    \mbox{}\\
\end{center}

\end{document}