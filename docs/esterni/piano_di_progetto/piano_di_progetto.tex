\documentclass[a4paper, 12pt]{article}
\usepackage{eurosym}
\usepackage{pdflscape}
\usepackage{pgfgantt}
\usepackage{pgfplots}
\usepackage{tabularx}
\usepackage[font=small,labelfont=bf]{caption}

\newcommand{\templates}{../../template}
\usepackage[a4paper, margin=2.5cm]{geometry}

\usepackage{enumitem}
\setlist[itemize]{noitemsep}
\setlist[enumerate]{noitemsep}

\let\oldpar\paragraph
\renewcommand{\paragraph}[1]{\oldpar{#1\\}\noindent}
\input{\templates/front_page}
\usepackage{hyperref}
\usepackage{array}
\usepackage{tabularx}

\def\vers#1-#2-#3-#4-#5\\{#1&#2&#3&#4&#5\\\hline}

\newcommand{\addversione}[5]{
	\ifdefined\versioni
		\let\old\versioni
		\renewcommand{\versioni}{#1&#2&#3&#4&#5\\\hline\old}
	\else
		\newcommand{\versioni}{#1&#2&#3&#4&#5\\\hline}
	\fi
}

\newcommand{\setversioni}[1]{\newcommand{\versioni}{#1}}

\newcommand{\makeversioni}{
	\begin{center}
		\begin{tabularx}{\textwidth}{|c|c|c|c|X|}
		\hline
		\textbf{Versione} & \textbf{Data} & \textbf{Persona} & \textbf{Attivtà} & \textbf{Descrizione} \\
		\hline
		\versioni
		\end{tabularx}
	\end{center}
	\clearpage
}

\settitolo{Piano di Progetto}
\setredattori{Elia Scandaletti \\ Giovanni Cocco \\ Matteo Galvagni \\ Federico Marchi \\ Alberto Nicoletti \\ Raffaele Oliviero \\ Andrea Volpe}
\setresponsabili{Giovanni Cocco}
\setdestuso{esterno}
\setdescrizione{
Questo documento serve a tracciare l'efficienza del progetto. Tiene traccia dei costi sostenuti fino ad oggi e li confronta con i costi preventivati, in relazione agli obiettivi fissati.
}

\addversione{0.0.0}{13/11/2021}{Elia Scandaletti}{Giovanni Cocco}{Stesura iniziale.}
\addversione{0.0.1}{07/02/2022}{Elia Scandaletti}{Giovanni Cocco}{Aggiornamento impegni secondo Verbale del 04/02/2022.}
\addversione{1.0.0}{11/02/2022}{Elia Scandaletti}{Giovanni Cocco}{Aggiornamento per RTB.}
\addversione{2.0.0}{19/03/2022}{Giovanni Cocco}{Elia Scandaletti}{Rifacimento documento.}
\addversione{2.1.0}{27/03/2022}{Giovanni Cocco}{Elia Scandaletti}{Ciclo 1.}
\addversione{2.2.0}{03/04/2021}{Matteo Galvagni}{Giovanni Cocco}{Aggiunto preventivo per ciclo 2.}
\addversione{2.3.0}{07/04/2021}{Matteo Galvagni}{Raffaele Oliviero}{Aggiornamento ciclo 2.}
\addversione{2.4.0}{07/04/2022}{Federico Marchi}{Giovanni Cocco}{Aggiunto preventivo per ciclo 3.}
\addversione{2.4.1}{08/04/2022}{Matteo Galvagni}{Giovanni Cocco}{Correzione errore Gantt.}
\addversione{2.5.0}{13/04/2022}{Federico Marchi}{Giovanni Cocco}{Aggiornamento ciclo 3.}
\addversione{2.6.0}{16/04/2022}{Alberto Nicoletti}{Giovanni Cocco}{Aggiunto preventivo per ciclo 4.}
\addversione{2.7.0}{20/04/2022}{Alberto Nicoletti}{Giovanni Cocco}{Chiusura ciclo 4.}
\addversione{2.7.1}{21/04/2022}{Elia Scandaletti}{Giovanni Cocco}{Correzione grammatica.}
\addversione{2.8.0}{21/04/2022}{Raffaele Oliviero}{Andrea Volpe}{Aggiunto preventivo per ciclo 5.}
\addversione{2.9.0}{27/04/2022}{Raffaele Oliviero}{Andrea Volpe}{Chiusura ciclo 5.}
\addversione{2.10.0}{28/04/2022}{Andrea Volpe}{Raffaele Oliviero}{Aggiunto preventivo per ciclo 6.}
\addversione{2.11.0}{04/05/2022}{Andrea Volpe}{Alberto Nicoletti}{Chiusura ciclo 6.}
\addversione{2.11.1}{07/05/2022}{Matteo Galvagni}{Elia Scandaletti}{Correzione dopo revisione PB.}
\addversione{2.12.0}{07/05/2022}{Alberto Nicoletti}{}{Aggiunta analisi narrativa dello stato d'avanzamento.}

\def\pgfcalendarmonthitname#1{%
\ifcase#1 \or Gennaio\or Febbraio\or Marzo\or Aprile\or Maggio\or Giugno\or Luglio\or Agosto\or Settembre\or Ottobre\or Novembre\or Dicembre\fi%
}
\usepgfplotslibrary{dateplot}

\begin{document}

\makefrontpage

\makeversioni

\tableofcontents
\clearpage

\section{Pianificazione generale}

\subsection{Diagramma di Gantt}
\begin{ganttchart}[
    expand chart = \textwidth,
    time slot format = isodate,
    hgrid,
    vgrid = {*6{draw=none}, dotted},
    y unit title = .8cm,
    y unit chart = .7cm,
    group peaks width = 2,
]{2021-12-01}{2022-06-29}
\gantttitlecalendar{year, month=itname} \\
\ganttgroup[progress=100]{RTB}{2021-12-01}{2022-02-08} \\
\ganttbar[progress=100]{Analisi dei requisiti}{2021-12-01}{2022-02-08} \\
\ganttbar[progress=100]{Analisi delle tecnologie}{2021-12-01}{2022-02-08} \\
\ganttbar[progress=100]{Sviluppo PoC}{2022-01-02}{2022-02-08} \\
\ganttgroup[progress=100]{PB}{2022-02-26}{2022-04-07} \\
\ganttbar[progress=100]{Specifica architetturale}{2022-02-26}{2022-04-07} \\
\ganttbar[progress=100]{Manuale Utente}{2022-02-26}{2022-04-07} \\
\ganttbar[progress=100]{Sviluppo Contratto}{2022-02-26}{2022-03-29} \\
\ganttbar[progress=100]{Sviluppo Server}{2022-02-26}{2022-03-29} \\
\ganttbar[progress=100]{Sviluppo Script Python}{2022-02-26}{2022-03-29} \\
\ganttbar[progress=100]{Sviluppo WebApp}{2022-02-26}{2022-03-29} \\
\ganttgroup[progress=75]{CA}{2022-04-07}{2022-05-12} \\
\ganttbar[progress=75]{Manuale Installazione}{2022-03-30}{2022-05-12} \\
\ganttbar[progress=100]{Completamento WebApp}{2022-04-14}{2022-05-04} \\
\ganttbar[progress=0]{Validazione}{2022-05-04}{2022-05-11}
\end{ganttchart}
\captionof{figure}{Diagramma di Gantt del progetto}


\subsection{Consuntivo attuale}
\begin{center}
    \begin{tabularx}{\textwidth}{|X|X|X|X|}
        \hline
        \multicolumn{4}{|c|}{\textbf{Ore effettive}}\\
        \hline
        \hline
        \textbf{Ruolo} & \textbf{Costo orario (\euro)} & \textbf{Ore} & \textbf{Prezzo (\euro)}\\
        \hline
        Responsabile    & 30 & 78  & 2340\\
        \hline
        Amministratore  & 20 & 85  & 1700\\
        \hline
        Analista        & 25 & 101.5  & 2537.5 \\
        \hline
        Progettista     & 25 & 109  & 2725\\
        \hline
        Programmatore   & 15 & 198.5  & 2977.5\\
        \hline
        Verificatore    & 15 & 67  & 1005 \\
        \hline
        \hline
        \textbf{Totale} &    & 639 & 13285\\
        \hline
    \end{tabularx}\\[8pt]
    \mbox{}\\
\end{center}

\subsection{Preventivo a finire}
\begin{center}
    \begin{tabularx}{\textwidth}{|X|X|X|X|}
        \hline
        \multicolumn{4}{|c|}{\textbf{Preventivo ore/costi}}\\
        \hline
        \hline
        \textbf{Ruolo} & \textbf{Costo orario (\euro)} & \textbf{Ore} & \textbf{Prezzo (\euro)}\\
        \hline
        Responsabile    & 30 & 9  & 270\\
        \hline
        Amministratore  & 20 & 7  & 140\\
        \hline
        Analista        & 25 & 2  & 50\\
        \hline
        Progettista     & 25 & 2  & 50\\
        \hline
        Programmatore   & 15 & 35  & 525\\
        \hline
        Verificatore    & 15 & 16  & 240\\
        \hline
        \hline
        \textbf{Totale} &    & 71 & 1275\\
        \hline
    \end{tabularx}\\[8pt]
    \mbox{}\\
\end{center}

\textbf{Data di fine prevista}: 12 Maggio 2022

\subsection{Stato d'avanzamento corrente}
La data di consegna è postiticipata di due settimane rispetto al preventivo iniziale. Non si prevedono ulteriori posticipi. In quella data - o prima - verrà richiesto il colloquio CA.
\\Con la richesta di colloquio CA si prevede di venire a conoscenza dei contenuti della relativa presentazione. Non appena saranno resi noti si inizierà la preparazione per il colloquio.
\\Il prodotto software è completo in tutte le sue parti. E' ora in corso la fase di validazione con controllo di soddisfazione dei requisiti.
\\I manuali utente sono completi. Il manuale sviluppatore è in conclusione.
\\Sono in atto aggiornamenti alla documentazione, in linea con la valutazione PB del prof. Vardanega.
\\La gestione di progetto procede senza intralci. Il WoW adottato a seguito delle problematiche del primo periodo ha dato i suoi frutti.

\section{Metodo di lavoro}
Il metodo di lavoro fa uso del framework Scrum. Si farà inoltre uso della tecnica di Continuos Integration.\\
Questo metodo si presta meglio di un modello a cascata data la difficoltà di realizzare una pianificazione accurata a monte con tecnologie poco conosciute.\\\\
Tuttavia si terrà che:
\begin{itemize}
    \item dei cicli non vadano a buon fine portando a iterazioni (ove incrementi sono sempre preferibili);
    \item delle attività critiche blocchino le altre portando a una perdita di parallelismo all'interno del gruppo.
\end{itemize}
I cicli di Scrum avverranno ad intervalli settimanali, ogni ciclo inizia con una riunione che comprende:
\begin{itemize}
    \item resoconto della settimana precedente;
    \item focus su problematiche riscontrate e soluzioni adottate;
    \item aggiornamento della pianificazione della settimana successiva;
    \item assegnazioni degli item ai componenti del gruppo da parte del responsabile.
\end{itemize}
I cicli di Scrum verranno elencati nella sezione seguente adottando una approccio schematico e disciplinato.\\
Il primo ciclo sarà riassuntivo del lavoro fino alla conclusione dell'RTB. Include anche il lavoro svolto tra la prima richiesta di RTB e il superamento effettivo.

\section{Sprint di Scrum}

\subsection{Sprint 0}
\textbf{Data di inizio:} 9-11-2021\\
\textbf{Data di fine pianificata:} 23-03-2022\\
\textbf{Data di fine effettiva:} 23-30-2022

\subsubsection{Pianificazione}\mbox{}

\begin{center}
    \begin{tabularx}{\textwidth}{|X|X|}
        \hline
        \multicolumn{2}{|c|}{\textbf{Item da realizzare}}\\
        \hline
        \hline
        \textbf{Item} & \textbf{Persone richieste}\\
        \hline
        Piano di progetto & 2\\
        \hline
        Piano di qualifica & 2\\
        \hline
        Norme di progetto & 2\\
        \hline
        Analisi delle tecnologie & 2\\
        \hline
        Analisi dei requisiti & 4\\
        \hline
        PoC & 3\\
        \hline
        Progettazione architetturale & 3\\
        \hline
        Sviluppo server & 1\\
        \hline
        Sviluppo web app & 2\\
        \hline
        Sviluppo script  & 1\\
        \hline
        Sviluppo contratto & 1\\
        \hline
    \end{tabularx}\\[8pt]
    \mbox{}\\
\end{center}

\begin{center}
    \begin{tabularx}{\textwidth}{|X|X|X|X|}
        \hline
        \multicolumn{4}{|c|}{\textbf{Preventivo ore/costi}}\\
        \hline
        \hline
        \textbf{Ruolo} & \textbf{Costo orario (\euro)} & \textbf{Ore} & \textbf{Prezzo (\euro)}\\
        \hline
        Responsabile    & 30 & 40  & 1200\\
        \hline
        Amministratore  & 20 & 60  & 1200\\
        \hline
        Analista        & 25 & 60  & 1500\\
        \hline
        Progettista     & 25 & 80  & 2000\\
        \hline
        Programmatore   & 15 & 45  & 675\\
        \hline
        Verificatore    & 15 & 20  & 300\\
        \hline
        \hline
        \textbf{Totale} &    & 305 & 6875\\
        \hline
    \end{tabularx}\\[8pt]
    \mbox{}\\
\end{center}

\begin{center}
    \begin{tabularx}{\textwidth}{|X|X|}
        \hline
        \multicolumn{2}{|c|}{\textbf{Possibili problematiche}}\\
        \hline
        \hline
        \textbf{Problema} & \textbf{Misure di mitigazione}\\
        \hline
        Difficoltà con le tecnologie & Si realizzeranno nel PoC prima le parti più critiche\\
        \hline
    \end{tabularx}\\[8pt]
    \mbox{}\\
\end{center}

\subsubsection{Resoconto}\mbox{}

\begin{center}
    \begin{tabularx}{\textwidth}{|X|X|X|X|}
        \hline
        \multicolumn{4}{|c|}{\textbf{Ore effettive}}\\
        \hline
        \hline
        \textbf{Ruolo} & \textbf{Costo orario (\euro)} & \textbf{Ore} & \textbf{Prezzo (\euro)}\\
        \hline
        Responsabile    & 30 & 50 (+10)  & 1500\\
        \hline
        Amministratore  & 20 & 60  & 1200\\
        \hline
        Analista        & 25 & 57 (-3)  & 1425 (-75)\\
        \hline
        Progettista     & 25 & 80  & 2000\\
        \hline
        Programmatore   & 15 & 66 (+21)  & 990 (+315)\\
        \hline
        Verificatore    & 15 & 19 (-1)  & 285 (-15)\\
        \hline
        \hline
        \textbf{Totale} &    & 332 (+27) & 7400 (+525)\\
        \hline
    \end{tabularx}\\[8pt]
    \mbox{}\\
\end{center}

\begin{center}
    \begin{tabularx}{\textwidth}{|X|X|}
        \hline
        \multicolumn{2}{|c|}{\textbf{Problematiche riscontrate}}\\
        \hline
        \hline
        \textbf{Problema} & \textbf{Soluzione adottata}\\
        \hline
        Difficoltà di tracciamento del lavoro & Utilizzo issues GitHub\\
        \hline
        Difficoltà di coordinamento & Riunioni settimanali\\
        \hline
    \end{tabularx}\\[8pt]
    \mbox{}\\
\end{center}

\begin{center}
    \begin{tabularx}{\textwidth}{|X|X|}
        \hline
        \multicolumn{2}{|c|}{\textbf{Nozioni apprese}}\\
        \hline
        \hline
        \textbf{Nozione} & \textbf{Conseguenza sulla pianificazione}\\
        \hline
        Necessità di meglio organizzare i documenti & Aumento dei tempi e costi inizialmente, poi la miglior documentazione porterà a un più celere sviluppo\\
        \hline
        Miglior comprensione delle tecnologie & Diminuzione delle ore previste di programmazione\\
        \hline
        Miglior compresone del capitolato & Pianificazione più accurata e minor incertezza futura\\
        \hline
    \end{tabularx}\\[8pt]
    \mbox{}\\
\end{center}

\subsection{Sprint 1}
\textbf{Data di inizio:} 23-03-2022\\
\textbf{Data di fine pianificata:} 30-03-2022\\
\textbf{Data di fine effettiva:} 30-03-2022

\subsubsection{Pianificazione}\mbox{}

\begin{center}
    \begin{tabularx}{\textwidth}{|X|X|}
        \hline
        \multicolumn{2}{|c|}{\textbf{Item da realizzare}}\\
        \hline
        \hline
        \textbf{Item} & \textbf{Persone richieste}\\
        \hline
        Eliminazione pdf dalla parte privata del repo & 1\\
        \hline
        Creazione e popolamento cartella pubblica & 1\\
        \hline
        Aggiornamento documento \textit{Analisi dei Requisiti} & 1\\
        \hline
        Contattare prof. Cardin riguardo documento \textit{Analisi dei Requisiti} & 1\\
        \hline
        Sviluppo server & 2\\
        \hline
        Sviluppo WebApp & 2\\
        \hline
        Sviluppo script per messa in vendita prodotto & 1\\
        \hline
        Completamento contratto & 2\\
        \hline
        Inizio stesura documento \textit{Specifiche Architetturali} & 1\\
        \hline
        Stesura dei documenti \textit{Manuale Acquirente} & 1\\
        \hline
        Stesura dei documenti \textit{Manuale E-Commerce} & 1\\
        \hline
    \end{tabularx}\\[8pt]
    \mbox{}\\
\end{center}

\begin{center}
    \begin{tabularx}{\textwidth}{|X|X|X|X|}
        \hline
        \multicolumn{4}{|c|}{\textbf{Preventivo ore/costi}}\\
        \hline
        \hline
        \textbf{Ruolo} & \textbf{Costo orario (\euro)} & \textbf{Ore} & \textbf{Prezzo (\euro)}\\
        \hline
        Responsabile    & 30 & 3  & 90\\
        \hline
        Amministratore  & 20 & 2  & 40\\
        \hline
        Analista        & 25 & 10  & 250\\
        \hline
        Progettista     & 25 & 20  & 500\\
        \hline
        Programmatore   & 15 & 60  & 900\\
        \hline
        Verificatore    & 15 & 10  & 150\\
        \hline
        \hline
        \textbf{Totale} &    & 105 & 1930\\
        \hline
    \end{tabularx}\\[8pt]
    \mbox{}\\
\end{center}

\begin{center}
    \begin{tabularx}{\textwidth}{|X|X|}
        \hline
        \multicolumn{2}{|c|}{\textbf{Possibili problematiche}}\\
        \hline
        \hline
        \textbf{Problema} & \textbf{Misure di mitigazione}\\
        \hline
        Eventuali dubbi sui documenti & Si chiederà un colloquio al docente Cardin per eventuali chiarimenti\\
        \hline
    \end{tabularx}\\[8pt]
    \mbox{}\\
\end{center}

\subsubsection{Resoconto}\mbox{}

\begin{center}
    \begin{tabularx}{\textwidth}{|X|X|X|X|}
        \hline
        \multicolumn{4}{|c|}{\textbf{Ore effettive}}\\
        \hline
        \hline
        \textbf{Ruolo} & \textbf{Costo orario (\euro)} & \textbf{Ore} & \textbf{Prezzo (\euro)}\\
        \hline
        Responsabile    & 30 & 3  & 90\\
        \hline
        Amministratore  & 20 & 2  & 40\\
        \hline
        Analista        & 25 & 10  & 250\\
        \hline
        Progettista     & 25 & 20  & 500\\
        \hline
        Programmatore   & 15 & 40 (-20) & 600 (-300)\\
        \hline
        Verificatore    & 15 & 10  & 150\\
        \hline
        \hline
        \textbf{Totale} &    & 85 & 1630 (-300)\\
        \hline
    \end{tabularx}\\[8pt]
    \mbox{}\\
\end{center}

\begin{center}
    \begin{tabularx}{\textwidth}{|X|X|}
        \hline
        \multicolumn{2}{|c|}{\textbf{Problematiche riscontrate}}\\
        \hline
        \hline
        \textbf{Problema} & \textbf{Soluzione adottata}\\
        \hline
        Necessarie correzione dei manuali & Si proseguirà la stesure nel prossimo ciclo di sprint\\
        \hline
        Dubbi sulle specifiche architetturali & Richiesta chiarimenti a Cardin\\
        \hline
    \end{tabularx}\\[8pt]
    \mbox{}\\
\end{center}

\begin{center}
    \begin{tabularx}{\textwidth}{|X|X|}
        \hline
        \multicolumn{2}{|c|}{\textbf{Nozioni apprese}}\\
        \hline
        \hline
        \textbf{Nozione} & \textbf{Conseguenza sulla pianificazione}\\
        \hline
        Necessità di chiarimenti sui documenti & Aumento dei tempi\\
        \hline
    \end{tabularx}\\[8pt]
    \mbox{}\\
\end{center}

\subsection{Sprint 2}
\textbf{Data di inizio:} 30-03-2022\\
\textbf{Data di fine pianificata:} 07-04-2022\\
\textbf{Data di fine effettiva:} 07-04-2022

\subsubsection{Pianificazione}\mbox{}

\begin{center}
    \begin{tabularx}{\textwidth}{|X|X|}
        \hline
        \multicolumn{2}{|c|}{\textbf{Item da realizzare}}\\
        \hline
        \hline
        \textbf{Item} & \textbf{Persone richieste}\\
        \hline
        Stesura documento \textit{Specifiche architetturali} & 2\\
        \hline
        Aggiornamento documento \textit{Analisi dei Requisiti} & 3\\
        \hline
        Stesura manuale e-commerce & 2\\
        \hline
    \end{tabularx}\\[8pt]
    \mbox{}\\
\end{center}

\begin{center}
    \begin{tabularx}{\textwidth}{|X|X|X|X|}
        \hline
        \multicolumn{4}{|c|}{\textbf{Preventivo ore/costi}}\\
        \hline
        \hline
        \textbf{Ruolo} & \textbf{Costo orario (\euro)} & \textbf{Ore} & \textbf{Prezzo (\euro)}\\
        \hline
        Responsabile    & 30 & 4  & 120\\
        \hline
        Amministratore  & 20 & 6  & 120\\
        \hline
        Analista        & 25 & 20  & 500\\
        \hline
        Progettista     & 25 & 6  & 150\\
        \hline
        Programmatore   & 15 & 14  & 210\\
        \hline
        Verificatore    & 15 & 8  & 120\\
        \hline
        \hline
        \textbf{Totale} &    & 58 & 1220\\
        \hline
    \end{tabularx}\\[8pt]
    \mbox{}\\
\end{center}

\begin{center}
    \begin{tabularx}{\textwidth}{|X|X|}
        \hline
        \multicolumn{2}{|c|}{\textbf{Possibili problematiche}}\\
        \hline
        \hline
        \textbf{Problema} & \textbf{Misure di mitigazione}\\
        \hline
        Eventuali dubbi sulla stesura dei documenti & Si farà una riunione per decidere come redigerli.\\
        \hline
    \end{tabularx}\\[8pt]
    \mbox{}\\
\end{center}

\subsubsection{Resoconto}\mbox{}

\begin{center}
    \begin{tabularx}{\textwidth}{|X|X|X|X|}
        \hline
        \multicolumn{4}{|c|}{\textbf{Ore effettive}}\\
        \hline
        \hline
        \textbf{Ruolo} & \textbf{Costo orario (\euro)} & \textbf{Ore} & \textbf{Prezzo (\euro)}\\
        \hline
        Responsabile    & 30 & 5 (+1) & 150 (+30)\\
        \hline
        Amministratore  & 20 & 3 (-3)  & 60 (-60)\\
        \hline
        Analista        & 25 & 25 (+5)  & 625 (+125)\\
        \hline
        Progettista     & 25 & 2 (-4) & 50 (-100)\\
        \hline
        Programmatore   & 15 & 10 (-4) & 150 (-60)\\
        \hline
        Verificatore    & 15 & 13.5 (+5.5) & 202.50 (+82.5)\\
        \hline
        \hline
        \textbf{Totale} &    & 58.5 (+0.5) & 1237.50 (+17.5)\\
        \hline
    \end{tabularx}\\[8pt]
    \mbox{}\\
\end{center}

\begin{center}
    \begin{tabularx}{\textwidth}{|X|X|}
        \hline
        \multicolumn{2}{|c|}{\textbf{Problematiche riscontrate}}\\
        \hline
        \hline
        \textbf{Problema} & \textbf{Soluzione adottata}\\
        \hline
        Metamask ha smesso di inviare correttamente transazioni & Il problema è stato affrontato da due membri scelti dal responsabile e si è rivelato riconducibile all'RPC utilizzata per la rete di test.
        Una volta cambiato link Metamask ha ripreso a funzionare.\\
        \hline
    \end{tabularx}\\[8pt]
    \mbox{}\\
\end{center}

\begin{center}
    \begin{tabularx}{\textwidth}{|X|X|}
        \hline
        \multicolumn{2}{|c|}{\textbf{Nozioni apprese}}\\
        \hline
        \hline
        \textbf{Nozione} & \textbf{Conseguenza sulla pianificazione}\\
        \hline
        Necessità di considerare nel preventivo possibili problematiche con le tecnologie utilizzate & Futura aggiunta di alcune ore nel preventivo per risoluzione di eventuali bug\\
        \hline
    \end{tabularx}\\[8pt]
    \mbox{}\\
\end{center}

\subsection{Sprint 3}
\textbf{Data di inizio:} 06-04-2022\\
\textbf{Data di fine pianificata:} 13-04-2022\\
\textbf{Data di fine effettiva:} 13-04-2022

\subsubsection{Pianificazione}\mbox{}

\begin{center}
    \begin{tabularx}{\textwidth}{|X|X|}
        \hline
        \multicolumn{2}{|c|}{\textbf{Item da realizzare}}\\
        \hline
        \hline
        \textbf{Item} & \textbf{Persone richieste}\\
        \hline
        Continuare stesura documento \textit{Specifiche architetturali} & 2\\
        \hline
        Continuare stesura documento \textit{Manuale Installazione} & 1\\
        \hline
        Css per la webapp & 3\\
        \hline
        Aggiornamento cartella pubblica & 2\\
        \hline
    \end{tabularx}\\[8pt]
    \mbox{}\\
\end{center}

\begin{center}
    \begin{tabularx}{\textwidth}{|X|X|X|X|}
        \hline
        \multicolumn{4}{|c|}{\textbf{Preventivo ore/costi}}\\
        \hline
        \hline
        \textbf{Ruolo} & \textbf{Costo orario (\euro)} & \textbf{Ore} & \textbf{Prezzo (\euro)}\\
        \hline
        Responsabile    & 30 & 5  & 150\\
        \hline
        Amministratore  & 20 & 15  & 300\\
        \hline
        Analista        & 25 & 5  & 125\\
        \hline
        Progettista     & 25 & 5  & 125\\
        \hline
        Programmatore   & 15 & 30  & 450\\
        \hline
        Verificatore    & 15 & 15  & 225\\
        \hline
        \hline
        \textbf{Totale} &    & 75 & 1375\\
        \hline
    \end{tabularx}\\[8pt]
    \mbox{}\\
\end{center}

\begin{center}
    \begin{tabularx}{\textwidth}{|X|X|}
        \hline
        \multicolumn{2}{|c|}{\textbf{Possibili problematiche}}\\
        \hline
        \hline
        \textbf{Problema} & \textbf{Misure di mitigazione}\\
        \hline
        Eventuali dubbi sulla realizzazione della grafica della webapp & Si richiederà un colloquio con il proponente per dei chiarimenti\\
        \hline
    \end{tabularx}\\[8pt]
    \mbox{}\\
\end{center}
\clearpage
\subsubsection{Resoconto}\mbox{}

\begin{center}
    \begin{tabularx}{\textwidth}{|X|X|X|X|}
        \hline
        \multicolumn{4}{|c|}{\textbf{Ore effettive}}\\
        \hline
        \hline
        \textbf{Ruolo} & \textbf{Costo orario (\euro)} & \textbf{Ore} & \textbf{Prezzo (\euro)}\\
        \hline
        Responsabile    & 30 & 5  & 150\\
        \hline
        Amministratore  & 20 & 15  & 300\\
        \hline
        Analista        & 25 & 9.5 (+4.5)  & 237.5 (+112.5)\\
        \hline
        Progettista     & 25 & 3 (-2) & 75 (-50)\\
        \hline
        Programmatore   & 15 & 25 (-5)  & 375 (-75)\\
        \hline
        Verificatore    & 15 & 8 (-7)  & 120 (-105)\\
        \hline
        \hline
        \textbf{Totale} &    & 65.5 (-9.5) & 1257.5 (-117.5)\\
        \hline
    \end{tabularx}\\[8pt]
    \mbox{}\\
\end{center}

\begin{center}
    \begin{tabularx}{\textwidth}{|X|X|}
        \hline
        \multicolumn{2}{|c|}{\textbf{Problematiche riscontrate}}\\
        \hline
        \hline
        \textbf{Problema} & \textbf{Soluzione adottata}\\
        \hline
        Difficoltà nel tener traccia nella cartella "Public" dei documenti aggiornati e delle relative versioni & Realizzazione di uno script per automatizzare il processo\\
        \hline
    \end{tabularx}\\[8pt]
    \mbox{}\\
\end{center}

\begin{center}
    \begin{tabularx}{\textwidth}{|X|X|}
        \hline
        \multicolumn{2}{|c|}{\textbf{Nozioni apprese}}\\
        \hline
        \hline
        \textbf{Nozione} & \textbf{Conseguenza sulla pianificazione}\\
        \hline
        Necessità di migliorare l'accessibilità della web app & Futura aggiunta di alcune ore da programmatore nel preventivo \\
        \hline
    \end{tabularx}\\[8pt]
    \mbox{}\\
\end{center}

\subsection{Sprint 4}
\textbf{Data di inizio:} 13-04-2022\\
\textbf{Data di fine pianificata:} 20-04-2022\\
\textbf{Data di fine effettiva:} 20-04-2022

\subsubsection{Pianificazione}\mbox{}

\begin{center}
    \begin{tabularx}{\textwidth}{|X|X|}
        \hline
        \multicolumn{2}{|c|}{\textbf{Item da realizzare}}\\
        \hline
        \hline
        \textbf{Item} & \textbf{Persone richieste}\\
        \hline
        Scrivere uno script per la pubblicazione automatica dei documenti redatti. & 1\\
        \hline
        Aggiornare la cartella public e le norme di progetto. & 1\\
        \hline
        Correggere l’ortografia in alcuni documenti. & 1\\
        \hline
        Finire la stesura del manuale d'installazione. & 1\\
        \hline
		Sviluppare la grafica della web-app. & 2 \\
		\hline
		Completare le funzionalità della web-app. & 2 \\
		\hline
		Rendere accessibile la web-app. & 2 \\
		\hline
    \end{tabularx}\\[8pt]
    \mbox{}\\
\end{center}

\begin{center}
    \begin{tabularx}{\textwidth}{|X|X|X|X|}
        \hline
        \multicolumn{4}{|c|}{\textbf{Preventivo ore/costi}}\\
        \hline
        \hline
        \textbf{Ruolo} & \textbf{Costo orario (\euro)} & \textbf{Ore} & \textbf{Prezzo (\euro)}\\
        \hline
        Responsabile    & 30 & 3  & 90\\
        \hline
        Amministratore  & 20 & 3  & 60\\
        \hline
        Analista        & 25 & 0  & 0\\
        \hline
        Progettista     & 25 & 2  & 50\\
        \hline
        Programmatore   & 15 & 10  & 150\\
        \hline
        Verificatore    & 15 & 3  & 45\\
        \hline
        \hline
        \textbf{Totale} &    & 21 & 395\\
        \hline
    \end{tabularx}\\[8pt]
    \mbox{}\\
\end{center}

\begin{center}
    \begin{tabularx}{\textwidth}{|X|X|}
        \hline
        \multicolumn{2}{|c|}{\textbf{Possibili problematiche}}\\
        \hline
        \hline
        \textbf{Problema} & \textbf{Misure di mitigazione}\\
        \hline
        Il colloquio PB può non avere un riscontro positivo. & Verranno messe in pausa le task programmate per concentrarsi sulla correzione degli errori.\\
        \hline
    \end{tabularx}\\[8pt]
    \mbox{}\\
\end{center}
\subsubsection{Resoconto}\mbox{}

\begin{center}
    \begin{tabularx}{\textwidth}{|X|X|X|X|}
        \hline
        \multicolumn{4}{|c|}{\textbf{Ore effettive}}\\
        \hline
        \hline
        \textbf{Ruolo} & \textbf{Costo orario (\euro)} & \textbf{Ore} & \textbf{Prezzo (\euro)}\\
        \hline
        Responsabile    & 30 & 3  & 90\\
        \hline
        Amministratore  & 20 & 3  & 60\\
        \hline
        Analista        & 25 & 0  & 0\\
        \hline
        Progettista     & 25 & 4(+2) & 100(+50)\\
        \hline
        Programmatore   & 15 & 13(+3)  & 195(+45)\\
        \hline
        Verificatore    & 15 & 2(-1)  & 30(-15)\\
        \hline
        \hline
        \textbf{Totale} &    & 25(+4) & 475(+80)\\
        \hline
    \end{tabularx}\\[8pt]
    \mbox{}\\
\end{center}

\begin{center}
    \begin{tabularx}{\textwidth}{|X|X|}
        \hline
        \multicolumn{2}{|c|}{\textbf{Problematiche riscontrate}}\\
        \hline
        \hline
        \textbf{Problema} & \textbf{Soluzione adottata}\\
        \hline
        Non è stato tenuto conto delle festività pasquali, causa di ritardi nello svolgimento delle task. & Alcune task vengono protratte al ciclo successivo. In futuro si farà una valutazione più sincera della disponibilità di lavoro dei singoli componenti. \\
        \hline
    \end{tabularx}\\[8pt]
    \mbox{}\\
\end{center}

\begin{center}
    \begin{tabularx}{\textwidth}{|X|X|}
        \hline
        \multicolumn{2}{|c|}{\textbf{Nozioni apprese}}\\
        \hline
        \hline
        \textbf{Nozione} & \textbf{Conseguenza sulla pianificazione}\\
        \hline
        La qualità di presentazione durante la revisione è da migliorare. Soprattutto nella gestione dei tempi e nel selezionare le cose da esporre in base all'interlocutore.  & Maggiore impegno nella preparazione delle slides e nel provare l'esposizione. \\
        \hline
    \end{tabularx}\\[8pt]
    \mbox{}\\
\end{center}

\subsection{Sprint 5}
\textbf{Data di inizio:} 20-04-2022\\
\textbf{Data di fine pianificata:} 27-04-2022\\
\textbf{Data di fine effettiva:} 27-04-2022

\subsubsection{Pianificazione}\mbox{}

\begin{center}
    \begin{tabularx}{\textwidth}{|X|X|}
        \hline
        \multicolumn{2}{|c|}{\textbf{Item da realizzare}}\\
        \hline
        \hline
        \textbf{Item} & \textbf{Persone richieste}\\
        \hline
        Finire la stesura del manuale d'installazione. & 1\\
        \hline
		Sviluppare la grafica della web-app. & 2 \\
		\hline
		Completare le funzionalità della web-app. & 2 \\
		\hline
		Rendere accessibile la web-app. & 1 \\
		\hline
		Creare una bozza della presentazione PB. & 2\\
		\hline
    \end{tabularx}\\[8pt]
    \mbox{}\\
\end{center}

\begin{center}
    \begin{tabularx}{\textwidth}{|X|X|X|X|}
        \hline
        \multicolumn{4}{|c|}{\textbf{Preventivo ore/costi}}\\
        \hline
        \hline
        \textbf{Ruolo} & \textbf{Costo orario (\euro)} & \textbf{Ore} & \textbf{Prezzo (\euro)}\\
        \hline
        Responsabile    & 30 & 4  & 120\\
        \hline
        Amministratore  & 20 & 12  & 240\\
        \hline
        Analista        & 25 & 0  & 0\\
        \hline
        Progettista     & 25 & 2  & 50\\
        \hline
        Programmatore   & 15 & 10  & 150\\
        \hline
        Verificatore    & 15 & 3  & 45\\
        \hline
        \hline
        \textbf{Totale} &    & 31 & 605\\
        \hline
    \end{tabularx}\\[8pt]
    \mbox{}\\
\end{center}

\begin{center}
    \begin{tabularx}{\textwidth}{|X|X|}
        \hline
        \multicolumn{2}{|c|}{\textbf{Possibili problematiche}}\\
        \hline
        \hline
        \textbf{Problema} & \textbf{Misure di mitigazione}\\
        \hline
        Qualora il colloquio con il prof. Cardin dovesse risultare positivo, sarà necessario un incontro per preparare la presentazione PB in vista del colloquio con il prof. Vardanega. & I membri del gruppo sono stati avvisati della possibilità e alcuni membri sono stati assegnati per la creazione di una bozza che verrà visionata e potenzialmente modificata da tutti i membri del gruppo.\\
        \hline
    \end{tabularx}\\[8pt]
    \mbox{}\\
\end{center}

\subsubsection{Resoconto}\mbox{}

\begin{center}
    \begin{tabularx}{\textwidth}{|X|X|X|X|}
        \hline
        \multicolumn{4}{|c|}{\textbf{Ore effettive}}\\
        \hline
        \hline
        \textbf{Ruolo} & \textbf{Costo orario (\euro)} & \textbf{Ore} & \textbf{Prezzo (\euro)}\\
        \hline
        Responsabile    & 30 & 7 (+3)  & 210 (+90)\\
        \hline
        Amministratore  & 20 & 2 (-10)  & 40 (-200)\\
        \hline
        Analista        & 25 & 0  & 0\\
        \hline
        Progettista     & 25 & 0 (-2) & 0 (-50)\\
        \hline
        Programmatore   & 15 & 16.5 (+6.5)  & 247.5(+97.5)\\
        \hline
        Verificatore    & 15 & 4.5 (+1.5)  & 67.5(+22.5)\\
        \hline
        \hline
        \textbf{Totale} &    & 30 (-1) & 565 (-40)\\
        \hline
    \end{tabularx}\\[8pt]
    \mbox{}\\
\end{center}

\begin{center}
    \begin{tabularx}{\textwidth}{|X|X|}
        \hline
        \multicolumn{2}{|c|}{\textbf{Problematiche riscontrate}}\\
        \hline
        \hline
        \textbf{Problema} & \textbf{Soluzione adottata}\\
        \hline
        Una delle attività di aggiunta funzionalità alla web app (l'aggiunta dei bottoni di copia) è stata completata con un intero ciclo di ritardo senza alcuna causa che giustifichi tale ritardo. & I membri e le attività che essi devono svolgere saranno monitorate con più attenzione per evitare che ciò si ripeta in futuro. \\
        \hline
    \end{tabularx}\\[8pt]
    \mbox{}\\
\end{center}

\begin{center}
    \begin{tabularx}{\textwidth}{|X|X|}
        \hline
        \multicolumn{2}{|c|}{\textbf{Nozioni apprese}}\\
        \hline
        \hline
        \textbf{Nozione} & \textbf{Conseguenza sulla pianificazione}\\
        \hline
        La parte della web app dedicata alla visualizzazione della lista delle transazioni è poco fruibile, seppur accessibile. & Sono state create delle Issues per rendere la sezione più usabile da parte degli utenti e in generale per migliorare la grafica. Alcuni di questi cambiamenti potrebbero richiedere la modifica dell'Analisi dei Requisiti. \\
        \hline
    \end{tabularx}\\[8pt]
    \mbox{}\\
\end{center}

\subsection{Sprint 6}
\textbf{Data di inizio:} 27-04-2022\\
\textbf{Data di fine pianificata:} 04-05-2022\\
\textbf{Data di fine effettiva:} 04-05-2022

\subsubsection{Pianificazione}\mbox{}

\begin{center}
    \begin{tabularx}{\textwidth}{|X|X|}
        \hline
        \multicolumn{2}{|c|}{\textbf{Item da realizzare}}\\
        \hline
        \hline
        \textbf{Item} & \textbf{Persone richieste}\\
        \hline
        Modificare presentazione transazioni. & 1\\
		\hline
        Togliere link circolari e segnalare link pagina corrente. & 1\\
		\hline
        Aggiungere breadcrumb. & 1\\
		\hline
        Togliere extraspace landing page. & 1\\
		\hline
        Fare bottoni responsive. & 1\\
		\hline
        Aggiungere id del prodotto per transazioni. & 1\\
		\hline
        Mettere stato transazione. & 2\\
		\hline
        Abbellire il footer. & 1\\
		\hline
    \end{tabularx}\\[8pt]
    \mbox{}\\
\end{center}

\begin{center}
    \begin{tabularx}{\textwidth}{|X|X|X|X|}
        \hline
        \multicolumn{4}{|c|}{\textbf{Preventivo ore/costi}}\\
        \hline
        \hline
        \textbf{Ruolo} & \textbf{Costo orario (\euro)} & \textbf{Ore} & \textbf{Prezzo (\euro)}\\
        \hline
        Responsabile    & 30 & 5  & 150\\
        \hline
        Amministratore  & 20 & 3  & 60\\
        \hline
        Analista        & 25 & 2  & 50\\
        \hline
        Progettista     & 25 & 2  & 50\\
        \hline
        Programmatore   & 15 & 25  & 375\\
        \hline
        Verificatore    & 15 & 10  & 150\\
        \hline
        \hline
        \textbf{Totale} &    & 47 & 835\\
        \hline
    \end{tabularx}\\[8pt]
    \mbox{}\\
\end{center}

\begin{center}
    \begin{tabularx}{\textwidth}{|X|X|}
        \hline
        \multicolumn{2}{|c|}{\textbf{Possibili problematiche}}\\
        \hline
        \hline
        \textbf{Problema} & \textbf{Misure di mitigazione}\\
        \hline
        La modifica della presentazione delle transazioni potrebbe risultare poco fruibile. & Tale modifica verrà scartata, mantenendo la presentazione delle transazioni attuale.\\
        \hline
        Lo studio di fattibilità per valutare la possibilità di togliere i link circolari e segnalare la pagina corrente potrebbe dare esito negativo. & Le attività di rimozione link circolari e la segnalazione del link della pagina corrente non verranno svolte.\\
        \hline
    \end{tabularx}\\[8pt]
    \mbox{}\\
\end{center}

\subsubsection{Resoconto}\mbox{}

\begin{center}
    \begin{tabularx}{\textwidth}{|X|X|X|X|}
        \hline
        \multicolumn{4}{|c|}{\textbf{Ore effettive}}\\
        \hline
        \hline
        \textbf{Ruolo} & \textbf{Costo orario (\euro)} & \textbf{Ore} & \textbf{Prezzo (\euro)}\\
        \hline
        Responsabile    & 30 & 11 (+6)  & 330 (+180) \\
        \hline
        Amministratore  & 20 & 0 (-3)  &  0 (-60) \\
        \hline
        Analista        & 25 & 0 (-2)  & 0 (-50) \\
        \hline
        Progettista     & 25 & 0 (-2) & 0 (-50) \\
        \hline
        Programmatore   & 15 & 28 (+3)  & 420 (+45) \\
        \hline
        Verificatore    & 15 & 10 &  150 \\
        \hline
        \hline
        \textbf{Totale} &    & 49 (+2) & 900 (+65)\\
        \hline
    \end{tabularx}\\[8pt]
    \mbox{}\\
\end{center}

\begin{center}
    \begin{tabularx}{\textwidth}{|X|X|}
        \hline
        \multicolumn{2}{|c|}{\textbf{Problematiche riscontrate}}\\
        \hline
        \hline
        \textbf{Problema} & \textbf{Soluzione adottata}\\
        \hline
        Durante il ciclo si è discusso più volte sul fatto di implementare l'attività "Aggiungere breadcrumb", rallentando il lavoro. & Si è deciso di seguire le attività preventivate, quindi realizzare la breadcrumb. \\
        \hline
        E' emerso durante il ciclo che i bottoni di copy non sono realizzabili per la versione mobile della webapp. & Si è deciso di modificare l'attività durante il ciclo, implementando i bottoni solo per la versione desktop della webapp. \\
        \hline
    \end{tabularx}\\[8pt]
    \mbox{}\\
\end{center}

\begin{center}
    \begin{tabularx}{\textwidth}{|X|X|}
        \hline
        \multicolumn{2}{|c|}{\textbf{Nozioni apprese}}\\
        \hline
        \hline
        \textbf{Nozione} & \textbf{Conseguenza sulla pianificazione}\\
        \hline
        Potrebbe capitare che un'attività pianificata sia fonte di discussione durante il ciclo. & Bisogna prestare maggiore attenzione durante la fase di pianificazione e nel caso vengano riscontrate problematiche per evitare di perdere tempo, le attività pianificate devono essere portate avanti durante il ciclo e poi eventualmente discusse al ciclo successivo. \\
        \hline
    \end{tabularx}\\[8pt]
    \mbox{}\\
\end{center}

\end{document}
