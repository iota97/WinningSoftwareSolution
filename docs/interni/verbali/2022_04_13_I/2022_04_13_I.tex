\documentclass[a4paper, 12pt]{article}

\newcommand{\templates}{../../../template}
\usepackage[a4paper, margin=2.5cm]{geometry}

\usepackage{enumitem}
\setlist[itemize]{noitemsep}
\setlist[enumerate]{noitemsep}

\let\oldpar\paragraph
\renewcommand{\paragraph}[1]{\oldpar{#1\\}\noindent}
\input{\templates/front_page}
\usepackage{hyperref}
\usepackage{array}
\usepackage{tabularx}

\def\vers#1-#2-#3-#4-#5\\{#1&#2&#3&#4&#5\\\hline}

\newcommand{\addversione}[5]{
	\ifdefined\versioni
		\let\old\versioni
		\renewcommand{\versioni}{#1&#2&#3&#4&#5\\\hline\old}
	\else
		\newcommand{\versioni}{#1&#2&#3&#4&#5\\\hline}
	\fi
}

\newcommand{\setversioni}[1]{\newcommand{\versioni}{#1}}

\newcommand{\makeversioni}{
	\begin{center}
		\begin{tabularx}{\textwidth}{|c|c|c|c|X|}
		\hline
		\textbf{Versione} & \textbf{Data} & \textbf{Persona} & \textbf{Attivtà} & \textbf{Descrizione} \\
		\hline
		\versioni
		\end{tabularx}
	\end{center}
	\clearpage
}

\settitolo{Verbale del 13/04/2022}
\setredattori{Alberto Nicoletti}
\setrevisori{Andrea Volpe}
\setdestuso{interno}
\setdescrizione{
Verbale dell'incontro del 13/04/2022.
}

\begin{document}

\makefrontpage

\section{Informazioni}
\textbf{Data}: 13/04/2022\\
\textbf{Durata}: 70 minuti\\
\textbf{Luogo}: Server Discord\\

\textbf{Partecipanti:}
\begin{itemize}
	\item Matteo Galvagni;
	\item Federico Marchi;
	\item Alberto Nicoletti;
	\item Raffaele Oliviero;
	\item Elia Scandaletti;
	\item Andrea Volpe.
\end{itemize}


\section{Ordine del giorno}
\begin{itemize}
	\item analisi criticità emerse nel ciclo di sprint 3;
	\item organizzazione per la presentazione del colloquio PB;
	\item suddivisione tasks e organizzazione ciclo di sprint 4.
\end{itemize}

\section{Svolgimento}

\subsection{Criticità ciclo di sprint 3}
Durante il ciclo di sprint 3 non sono emerse alcune criticità.

\subsection{Presentazione PB}
Il colloquio con il prof. Cardin si svolgerà venerdì 15 alle 14:30 e verterà su una dimostrazione della demo di ShopChain (5 minuti) e la spiegazione dei punti salienti della nostra architettura (15 minuti).\\
Durante la presentazione Giovanni condividerà lo schermo per dimostrare il funzionamento della demo. Mostrerà anche le slide riguardo le SA. Matteo ne esporrà i concetti.\\
Venerdì mattina ci si trova su Discord per fare il punto sulla presentazione. Eventualmente ci si organizza per fare intervenire altri componenti.

\subsection{Tasks per il ciclo di sprint 4}
Il responsabile selezionato per il ciclo di sprint 4 è Alberto Nicoletti. \\
Sono state determinate le seguenti tasks per il ciclo di sprint 3:
\begin{enumerate}
    \item Scrivere uno script per la pubblicazione automatica dei documenti redatti;
    \item aggiornare la cartella public e le norme di progetto;
    \item correggere l'ortografia in alcuni documenti;
	\item finire manuale installazione
	\item sviluppare la grafica della web-app
	\item completare la web app (tasti copia, footer, metatag...)
	\item rendere accessibile la web-app
\end{enumerate}

\section{Impegni presi}
L'assegnazione delle tasks ai membri per il ciclo di sprint 4 è avvenuta come segue:
\begin{itemize}
	\item Elia Scandaletti: task 1, 2 e 3;
	\item Federico Marchi: task 4;
	\item Federico Marchi, Alberto Nicoletti, Raffaele Oliviero e Andrea Volpe: task 5, 6 e 7;
	\item Giovanni Cocco e Matteo Galvagni : revisione.
\end{itemize}

\end{document}