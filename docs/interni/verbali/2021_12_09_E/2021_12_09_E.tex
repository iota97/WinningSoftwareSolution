\documentclass[a4paper, 12pt]{article}

\newcommand{\templates}{../../../template}
\usepackage[a4paper, margin=2.5cm]{geometry}

\usepackage{enumitem}
\setlist[itemize]{noitemsep}
\setlist[enumerate]{noitemsep}

\let\oldpar\paragraph
\renewcommand{\paragraph}[1]{\oldpar{#1\\}\noindent}
\input{\templates/front_page}
\usepackage{hyperref}
\usepackage{array}
\usepackage{tabularx}

\def\vers#1-#2-#3-#4-#5\\{#1&#2&#3&#4&#5\\\hline}

\newcommand{\addversione}[5]{
	\ifdefined\versioni
		\let\old\versioni
		\renewcommand{\versioni}{#1&#2&#3&#4&#5\\\hline\old}
	\else
		\newcommand{\versioni}{#1&#2&#3&#4&#5\\\hline}
	\fi
}

\newcommand{\setversioni}[1]{\newcommand{\versioni}{#1}}

\newcommand{\makeversioni}{
	\begin{center}
		\begin{tabularx}{\textwidth}{|c|c|c|c|X|}
		\hline
		\textbf{Versione} & \textbf{Data} & \textbf{Persona} & \textbf{Attivtà} & \textbf{Descrizione} \\
		\hline
		\versioni
		\end{tabularx}
	\end{center}
	\clearpage
}

\settitolo{Verbale del 09/12/2021 con SyncLab}
\setredattori{Giovanni Cocco}
\setcommittenti{SyncLab}
\setprogetto{ShopChain}
\setdestuso{interno}
\setdescrizione{
Verbale dell'incontro del 09/12/2021 con l'azienda SyncLab. 
}

\begin{document}

\makefrontpage

\section{Informazioni}
\textbf{Data}: 09/12/2021\\
\textbf{Durata}: 25 minuti\\
\textbf{Luogo}: Aula Virtuale Google Meet\\

\textbf{Partecipanti}
\begin{itemize}
	\item Andrea Volpe
	\item Giovanni Cocco
	\item Federico Marchi
	\item Alberto Nicoletti
	\item Matteo Galvagni
	\item Elia Scandaletti
	\item Raffaele Oliviero
	\item Fabio Pallaro
\end{itemize}

\section{Incontro richiesto dal proponente}

\subsection{Server Discord}
Il proponente ci ha invitato nel server Discord aziendale dove avremo un canale dedicato e potremmo discutere del progetto sia tra noi che con i dipendenti dell'azienda.\\\\
Per accedere al server è richiesto che il nostro nome utente sia: \emph{Nome Cognome}.

\subsection{Casi d'uso}
Su nostra richiesta il proponente si è detto disponibile a verificare previo invio via mail il documento dei casi d'uso per controllare eventuali omissioni.

\end{document}
