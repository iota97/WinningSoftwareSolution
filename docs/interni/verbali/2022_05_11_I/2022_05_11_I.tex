\documentclass[a4paper, 12pt]{article}

\newcommand{\templates}{../../../template}
\usepackage[a4paper, margin=2.5cm]{geometry}

\usepackage{enumitem}
\setlist[itemize]{noitemsep}
\setlist[enumerate]{noitemsep}

\let\oldpar\paragraph
\renewcommand{\paragraph}[1]{\oldpar{#1\\}\noindent}
\input{\templates/front_page}
\usepackage{hyperref}
\usepackage{array}
\usepackage{tabularx}

\def\vers#1-#2-#3-#4-#5\\{#1&#2&#3&#4&#5\\\hline}

\newcommand{\addversione}[5]{
	\ifdefined\versioni
		\let\old\versioni
		\renewcommand{\versioni}{#1&#2&#3&#4&#5\\\hline\old}
	\else
		\newcommand{\versioni}{#1&#2&#3&#4&#5\\\hline}
	\fi
}

\newcommand{\setversioni}[1]{\newcommand{\versioni}{#1}}

\newcommand{\makeversioni}{
	\begin{center}
		\begin{tabularx}{\textwidth}{|c|c|c|c|X|}
		\hline
		\textbf{Versione} & \textbf{Data} & \textbf{Persona} & \textbf{Attivtà} & \textbf{Descrizione} \\
		\hline
		\versioni
		\end{tabularx}
	\end{center}
	\clearpage
}

\settitolo{Verbale del 11/05/2022}
\setredattori{Federico Marchi}
\setrevisori{Andrea Volpe}
\setdestuso{interno}
\setdescrizione{
Verbale dell'incontro del 11/05/2022.
}

\begin{document}

\makefrontpage

\section{Informazioni}
\textbf{Data}: 11/05/2022\\
\textbf{Ora}: 18:00\\
\textbf{Durata}: 20 minuti\\
\textbf{Luogo}: Server Discord\\

\textbf{Partecipanti:}
\begin{itemize}
  \item Giovanni Cocco;
  \item Federico Marchi;
	\item Elia Scandaletti;
	\item Andrea Volpe.
\end{itemize}


\section{Ordine del giorno}
\begin{itemize}
	\item resoconto tasks del ciclo di sprint 7;
  \item criticità ciclo di sprint 7;
\end{itemize}

\section{Svolgimento}

\subsection{Resoconto del ciclo di sprint 7}
Sono state svolte tutte le tasks del ciclo di sprint 7. I seguenti documenti sono stati sistemati seguendo le indicazioni del Professore Vardanega nell'esito della PB:
\begin{itemize}
  \item \textit{Piano di Progetto};
  \item \textit{Piano di Qualifica};
  \item \textit{Manuale acquirente};
  \item \textit{Manuale e-commerce};
  \item \textit{Manuale sviluppatore}.
\end{itemize}
Inoltre è stata effettuata la validazione la quale è avvenuta correttamente, sono stati dunque rispettati tutti i requisiti.
\subsection{Criticità ciclo di sprint 7}
Durante l'incontro con il proponente sono sorti problemi legati alle tecnologie utilizzate al momento della presentazione del prodotto realizzato. Durante il settimo ciclo di sprint è stata trovata una soluzione alternativa per ovviare ai problemi riscontrati.

\section{Impegni presi}
È stato deciso di contattare il Professore Vardanega con una mail nella mattinata del giorno 12/05/2022, per richiedere il colloquio per la CA.

\end{document}
