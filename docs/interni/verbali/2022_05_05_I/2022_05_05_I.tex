\documentclass[a4paper, 12pt]{article}

\newcommand{\templates}{../../../template}
\usepackage[a4paper, margin=2.5cm]{geometry}

\usepackage{enumitem}
\setlist[itemize]{noitemsep}
\setlist[enumerate]{noitemsep}

\let\oldpar\paragraph
\renewcommand{\paragraph}[1]{\oldpar{#1\\}\noindent}
\input{\templates/front_page}
\usepackage{hyperref}
\usepackage{array}
\usepackage{tabularx}

\def\vers#1-#2-#3-#4-#5\\{#1&#2&#3&#4&#5\\\hline}

\newcommand{\addversione}[5]{
	\ifdefined\versioni
		\let\old\versioni
		\renewcommand{\versioni}{#1&#2&#3&#4&#5\\\hline\old}
	\else
		\newcommand{\versioni}{#1&#2&#3&#4&#5\\\hline}
	\fi
}

\newcommand{\setversioni}[1]{\newcommand{\versioni}{#1}}

\newcommand{\makeversioni}{
	\begin{center}
		\begin{tabularx}{\textwidth}{|c|c|c|c|X|}
		\hline
		\textbf{Versione} & \textbf{Data} & \textbf{Persona} & \textbf{Attivtà} & \textbf{Descrizione} \\
		\hline
		\versioni
		\end{tabularx}
	\end{center}
	\clearpage
}

\settitolo{Verbale del 05/05/2022}
\setredattori{Elia Scandaletti}
\setrevisori{Matteo Galvagni}
\setdestuso{interno}
\setdescrizione{
Verbale dell'incontro del 05/05/2022.
}

\begin{document}

\makefrontpage

\section{Informazioni}
\textbf{Data}: 05/05/2022\\
\textbf{Ora}: 16:45\\
\textbf{Durata}: 40 minuti\\
\textbf{Luogo}: Server Discord\\

\textbf{Partecipanti:}
\begin{itemize}
	\item Matteo Galvagni;
	\item Federico Marchi;
	\item Elia Scandaletti.
\end{itemize}


\section{Ordine del giorno}
\begin{itemize}
	\item identificazione causa fallimento delle transazioni durante colloquio col proponente. Vedi verbale esterno del 05/05/2022.
\end{itemize}

\section{Svolgimento}

\subsection{Causa del fallimento}
Durante la riunione, facendo varie prove, è emerso che la causa di ciò sta nell'indirizzo RPC impostato nell'estensione MetaMask per la testnet Mumbai. Per risolvere il problema, è stato deciso di passare a una RPC privata. \\
Nel mentre, è emerso un problema che provoca occasionalmente il crash del server a causa di interruzioni alla connessione socket dal lato di Moralis. Per ovviare a questi rari casi, poiché modificare l'architettura costerebbe troppo, si è optato per un riavvio automatico del server.

Sono state create la seguente issue:
\begin{enumerate}[label=\textbf{\#\theenumi}]
  \setcounter{enumi}{191}
	\item far ripartire il server in caso di crash.
\end{enumerate}

\section{Impegni presi}
La task \#192 è stata assegnata a Giovanni Cocco. \\
Matteo Galvagni si è preso in carico il compito di trovare una soluzione che consenta di collegarsi a un server locale da remoto.

\end{document}
