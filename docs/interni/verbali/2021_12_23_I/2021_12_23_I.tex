\documentclass[a4paper, 12pt]{article}

\newcommand{\templates}{../../../template}
\usepackage[a4paper, margin=2.5cm]{geometry}

\usepackage{enumitem}
\setlist[itemize]{noitemsep}
\setlist[enumerate]{noitemsep}

\let\oldpar\paragraph
\renewcommand{\paragraph}[1]{\oldpar{#1\\}\noindent}
\input{\templates/front_page}
\usepackage{hyperref}
\usepackage{array}
\usepackage{tabularx}

\def\vers#1-#2-#3-#4-#5\\{#1&#2&#3&#4&#5\\\hline}

\newcommand{\addversione}[5]{
	\ifdefined\versioni
		\let\old\versioni
		\renewcommand{\versioni}{#1&#2&#3&#4&#5\\\hline\old}
	\else
		\newcommand{\versioni}{#1&#2&#3&#4&#5\\\hline}
	\fi
}

\newcommand{\setversioni}[1]{\newcommand{\versioni}{#1}}

\newcommand{\makeversioni}{
	\begin{center}
		\begin{tabularx}{\textwidth}{|c|c|c|c|X|}
		\hline
		\textbf{Versione} & \textbf{Data} & \textbf{Persona} & \textbf{Attivtà} & \textbf{Descrizione} \\
		\hline
		\versioni
		\end{tabularx}
	\end{center}
	\clearpage
}

\settitolo{Verbale del 23/12/2021}
\setprogetto{ShopChain}
\setcommittenti{SyncLab}
\setredattori{Andrea Volpe}
\setdestuso{interno}
\setdescrizione{
Verbale dell'incontro del 23/12/2021. 
}

\begin{document}

\makefrontpage

\section{Informazioni}

\begin{itemize}
\item Data: 23/12/2021
\item Ora: 15:30
\item Durata: 1:30
\item Luogo: Server Discord
\item Partecipanti: 
\begin{itemize}
\item Andrea Volpe
\item Giovanni Cocco
\item Federico Marchi
\item Alberto Nicoletti
\item Matteo Galvagni
\item Raffaele Oliviero
\end{itemize}
\end{itemize}

\section{Ordine del giorno}
\begin{enumerate}
\item Revisione collettiva del lavoro svolto dai vari sotto-team per la stesura dei documenti:
\begin{itemize}
\item Analisi dei requisiti
\item Analisi delle tecnologie
\item Piano di progetto e Piano di qualifica
\end{itemize}
\item Discussione dubbi su incontro con SyncLab del 22/12/2021.
\end{enumerate}

\section{Riassunto dei contenuti}
Andrea Volpe e Alberto Nicoletti ha illustrato agli altri membri del team i casi d'uso realizzati per il documento Analisi dei requisiti.
Sono state concordate con gli altri membri del team modifiche correttive ai casi d'uso realizzati.

Matteo Galvagni e Federico Marchi hanno terminato il documento Analisi delle tecnologie e Matteo Galvagni ha spiegato 
il lavoro svolto.

Elia Scandaletti e Federico Marchi stanno procedendo coi documenti piano di progetto e piano di qualifica, devono risolvere
alcuni dubbi per quanto riguarda il loro contenuto.

\paragraph{Sono stati discussi dubbi in merito all'incontro del 22/12/2021:}
\begin{itemize}
\item La soluzione proposta da SyncLab di far accettare manualmente al venditore i pagamenti degli acquirenti mediante 
una nostra landing page che ha il vantaggio di non modificare il backend dell'ecommerce, ci sembra una cosa poco funzionale
a livello di usabilità del sistema;
\item Come consigliato da SyncLab, salvare l'ecommerce e l'id del prodotto acquistato in blockchain è una cosa negativa a livello
di privacy dell'acquirente;
\item Linguaggi da preferire per realizzare il backend (tra Java e NodeJS);
\end{itemize}

\paragraph{Decisioni prese riguardo i dubbi:}
\begin{itemize}
\item Apportare delle modifiche al backend dell'ecommerce per installare uno script che valida in maniera automatica con la firma
del venditore i pagamenti ricevuti, porta a una piccola modifica iniziale del proprio ecommerce da parte del venditore
ma i vantaggi nell'usabilità a lungo termine sono superiori, dato che il venditore non deve più manualmente confermare o rifiutare
ogni transazione manualmente (potrebbero essere molte per ecommerce grandi). Quindi preferiamo la soluzione che installi uno 
script nel backend del venditore;
\item Non salviamo ecommerce e id prodotto in blockchain, ma le salviamo in un nostro database con id transazione in blockchain, id 
prodotto ed ecommerce;
\item Chiediamo a SyncLab quale linguaggio preferiscano venga da noi utilizzato;
\end{itemize}

\section{Impegni presi}
\begin{itemize}
\item Matteo Galvagni invierà una mail a SyncLab per avere conferma/risposta sulle decisioni prese nei punti 1, 2, 3.
\item Andrea Volpe e Alberto Nicoletti apporteranno le modifiche ai casi d'uso concordate con il team.
\end{itemize}

\end{document}
