\documentclass[a4paper, 12pt]{article}

\newcommand{\templates}{../../../template}
\usepackage[a4paper, margin=2.5cm]{geometry}

\usepackage{enumitem}
\setlist[itemize]{noitemsep}
\setlist[enumerate]{noitemsep}

\let\oldpar\paragraph
\renewcommand{\paragraph}[1]{\oldpar{#1\\}\noindent}
\input{\templates/front_page}
\usepackage{hyperref}
\usepackage{array}
\usepackage{tabularx}

\def\vers#1-#2-#3-#4-#5\\{#1&#2&#3&#4&#5\\\hline}

\newcommand{\addversione}[5]{
	\ifdefined\versioni
		\let\old\versioni
		\renewcommand{\versioni}{#1&#2&#3&#4&#5\\\hline\old}
	\else
		\newcommand{\versioni}{#1&#2&#3&#4&#5\\\hline}
	\fi
}

\newcommand{\setversioni}[1]{\newcommand{\versioni}{#1}}

\newcommand{\makeversioni}{
	\begin{center}
		\begin{tabularx}{\textwidth}{|c|c|c|c|X|}
		\hline
		\textbf{Versione} & \textbf{Data} & \textbf{Persona} & \textbf{Attivtà} & \textbf{Descrizione} \\
		\hline
		\versioni
		\end{tabularx}
	\end{center}
	\clearpage
}

\settitolo{Verbale del 11/01/2022}
\setprogetto{ShopChain}
\setcommittenti{SyncLab}
\setredattori{Raffaele Oliviero}
\setdestuso{interno}
\setdescrizione{
Verbale dell'incontro del 11/01/2022. 
}

\begin{document}

\makefrontpage

\section{Informazioni}

\begin{itemize}
\item Data: 11/01/2022
\item Ora: 15:00
\item Durata: 0:45
\item Luogo: Server Discord
\item Partecipanti: 
\begin{itemize}
\item Andrea Volpe
\item Giovanni Cocco
\item Alberto Nicoletti
\item Matteo Galvagni
\item Raffaele Oliviero
\end{itemize}
\end{itemize}

\section{Ordine del giorno}
\begin{enumerate}
\item Revisione del progresso della stesura dei documenti e pianificazione PoC.
\end{enumerate}

\section{Riassunto dei contenuti}

Andrea Volpe e Alberto Nicoletti hanno stimato che il tempo richiesto per terminare la stesura dei casi d'uso sarà di 3 giorni, mentre il documento di Analisi dei Requisiti potrà essere pronto per il 23/01/2022.


Matteo Galvagni ha dichiarato che il documento dell'Analisi delle Tecnologie è quasi terminato e deve solo essere revisionato.


Si è pianificata la PoC, a cui lavoreranno Matteo Galvagni, Giovanni Cocco e Federico Marchi. La data di consegna per la PoC è il 23/01/2022.


\section{Impegni presi}
\begin{itemize}
\item Andrea Volpe e Alberto Nicoletti termineranno la stesura dei documenti a loro assegnati.
\item Matteo Galvagni, Giovanni Cocco e Federico Marchi si occuperanno di creare la PoC.
\item Raffaele Oliviero aggiornerà il Piano di Progetto secondo quanto stabilito durante l'incontro.
\end{itemize}

\end{document}