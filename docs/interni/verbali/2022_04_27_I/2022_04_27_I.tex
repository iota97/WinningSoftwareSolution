\documentclass[a4paper, 12pt]{article}

\newcommand{\templates}{../../../template}
\usepackage[a4paper, margin=2.5cm]{geometry}

\usepackage{enumitem}
\setlist[itemize]{noitemsep}
\setlist[enumerate]{noitemsep}

\let\oldpar\paragraph
\renewcommand{\paragraph}[1]{\oldpar{#1\\}\noindent}
\input{\templates/front_page}
\usepackage{hyperref}
\usepackage{array}
\usepackage{tabularx}

\def\vers#1-#2-#3-#4-#5\\{#1&#2&#3&#4&#5\\\hline}

\newcommand{\addversione}[5]{
	\ifdefined\versioni
		\let\old\versioni
		\renewcommand{\versioni}{#1&#2&#3&#4&#5\\\hline\old}
	\else
		\newcommand{\versioni}{#1&#2&#3&#4&#5\\\hline}
	\fi
}

\newcommand{\setversioni}[1]{\newcommand{\versioni}{#1}}

\newcommand{\makeversioni}{
	\begin{center}
		\begin{tabularx}{\textwidth}{|c|c|c|c|X|}
		\hline
		\textbf{Versione} & \textbf{Data} & \textbf{Persona} & \textbf{Attivtà} & \textbf{Descrizione} \\
		\hline
		\versioni
		\end{tabularx}
	\end{center}
	\clearpage
}

\settitolo{Verbale del 27/04/2022}
\setredattori{Federico Marchi}
\setrevisori{Elia Scandaletti}
\setdestuso{interno}
\setdescrizione{
Verbale dell'incontro del 27/04/2022.
}

\begin{document}

\makefrontpage

\section{Informazioni}
\textbf{Data}: 27/04/2022\\
\textbf{Durata}: 70 minuti\\
\textbf{Luogo}: Server Discord\\

\textbf{Partecipanti:}
\begin{itemize}
	\item Matteo Galvagni;
	\item Alberto Nicoletti;
	\item Raffaele Oliviero;
	\item Elia Scandaletti;
  \item Federico Marchi;
	\item Andrea Volpe.
\end{itemize}


\section{Ordine del giorno}
\begin{itemize}
	\item resoconto tasks del ciclo di sprint 5;
  \item criticità ciclo di sprint 5;
	\item suddivisione tasks e organizzazione ciclo di sprint 6.
\end{itemize}

\section{Svolgimento}

\subsection{Resoconto del ciclo di sprint 5}
Per quanto riguarda il documento \textit{Manuale Sviluppatore} è stato ultimato e verificato.\\
É stata realizzata la presentazione per la PB e inoltre è avvenuto il colloquio per la PB in data 26/04/2022.\\
Per quanto riguarda invece la web app:
\begin{itemize}
  \item è stata sistemata l'accessibilità;
  \item è stato aggiornato il logo;
  \item sono stati aggiunti i tasti di copia per gli address di venditore e acquirente;
  \item è stato parzialmente sistemato il footer, il quale però richiede ulteriori modifiche;
  \item è stata eseguita la formattazione dei prezzi.
\end{itemize}
Tutte le tasks del ciclo di sprint 5 sono state completate.

\subsection{Criticità ciclo di sprint 5}
Non sono sorte criticità durante il quinto ciclo di sprint.


\subsection{Tasks per il ciclo di sprint 6}
Il responsabile selezionato per il ciclo di sprint 6 è Andrea Volpe. \\
Sono state determinate le seguenti tasks per il ciclo di sprint 6:
\begin{enumerate}
	\item modificare la visualizzazione delle transazioni;
  \item valutare ed eventualmente sistemare il link corrente nel menù di navigazione;
  \item aggiungere il breadcrumb;
  \item sistemare la visualizzazione della landing page;
  \item rendere i bottoni responsive;
  \item aggiungere l'id del prodotto nella lista di transazioni e dunque aggiornare il documento \textit{Analisi dei requisiti};
  \item aggiungere lo stato della transazione nella lista di transazioni;
  \item sistemare il footer: abbellire i link, aggiungere loghi xhtml, css e work on my machine.
\end{enumerate}

\section{Impegni presi}
I membri del gruppo Giovanni Cocco e Matteo Galvagni non eseguiranno alcuna tasks durante il sesto di ciclo di sprint poiché hanno già raggiunto un elevato numero di ore.
L'assegnazione delle tasks ai membri per il ciclo di sprint 6 è avvenuta come segue:
\begin{itemize}
  \item Federico Marchi: tasks 1 e 7;
  \item Elia Scandaletti: tasks 2 e 3;
  \item Raffaele Oliviero: tasks 5 e 6;
  \item Andrea Volpe: task 4;
  \item Alberto Nicoletti: tasks 7 e 8.
\end{itemize}
Inoltre il gruppo ha preso l'impegno di contattare il proponente.

\end{document}
