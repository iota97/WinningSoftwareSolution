\documentclass[a4paper, 12pt]{article}

\newcommand{\templates}{../../../template}
\usepackage[a4paper, margin=2.5cm]{geometry}

\usepackage{enumitem}
\setlist[itemize]{noitemsep}
\setlist[enumerate]{noitemsep}

\let\oldpar\paragraph
\renewcommand{\paragraph}[1]{\oldpar{#1\\}\noindent}
\input{\templates/front_page}
\usepackage{hyperref}
\usepackage{array}
\usepackage{tabularx}

\def\vers#1-#2-#3-#4-#5\\{#1&#2&#3&#4&#5\\\hline}

\newcommand{\addversione}[5]{
	\ifdefined\versioni
		\let\old\versioni
		\renewcommand{\versioni}{#1&#2&#3&#4&#5\\\hline\old}
	\else
		\newcommand{\versioni}{#1&#2&#3&#4&#5\\\hline}
	\fi
}

\newcommand{\setversioni}[1]{\newcommand{\versioni}{#1}}

\newcommand{\makeversioni}{
	\begin{center}
		\begin{tabularx}{\textwidth}{|c|c|c|c|X|}
		\hline
		\textbf{Versione} & \textbf{Data} & \textbf{Persona} & \textbf{Attivtà} & \textbf{Descrizione} \\
		\hline
		\versioni
		\end{tabularx}
	\end{center}
	\clearpage
}

\settitolo{Verbale del 01/12/2021}
\setprogetto{ShopChain}
\setcommittenti{SyncLab}
\setredattori{Elia Scandaletti}
\setdestuso{interno}
\setdescrizione{
Verbale dell'incontro del 01/12/2021. 
}

\begin{document}

\makefrontpage

\section{Informazioni}

\begin{itemize}
\item Data: 01/12/2021
\item Ora: 16:00
\item Durata: 1:40
\item Luogo: Server Discord
\item Partecipanti: 
\begin{itemize}
\item Matteo Galvagni
\item Federico Marchi
\item Raffaele Oliviero
\item Elia Scandaletti
\item Andrea Volpe
\end{itemize}
\end{itemize}

\section{Ordine del giorno}
\begin{enumerate}
\item Discussione sui prossimi obiettivi;
\item suddivisione e assegnazione dei compiti;
\item varie ed eventuali.
\end{enumerate}

\section{Riassunto dei contenuti}

Tutti i partecipanti hanno concordato la necessità di procedere nei seguenti lavori:
\begin{itemize}
\item analisi dei requisiti;
\item analisi delle tecnologie;
\item stesura dei piani di qualifica e di progetto.
\end{itemize}

Per svolgere questi compiti si è deciso di dividersi in tre gruppi di lavoro.
È anche stato ritenuto necessario effettuare un incontro di verifica dell'avanzamento dei lavori intermedio in cui ogni gruppo esporrà il punto della situazione in riferimento al metodo utilizzato e al progresso raggiunto.

Come scadenza per il completamento delle analisi dei requisiti e delle tecnologie e l'aggiornamento dei piani di qualifica e di progetto è stato scelto il 23/12/2021. Come data per il controllo intermedio è stato scelto il 15/12/2021.

Si è discusso a lungo sulla possibilità di avere un verificatore per ciascun gruppo di lavoro, piuttosto che un verificatore per tutti i gruppi. Nel primo caso, i verificatori avrebbero dovuto ricoprire anche altri ruoli in contemporanea. Alla fine ha prevalso l'idea di un verificatore unico.

Di seguito sono elencati i gruppi di lavoro, i loro componenti e i loro compiti:
\begin{itemize}
\item \textbf{Analisi dei requisiti:}
Il gruppo, composto da due analisti, si occuperà dell'analisi dei requisiti e della stesura del relativo documento. È mandatorio per il gruppo il confronto con il proponente.

\item \textbf{Analisi delle tecnologie:}
Il gruppo, composto da due progettisti, si occuperà di scegliere le tecnologie, i framework e le librerie da utilizzare nel progetto. Inoltre, il gruppo dovrà documentare le scelte fatte, con particolare attenzione alle metriche usate.

Nell'analisi delle tecnologie sarà centrale la scelta della blockchain da utlizzare, come da precedenti accordi col proponente.

\item \textbf{Piani di qualifica e di progetto:}
Il gruppo, composto da un responsabile e un amministratore, si occupa della stesura dei piani di qualifica e di progetto e del loro aggiornamento fino al 23/12/2021.

Il gruppo si occuperà anche dell'eventuale aggiornamento delle norme di progetto.

\item \textbf{Verifica:}
Il verificatore dovrà controllare che tutto il materiale prodotto sia conforme alle norme di progetto e coerente con sè stesso.
\end{itemize}

Infine, Andrea Volpe ha suggerito l'uso di GitHub Issues per tenere traccia dei TODO e dell'avanzamento del progetto. Dopo alcune perplessità sul funzionamento della piattaforma, si è concordato che Andrea approfondirà l'argomento; successivamente relazionerà al gruppo le funzionalità nel dettaglio e per decidere insieme le norme da utilizzare.

\section{Impegni presi}
\paragraph{Prossimi incontri}
\begin{itemize}
\item \textbf{15/12:} incontro intermedio di verifica di avanzamento;
\item \textbf{23/12:} incontro per la condivisione e l'esposizione dei documenti prodotti.
\end{itemize}

\paragraph{Assegnazione dei ruoli:}
\begin{itemize}
\item analisi dei requisiti:
	\begin{itemize}
	\item Andrea Volpe: analista;
	\item Alberto Nicoletti: analista;
	\end{itemize}
\item analisi delle tecnologie:
	\begin{itemize}
	\item Matteo Galvagni: progettista;
	\item Federico Marchi: progettista;
	\end{itemize}
\item piani di progetto e di qualifica:
	\begin{itemize}
	\item Elia Scandaletti: responsabile;
	\item Raffaele Oliviero: amministratore;
	\end{itemize}
\item verifica:
	\begin{itemize}
	\item Giovanni Cocco: verificatore.
	\end{itemize}
\end{itemize}

\paragraph{Uso dei branch}
Ogni gruppo creerà un suo branch su GitHub e ogni membro del gruppo sarà libero di farci dei merge. I merge del branch del gruppo sul branch main potranno essere fatti solo dal verificatore, a seguito di una pull request da un membro del gruppo.

\paragraph{GitHub Issues}
Andrea Volpe si impegna ad approfondire il funzionamento di GitHub Issues e a relazionare al gruppo quanto appreso.

\end{document}
