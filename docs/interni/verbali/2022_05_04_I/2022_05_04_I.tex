\documentclass[a4paper, 12pt]{article}

\newcommand{\templates}{../../../template}
\usepackage[a4paper, margin=2.5cm]{geometry}

\usepackage{enumitem}
\setlist[itemize]{noitemsep}
\setlist[enumerate]{noitemsep}

\let\oldpar\paragraph
\renewcommand{\paragraph}[1]{\oldpar{#1\\}\noindent}
\input{\templates/front_page}
\usepackage{hyperref}
\usepackage{array}
\usepackage{tabularx}

\def\vers#1-#2-#3-#4-#5\\{#1&#2&#3&#4&#5\\\hline}

\newcommand{\addversione}[5]{
	\ifdefined\versioni
		\let\old\versioni
		\renewcommand{\versioni}{#1&#2&#3&#4&#5\\\hline\old}
	\else
		\newcommand{\versioni}{#1&#2&#3&#4&#5\\\hline}
	\fi
}

\newcommand{\setversioni}[1]{\newcommand{\versioni}{#1}}

\newcommand{\makeversioni}{
	\begin{center}
		\begin{tabularx}{\textwidth}{|c|c|c|c|X|}
		\hline
		\textbf{Versione} & \textbf{Data} & \textbf{Persona} & \textbf{Attivtà} & \textbf{Descrizione} \\
		\hline
		\versioni
		\end{tabularx}
	\end{center}
	\clearpage
}

\settitolo{Verbale del 04/05/2022}
\setredattori{Elia Scandaletti}
\setrevisori{Andrea Volpe}
\setdestuso{interno}
\setdescrizione{
Verbale dell'incontro del 04/05/2022.
}

\begin{document}

\makefrontpage

\section{Informazioni}
\textbf{Data}: 04/05/2022\\
\textbf{Ora}: 18:00\\
\textbf{Durata}: 70 minuti\\
\textbf{Luogo}: Server Discord\\

\textbf{Partecipanti:}
\begin{itemize}
	\item Matteo Galvagni;
	\item Alberto Nicoletti;
	\item Raffaele Oliviero;
	\item Elia Scandaletti;
	\item Andrea Volpe.
\end{itemize}


\section{Ordine del giorno}
\begin{itemize}
	\item resoconto tasks del ciclo di sprint 6;
  \item criticità ciclo di sprint 6;
	\item suddivisione tasks e organizzazione ciclo di sprint 7.
\end{itemize}

\section{Svolgimento}

\subsection{Resoconto del ciclo di sprint 6}
Svolgendo la task 5, è emerso che non è fattibile implementare i bottoni di copia degli indirizzi sulle piattaforme mobile. Questo a causa delle limitazioni imposte dalle tecnologie utilizzate. Per questo motivo la task è stata completata parzialmente. La task è comunque da considerarsi completata, perché il costo necessario a risolvere il problema sarebbe troppo grande. \\
Tutte le altre tasks del sesto ciclo di sprint sono state completate.

\subsection{Criticità ciclo di sprint 6}
Durante lo scorso ciclo, alcuni membri del gruppo hanno contestato una task decisa nella precedente riunione settimanale. Questo ha portato a una discussione all'interno del gruppo sul fatto di portare a termine o meno tale task. Al di là del suo contenuto, la discussione ha rallentato lo sviluppo della feature in questione. Per questo motivo si è deciso che quanto concordato in una riunione non sarà messo in discussione fino alla successiva. In caso emergessero contestazioni o criticità, si dovrà fare riferimento al responsabile che deciderà come agire.


\subsection{Tasks per il ciclo di sprint 7}
Il responsabile per il ciclo di sprint 7 è Elia Scandaletti. \\
Sono state determinate le seguenti tasks per il ciclo di sprint 7:
\begin{enumerate}[label=\textbf{\#\theenumi}]
  \setcounter{enumi}{179}
	\item completamento manuali;
  \item analisi narrativa d'insieme \textit{Piano di Progetto};
  \item analisi narrativa d'insieme \textit{Piano di Qualifica};
  \item sistemazione cartella \texttt{public} come da esito PB;
  \item aggiornamento cronologia delle modifiche dei documenti;
  \item modificare SCRUM in Scrum;
  \item validazione.
\end{enumerate}

\section{Impegni presi}
L'assegnazione delle tasks ai membri per il ciclo di sprint 7 è avvenuta come segue:
\begin{itemize}
  \item Matteo Galvagni: task \#185;
  \item Federico Marchi: tasks \#180 e \#186;
  \item Alberto Nicoletti: task \#181;
  \item Raffaele Oliviero: task \#182;
  \item Elia Scandaletti: tasks \#184 e \#186;
  \item Andrea Volpe: tasks \#183 e \#186.
\end{itemize}

\end{document}
