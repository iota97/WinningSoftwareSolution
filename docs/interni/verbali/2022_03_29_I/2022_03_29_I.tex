\documentclass[a4paper, 12pt]{article}

\newcommand{\templates}{../../../template}
\usepackage[a4paper, margin=2.5cm]{geometry}

\usepackage{enumitem}
\setlist[itemize]{noitemsep}
\setlist[enumerate]{noitemsep}

\let\oldpar\paragraph
\renewcommand{\paragraph}[1]{\oldpar{#1\\}\noindent}
\input{\templates/front_page}
\usepackage{hyperref}
\usepackage{array}
\usepackage{tabularx}

\def\vers#1-#2-#3-#4-#5\\{#1&#2&#3&#4&#5\\\hline}

\newcommand{\addversione}[5]{
	\ifdefined\versioni
		\let\old\versioni
		\renewcommand{\versioni}{#1&#2&#3&#4&#5\\\hline\old}
	\else
		\newcommand{\versioni}{#1&#2&#3&#4&#5\\\hline}
	\fi
}

\newcommand{\setversioni}[1]{\newcommand{\versioni}{#1}}

\newcommand{\makeversioni}{
	\begin{center}
		\begin{tabularx}{\textwidth}{|c|c|c|c|X|}
		\hline
		\textbf{Versione} & \textbf{Data} & \textbf{Persona} & \textbf{Attivtà} & \textbf{Descrizione} \\
		\hline
		\versioni
		\end{tabularx}
	\end{center}
	\clearpage
}

\settitolo{Verbale del 29/03/2022}
\setredattori{Raffaele Oliviero}
\setrevisori{Federico Marchi}
\setdestuso{interno}
\setdescrizione{
Verbale dell'incontro del 29/03/2022.
}

\begin{document}

\makefrontpage

\section{Informazioni}
\textbf{Data}: 29/03/2022\\
\textbf{Durata}: 1 ora\\
\textbf{Luogo}: Server Discord\\

\textbf{Partecipanti:}
\begin{itemize}
	\item Giovanni Cocco;
	\item Matteo Galvagni;
	\item Elia Scandaletti;
	\item Federico Marchi;
	\item Raffaele Oliviero;
	\item Andrea Volpe.
\end{itemize}


\section{Ordine del giorno}
\begin{itemize}
	\item analisi criticità emerse nel ciclo di sprint 1;
	\item suddivisione tasks e organizzazione ciclo di sprint 2.
\end{itemize}

\section{Svolgimento}

\subsection{Criticità ciclo di sprint 1}
Durante il consuntivo del ciclo di sprint 1 è emerso che il rapporto tra le ore effettive e le ore produttive è ancora alto, e i membri del gruppo devono impegnarsi ad abbassarlo.


\subsection{Tasks per il ciclo di sprint 2}
Il responsabile selezionato per il ciclo di sprint 2 è Matteo Galvagni. \\
Sono state determinate le seguenti tasks per il ciclo di sprint 2:
\begin{enumerate}
	\item ultimare la stesura dei documenti \textit{Manuale Acquirente} e \textit{Manuale e-commerce} e iniziare la stesura del \textit{Manuale di Installazione};
	\item continuare la stesura del documento \textit{Specifiche Architetturali}, in particolare dando peso ai pro e ai contro delle scelte adottate;
	\item aggiornare il documento \textit{Analisi dei Requisiti} che presenta ancora delle inesattezze e aggiungere i casi d'uso dell'annullamento della transazione da parte dell'e-commerce e del refund manuale dell'utente.
\end{enumerate}

\section{Impegni presi}
L'assegnazione delle tasks ai membri per il ciclo di sprint 2 è avvenuta come segue:
\begin{itemize}
	\item Andrea Volpe, Alberto Nicoletti e Raffaele Oliviero: task 3;
	\item Matteo Galvagni e Elia Scandaletti: task 2;
	\item Giovanni Cocco: Verifica dei prodotti sviluppati durante il ciclo di sprint;
	\item Federico Marchi: task 1;
\end{itemize}

\end{document}