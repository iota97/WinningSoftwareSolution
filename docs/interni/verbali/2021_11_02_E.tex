\documentclass[a4paper,12pt]{article}
\author{}
\date{}

\begin{document}
\title{Verbale del 2/11/2021 con Sanmarco Informatica}
\maketitle

\section{Informazioni}
\textbf{Data}: 2/11/2021\\
\textbf{Durata}: 25 minuti\\
\textbf{Luogo}: Aula virtuale \textit{Google Meet}\\\\
\textbf{Partecipanti}:
\begin{itemize}
	\item Albero Nicoletti
	\item Andrea Volpe
	\item Matteo Galvagni
	\item Raffaele Oliviero
    \item Giovanni Cocco
    \item Marco Magnabosco
\end{itemize}

\section{Ordine del giorno}
Incontro conoscitivo, chiarire ciò che è proposto nel capitolato.

\section{Riassunto dei contenuti}

\begin{itemize}
\item Dopo una breve presentazione, i componenti del gruppo hanno posto delle domande al proponente.

\item \textbf{Quantità di dati sottoposta all'analisi.} Il proponente ha spiegato che la quantità può essere variabile, ma un esempio realistico è: 10 linee di produzione con 10 pezzi al secondo, ogni pezzo con 10 misure: 1000 misurazioni al secondo.

\item \textbf{Generazione carte di controllo.} Il dubbio era tra generazione automatica o solo alla richiesta dell'utente. Il proponente ha chiarito che l'analisi è fatta in automatico all'arrivo di nuovi dati da analizzare. Le relative carte di controllo sono quindi in costante aggiornamento, anche per controllare fenomeni di tendenze, ma mostrate graficamente all'utente solo su richiesta

\item \textbf{Tipologia sono i dati da analizzare.} Anche se le misurazioni nella produzione possono essere fatte su vari prodotti, comunque si tratta di misure simili tra loro (esempio: diametro di un foro, lunghezza di un lato). Chiaramente, vengono prodotti svariati pezzi per ogni tipo per avere un campione di dati sufficente su cui misurare una tendenza statistica.
\end{itemize}

\end{document}