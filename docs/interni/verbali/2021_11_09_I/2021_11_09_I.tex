\documentclass[a4paper,12pt]{article}
\author{}
\date{}
\begin{document}

% Verbale del 09/11/2021
% Stesura documento: Alberto, revisione e conversione in latex: Andrea

\title{Verbale del 9/11/2021}
\maketitle

\section{Informazioni}

\begin{itemize}
\item Data: 9/11/2021
\item Ora: 16:00
\item Durata: 3 ore
\item Luogo: Server Discord
\item Partecipanti: 
\begin{itemize}
\item Alberto Nicoletti
\item Andrea Volpe
\item Elia Scandaletti
\item Federico Marchi
\item Raffaele Oliviero
\item Matteo Galvagni
\end{itemize}
\end{itemize}

\section{Ordine del giorno}
Definire i primi documenti da stilare e gli strumenti da usare.

\section{Riassunto dei contenuti}

\begin{itemize}
\item Si è scelto di usare Github come repository, Latex come linguaggio per produrre i documenti. Si è valutato Overleaf come editor per i documenti. Questa possibilità è stata scartata dato che overleaf è risultato un software con licenza a pagamento. Si è preferito l'utilizzo dell'editing locale ed individuale, poi condividento i documenti nella repository.
\item Si è iniziato a compilare il documento di norme di progetto.
\item Si è definita la necessità di ulteriori 2 incontri prima della presentazione della candidatura.
\item Si è diviso il lavoro di compilazione di verbali degli incontri con i proponenti in gruppi: Raffaele ed Alberto per il capitolato C3, Andrea, Elia e Matteo per il capitolato C2.
\item Si sono definiti i documenti da presentare per la candidatura: verbali incontri con proponenti, preventivo costi, impegno di consegna.
\end{itemize}

\section{Impegni presi}
\begin{itemize}
\item{Stilare i verbali degli incontri con proponenti}
\end{itemize}

\end{document}
