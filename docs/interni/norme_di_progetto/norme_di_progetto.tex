\documentclass[a4paper, 12pt]{article}

\usepackage[a4paper, margin=2.5cm]{geometry}

\usepackage{enumitem}
\setlist[itemize]{noitemsep}
\setlist[enumerate]{noitemsep}

\let\oldpar\paragraph
\renewcommand{\paragraph}[1]{\oldpar{#1\\}\noindent}
\input{../../template/front_page}
\usepackage{hyperref}
\usepackage{array}
\usepackage{tabularx}

\def\vers#1-#2-#3-#4-#5\\{#1&#2&#3&#4&#5\\\hline}

\newcommand{\addversione}[5]{
	\ifdefined\versioni
		\let\old\versioni
		\renewcommand{\versioni}{#1&#2&#3&#4&#5\\\hline\old}
	\else
		\newcommand{\versioni}{#1&#2&#3&#4&#5\\\hline}
	\fi
}

\newcommand{\setversioni}[1]{\newcommand{\versioni}{#1}}

\newcommand{\makeversioni}{
	\begin{center}
		\begin{tabularx}{\textwidth}{|c|c|c|c|X|}
		\hline
		\textbf{Versione} & \textbf{Data} & \textbf{Persona} & \textbf{Attivtà} & \textbf{Descrizione} \\
		\hline
		\versioni
		\end{tabularx}
	\end{center}
	\clearpage
}

\settitolo{Norme di progetto}
\setredattori{Elia Scandaletti}
\setversione{0.0.0}
\setdestuso{interno}
\setdescrizione{
Questo documento contiene le procedure, gli strumenti e i criteri di qualità che verranno usati nel progetto. 
}

\addversione{0.0.0}{13/11/2021}{Elia Scandaletti}{Redazione}{Stesura \ref{documentazione}}
\addversione{0.0.1}{14/11/2021}{Elia Scandaletti}{Redazione}{Stesura \ref{configurazione-github}}

\begin{document}

\makefrontpage

\makeversioni

\section{Introduzone}
\subsection{Scopo del documento}
Lo scopo del presente documento è definire procedure, strumenti e criteri di qualità al fine di stabilire un Way of Working. Ogni membro del gruppo sarà tenuto a rispettare le indicazioni qui presentate.

\section{Processi di supporto}

\subsection{Documentazione}\label{documentazione}

\subsubsection{Strumenti utilizzati}
Per la stesura dei documenti verrà utilizzato il linguaggio \LaTeX.

\subsubsection{Ciclo di vita}
Ogni documento passa per le seguenti fasi di vita:
\begin{itemize}
\item \textbf{Pianificazione:} il documento viene discusso e delineato per sommi capi sulla base delle necessità a cui deve rispondere;
\item \textbf{Redazione:} un membro del gruppo, nel ruolo di redattore, scrive fisicamente il documento;
\item \textbf{Revisione:} un membro del gruppo, nel ruolo di revisore, controlla che non siano presenti errori grammaticali e che sia aderente alle norme di progetto;
\item \textbf{Approvazione:} un membro del gruppo, nel ruolo di responsabile, verifica che il contenuto del documento sia corretto e coerente con lo scopo dello stesso.
\end{itemize}

Non è possibile per un membro del gruppo coprire più ruoli nella gestione dello stesso documento.

Una volta che un documento è revisionato o approvato può comunque tornare in fase di stesura sulla base delle esigenze del momento.

\subsubsection{Versionamento}
Ogni documento è univocamente identificabile dal titolo e dalla versione. La versione è definita da tre numeri nel formato \texttt{$<$x$>$.$<$y$>$.$<$z$>$}.

La prima versione è la \texttt{0.0.0}

Il numero \texttt{$<$z$>$} viene modificato ogni volta che viene effettuata una stesura del documento.

Il numero \texttt{$<$y$>$} viene modificato ogni volta che viene effettuata una revisione del documento.

Il numero \texttt{$<$x$>$} viene modificato ogni volta che viene effettuata un'approvazione del documento.

Se il numero \texttt{$<$x$>$} viene modificato, \texttt{$<$y$>$} e \texttt{$<$z$>$} tornano a 0.

Se il numero \texttt{$<$y$>$} viene modificato, \texttt{$<$z$>$} torna a 0.

I verbali sono esclusi dal sistema di versionamento poiché non possono essere modificati nel tempo.

\subsubsection{Organizzazione dei file}
Tutta la documentazione sarà contenuta all'interno della cartella \texttt{docs/}. La cartella è a sua volta divisa in due sottocartelle \texttt{docs/interni/} e \texttt{docs/esterni/}, che conterranno rispettivamente la documentazione interna al gruppo e quella da condividere con gli altri stakeholders.

Tutti i verbali saranno conservati nella sottocartella \texttt{interni/verbali/}.

Per ogni documento sarà presente una cartella che sarà chiamata \texttt{$<$nome$>$/}, dove \texttt{$<$nome$>$} è il nome del documento come specificato nella sezione \ref{documentazione-documenti}.

Il file principale di ogni documento sarà all'interno della rispettiva cartella e sarà chiamato \texttt{$<$nome$>$.tex}, dove \texttt{$<$nome$>$} è il nome del file come definito nella sezione \ref{documentazione-documenti}.

Eventuali altri file utilizzati nella stesura del documento saranno posizionati all'interno della stessa cartella.

\subsubsection{Struttura dei documenti}
Ogni documento avrà una pagina di intestazione e una pagina con l'elenco delle versioni, laddove il versionamento del documento è ammesso.
Sarà inserita una pagina con l'indice, a meno di indicazione contraria nella sezione \ref{documentazione-documenti}.

\paragraph{Pagina di intestazione}
La pagina di intestazione contiene:
\begin{itemize}
\item logo e nome del gruppo;
\item nome del progetto e azienda cliente;
\item contatto;
\item titolo del documento;
\item informazioni sul documento:
\begin{itemize}
	\item elenco dei redattori;
	\item elenco dei revisori;
	\item elenco dei responsabili;
	\item versione, laddove prevista;
	\item destinazione d'uso;
\end{itemize}
\item breve descrizione.
\end{itemize}

La destinazione d'uso può essere ``interno'' o ``esterno''.

Ogni documento userà per la prima pagina il template \texttt{docs/template/front\_page.tex}.

Eventuali risorse che dovessero servire per il template, saranno nella stessa cartella.

\paragraph{Elenco delle versioni}
L'elenco delle versioni è una tabella con le seguenti colonne:
\begin{itemize}
\item versione;
\item data;
\item persona;
\item attività;
\item descrizione.
\end{itemize}

Con persona si intende il membro del gruppo che svolge l'attività.

Le possibili attività su un documento sono ``Redazione'', ``Revisione'' e ``Approvazione''.

Le righe della tabella sono in ordine cronologico inverso.

\paragraph{Stile}
Ogni documento dovrà includere il file \texttt{docs/template/style.tex} che contiene indicazioni di stile.



\subsubsection{Convenzioni utilizzate}
\paragraph{Riferimenti a file e cartelle}
Per indicare il nome di un file o di una cartella si utilizza il  \texttt{testo monospaziato}.

Il nome di una cartella termina sempre con \texttt{/}.

\paragraph{Stringhe e nomi}
Le stringhe vengono scritte tra doppi apici `` ''.

Per indicare un parametro in un nome o una stringa si usa del testo tra parentesi angolari $<$ $>$. Il parametro non può contenere spazi.

Per indicare una parte di nome o stringa opzionale si usa del testo tra parentesi quadre [ ].

\paragraph{Riferimenti tra documenti}
Per riferirsi a sezioni all'interno dello stesso documento si usa il comando \verb-\ref-.

Per riferirsi a sezioni di un altro documento si usa il nome di quella sezione scritto in corsivo.

\subsubsection{Documenti}\label{documentazione-documenti}

\paragraph{Norme di progetto}
\subparagraph{Scopo}
Lo scopo delle norme di progetto è definire procedure, strumenti e criteri di qualità al fine di stabilire un Way of Working. Ogni membro del gruppo sarà tenuto a rispettare le indicazioni lì presentate.
\subparagraph{Titolo}
Il titolo del documento è ``Norme di progetto".
\subparagraph{Nome del file}
Il file sarà chiamato \texttt{norme\_di\_progetto.tex}.

\paragraph{Verbali}
\subparagraph{Scopo}
Lo scopo dei verbali è tenere traccia del dialogo interno ed esterno al gruppo e delle eventuali decisioni prese.
\subparagraph{Titolo}
Ogni verbale sarà titolato ``Verbale del $<$data$>$[ con $<$esterni$>$]", dove $<$data$>$ indica la data dell'incontro nel formato europeo. In caso all'incontro siano presenti persone esterne al gruppo, verrà usata anche la dicitura ``con $<$esterni$>$", dove $<$esterni$>$ indica i partecipanti alla riunione separati da virgole. Gli esterni possono essere identificati collettivamente tramite il nome dell'azienda o ente che rappresentano, tramite nome e cognome o tramite titolo e cognome.

Esempi:
\begin{itemize}
\item ``Verbale del 09/11/2021'';
\item ``Verbale del 28/10/2021 con SyncLab'';
\item ``Verbale del 25/12/2021 con prof. Cardin''.
\end{itemize}
\subparagraph{Nome dei file}
I file saranno chiamati \texttt{$<$data$>$\_$<$tipo$>$.tex}, dove \texttt{$<$data$>$} è la data dell'incontro in formato \texttt{$<$yyyy$>$\_$<$mm$>$\_$<$dd$>$} e \texttt{$<$tipo$>$} è \texttt{I} se non sono presenti persone esterne al gruppo, \texttt{E} altrimenti.
\subparagraph{Indice}
Non è presente una pagina con l'indice.
\subparagraph{Struttura}
Ogni verbale deve riportare:
\begin{itemize}
\item data, ora e durata;
\item luogo;
\item partecipanti;
\item ordine del giorno;
\item riassunto dei contenuti;
\item eventuali impegni assunti.
\end{itemize}

\subsection{Configurazione}

\subsubsection{Repository GitHub}\label{configurazione-github}
Per il versionamento del progetto si è scelto di utilizzare git su piattaforma GitHub.

La repository \url{https://github.com/iota97/WinningSoftwareSolution} conterrà tutti i file del progetto.

\paragraph{Workflow di lavoro}
Ogni modifica alla repository dovrà seguire i seguenti passaggi:
\begin{itemize}
\item modifica del codice su repository locale;
\item push su un branch ad hoc nominato \texttt{$<$iniziali\_autore$>$\_$<$descrizione$>$};
\item pull request tramite piattaforma web GitHub;
\item review della pull request;
\item merge sul branch \texttt{main}.
\end{itemize}

Prima di effettuare un push alla repository remota o un merge al branch \texttt{main} è necessario fare uno squash dei commit superflui.

Chi effettua la modifica, chi fa la review e chi accetta il merge devono essere tre membri del gruppo distinti.

Dopo che è stato fatto il merge di una pull request, il corrispondente branch va eliminato.

È consentito modifcare solamente i propri branch.

Non è consentito fare il caricamento diretto di file tramite interfaccia web o desktop.

Non è consentito forzare i push sulla repository remota.

\paragraph{Comandi utili}
Alcuni comandi utili sono:
\begin{itemize}
\item \textbf{Sincronizzazione con repository remota:} \texttt{git pull};
\item \textbf{Creazione di un nuovo branch}: \texttt{git branch $<$nome\_branch$>$};
\item \textbf{Passaggio a un branch specificata:} \texttt{git checkout $<$nome\_branch$>$};
\item \textbf{Aggiunta delle modifiche alla stage area:} \texttt{git add .};
\item \textbf{Creazione del commit con le modifiche:} \texttt{git commit};
\item \textbf{Push sul remote di un nuovo branch:} \texttt{git push --set-upstream origin $<$nome\_branch$>$};
\item \textbf{Push su un branch già esistente:} \texttt{git push}.
\end{itemize}

\end{document}