\documentclass[a4paper, 12pt]{article}

\newcommand{\templates}{../../template}
\usepackage[hidelinks]{hyperref}
\usepackage[a4paper, margin=2.5cm]{geometry}

\usepackage{enumitem}
\setlist[itemize]{noitemsep}
\setlist[enumerate]{noitemsep}

\let\oldpar\paragraph
\renewcommand{\paragraph}[1]{\oldpar{#1\\}\noindent}
\input{\templates/front_page}
\usepackage{hyperref}
\usepackage{array}
\usepackage{tabularx}

\def\vers#1-#2-#3-#4-#5\\{#1&#2&#3&#4&#5\\\hline}

\newcommand{\addversione}[5]{
	\ifdefined\versioni
		\let\old\versioni
		\renewcommand{\versioni}{#1&#2&#3&#4&#5\\\hline\old}
	\else
		\newcommand{\versioni}{#1&#2&#3&#4&#5\\\hline}
	\fi
}

\newcommand{\setversioni}[1]{\newcommand{\versioni}{#1}}

\newcommand{\makeversioni}{
	\begin{center}
		\begin{tabularx}{\textwidth}{|c|c|c|c|X|}
		\hline
		\textbf{Versione} & \textbf{Data} & \textbf{Persona} & \textbf{Attivtà} & \textbf{Descrizione} \\
		\hline
		\versioni
		\end{tabularx}
	\end{center}
	\clearpage
}

\settitolo{Norme di progetto}
\setprogetto{ShopChain}
\setcommittenti{SyncLab}
\setredattori{Elia Scandaletti \\ Raffaele Oliviero \\ Giovanni Cocco \\ Alberto Nicoletti}
\setdestuso{interno}
\setdescrizione{
Questo documento contiene le procedure, gli strumenti e i criteri di qualità che verranno usati nel progetto. 
}

\addversione{0.0.0}{13/11/2021}{Elia Scandaletti}{Redazione}{Stesura \ref{documentazione}}
\addversione{0.0.1}{14/11/2021}{Elia Scandaletti}{Redazione}{Stesura \ref{configurazione-github}}
\addversione{0.0.2}{07/12/2021}{Elia Scandaletti}{Redazione}{Inserimento norme da verbale del 01/12/2021}
\addversione{0.0.3}{22/12/2021}{Elia Scandaletti}{Redazione}{Stesura \ref{documentazione-documenti-verbali-stesura}}
\addversione{0.0.4}{09/01/2022}{Raffaele Oliviero}{Redazione}{Ristrutturazione documento}
\addversione{0.0.5}{09/01/2022}{Raffaele Oliviero}{Redazione}{Stesura \ref{documentazione-documenti-piano-di-qualifica} e \ref{verifica-e-validazione}}
\addversione{0.0.6}{16/01/2022}{Elia Scandaletti}{Redazione}{Adeguamento di \ref{verifica-e-validazione} a decisioni collettive}
\addversione{0.0.7}{04/02/2022}{Giovanni Cocco}{Redazione}{Stesure di \ref{revisioni-di-avanzamento-interne}}
\addversione{0.0.8}{08/02/2022}{Raffaele Oliviero}{Redazione}{Stesura di \ref{analisi-dei-requisiti}, \ref{analisi-delle-tecnologie} e \ref{piano-di-progetto}}
\addversione{0.0.9}{16/02/2022}{Giovanni Cocco}{Redazione}{Stesura di \ref{norme-sw}}
\addversione{2.0.0}{20/03/2022}{Alberto Nicoletti}{Redazione}{Rifacimento documento e stesura \ref{fornitura}}

\begin{document}

\makefrontpage

\makeversioni

\tableofcontents
\pagebreak

\section{Introduzione}
\subsection{Scopo del documento}
Lo scopo del presente documento è definire procedure, strumenti e criteri di qualità al fine di stabilire un Way of Working. Ogni membro del gruppo sarà tenuto a rispettare le indicazioni qui presentate.

\section{Processi Primari}
\subsection{Fornitura}\label{fornitura}
Il processo di fornitura determina le attività e le risorse che servo per gestire e assicurare il progetto. In qusta sezione vengono stilate le norme per la gestione dei rapporti con il proponente ed il committente.
\subsubsection{Repository pubblica}
Durante tutto il progetto verrà usata un' unica repository pubblica. La parte di repository che interessa al proponente ed al committente è la cartella \textit{public} definita come parte pubblica della repository. Questa cartella contiene un file read-me in cui viene spiegato ed elencato il contenuto della parte pubblica con un avviso che chiede di restarne all'interno, ignorando il resto della repository, detta parte privata.\\\\
La parte pubblica contiene i documenti necessari alle revisioni di periodo con il committente.\\
Ogni file nella parte pubblica deve indicare la versione nel nome del file stesso.\\
Ogni modifica alla parte pubblica deve essere approvata dal responsabile, cui spetta accettare le pull request della parte pubblica.
\subsubsection{Contatti con proponente e committente}

Tutti i contatti con il proponente ed il committente avvengono tramite la mail del gruppo \textit{winnningsoftwaresolution@gmail.com}.\\\\
L'invio di una mail nasce dalla richiesta di uno o più componenti del gruppo e necessita di una discussione con il responsabile e della sua approvazione.\\\\
Le mail per la presentazione ad una revisione di periodo necessita della conferma di tutti i componenti, che valutano l'effettiva preparazione del gruppo per la revisione. Questa mail deve contenere la lettera di presentazione con il link alla parte pubblica della repository.\\\\
I colloqui devono essere richiesti tramite mail. La reale necessita di un colloquio va discussa durante la riunione settimanale.\\\\
I colloqui con il proponente si svologo su meet o su discord. I colloqui con il committente si svolgono su zoom.
\subsection{Sviluppo}

\section{Processi di supporto}
\subsection{Documentazione}
\subsubsection{Strumenti Utilizzati}
Per la stesure dei documenti verrà utilizzato il linguaggio \LaTeX.
Ogni documento dovrà includere il file \texttt{docs/template/style.tex} che contiene indicazioni di stile.
Ogni documento userà per la prima pagina il template \texttt{docs/template/front\_page.tex}.
Eventuali risorse che dovessero servire per il template, saranno nella stessa cartella.
\subsubsection{Struttura}
Ogni documento avrà:
\begin{itemize}
\item Una pagina di intestazione
\item Una pagina di elenco delle versioni, dove ammesso
\item Una pagina di indice, dove ammesso
\item Il contenuto
\end{itemize}

Le sezioni avranno le seguenti caratteristiche:

\paragraph{Pagina di intestazione}
La pagina di intestazione contiene:
\begin{itemize}
\item logo e nome del gruppo;
\item nome del progetto e azienda cliente;
\item contatto;
\item titolo del documento;
\item informazioni sul documento:
\begin{itemize}
	\item elenco dei redattori;
	\item elenco dei revisori;
	\item elenco dei responsabili;
	\item versione, laddove prevista;
	\item destinazione d'uso;
\end{itemize}
\item breve descrizione.
\end{itemize}

La destinazione d'uso può essere ``interno'' o ``esterno''.



\paragraph{Elenco delle versioni}
L'elenco delle versioni è una tabella con le seguenti colonne:
\begin{itemize}
\item versione;
\item data;
\item persona;
\item attività;
\item descrizione.
\end{itemize}

Con persona si intende il membro del gruppo che svolge l'attività.

Le possibili attività su un documento sono ``Redazione'', ``Revisione'' e ``Approvazione''.

Le righe della tabella sono in ordine cronologico inverso.

\paragraph{Indice}
L'indice è creato usando il comando \verb-\tableofcontents-.

\subsubsection{Convenzioni utilizzate}
\paragraph{Riferimenti a file e cartelle}
Per indicare il nome di un file o di una cartella si utilizza il  \texttt{testo monospaziato}.

Il nome di una cartella termina sempre con \texttt{/}.

\paragraph{Stringhe e nomi}
Le stringhe vengono scritte tra doppi apici `` ''.

Per indicare un parametro in un nome o una stringa si usa del testo tra parentesi angolari $<$ $>$. Il parametro non può contenere spazi.

Per indicare una parte di nome o stringa opzionale si usa del testo tra parentesi quadre [ ].

\paragraph{Riferimenti tra documenti}
Per riferirsi a sezioni all'interno dello stesso documento si usa il comando \verb-\ref-.

Per riferirsi a sezioni di un altro documento si usa il nome di quella sezione scritto in corsivo.

\subsection{Versionamento}
Per il versionamento si farà uso del Versionamento Semantico, secondo le seguenti \href{https://semver.org/lang/it/#specifica-di-versionamento-semantico-semver}{linee guida}, se non indicato diversamente successivamente.

Un qualsiasi scatto di versione può avvenire solo attraverso una previa verifica
\subsection{Verifica}
\subsubsection{Regole Generali}
\begin{itemize}
\item Il verificatore si occupa di controllare le Pull Request aperte dagli altri membri del gruppo
\item Il verificatore non può aver contribuito ai prodotti che deve verificare.
\item Il verificatore non può modificare i prodotti che verifica, ad eccezione di correzioni grammaticali.
\item Il verificatore, qualora trovasse errori, è tenuto a commentare le PR, usando la funzione apposita di GitHub.
\end{itemize}
\subsubsection{Metodi di Verifica}
Il verificatore può far uso dei test automatici presenti nel \textit{Piano di qualifica}.
Per le sezioni di codice è consigliato un walkthrough del prodotto, similmente si dovrebbero chiedere a chi abbia scritto il codice eventuali delucidazioni su elementi poco chiari 
\subsection{Change Management}
Nel caso si dovessero cambiare sezione del codice bisogna controllare che eventuali modifiche non vadano ad impattare altri elementi. Per evitare ciò bisogna fare uso del \textit{Diagramma delle classi} presente nelle \textit{Specifiche architetturali}.
\section{Processi organizzativi}
\subsection{Gestione dei processi}
\subsection{Miglioramento dei processi}

\pagebreak
\section{norme vecchie}
\subsection{Documentazione}\label{documentazione}

\subsubsection{Strumenti utilizzati}
Per la stesura dei documenti verrà utilizzato il linguaggio \LaTeX.

\subsubsection{Ciclo di vita}
Ogni documento passa per le seguenti fasi di vita:
\begin{itemize}
\item \textbf{Pianificazione:} il documento viene discusso e delineato per sommi capi sulla base delle necessità a cui deve rispondere;
\item \textbf{Redazione:} un membro del gruppo, nel ruolo di redattore, scrive fisicamente il documento;
\item \textbf{Revisione:} un membro del gruppo, nel ruolo di revisore, controlla che non siano presenti errori grammaticali e che sia aderente alle norme di progetto;
\item \textbf{Approvazione:} un membro del gruppo, nel ruolo di responsabile, verifica che il contenuto del documento sia corretto e coerente con lo scopo dello stesso.
\end{itemize}

Non è possibile per un membro del gruppo coprire più ruoli nella gestione dello stesso documento.

Una volta che un documento è revisionato o approvato può comunque tornare in fase di stesura sulla base delle esigenze del momento.

\subsubsection{Versionamento}
Ogni documento è univocamente identificabile dal titolo e dalla versione. La versione è definita da tre numeri nel formato \texttt{$<$x$>$.$<$y$>$.$<$z$>$}.

La prima versione è la \texttt{0.0.0}

Il numero \texttt{$<$z$>$} viene modificato ogni volta che viene effettuata una stesura del documento.

Il numero \texttt{$<$y$>$} viene modificato ogni volta che viene effettuata una revisione del documento.

Il numero \texttt{$<$x$>$} viene modificato ogni volta che viene effettuata un'approvazione del documento.

Se il numero \texttt{$<$x$>$} viene modificato, \texttt{$<$y$>$} e \texttt{$<$z$>$} tornano a 0.

Se il numero \texttt{$<$y$>$} viene modificato, \texttt{$<$z$>$} torna a 0.

I verbali sono esclusi dal sistema di versionamento poiché non possono essere modificati nel tempo.

\subsubsection{Organizzazione dei file}
Tutta la documentazione sarà contenuta all'interno della cartella \texttt{docs/}. La cartella è a sua volta divisa in due sottocartelle \texttt{docs/interni/} e \texttt{docs/esterni/}, che conterranno rispettivamente la documentazione interna al gruppo e quella da condividere con gli altri stakeholders.

Tutti i verbali saranno conservati nella sottocartella \texttt{interni/verbali/}.

Per ogni documento sarà presente una cartella che sarà chiamata \texttt{$<$nome$>$/}, dove \texttt{$<$nome$>$} è il nome del documento come specificato nella sezione \ref{documentazione-documenti}.

Il file principale di ogni documento sarà all'interno della rispettiva cartella e sarà chiamato \texttt{$<$nome$>$.tex}, dove \texttt{$<$nome$>$} è il nome del file come definito nella sezione \ref{documentazione-documenti}.

Eventuali altri file utilizzati nella stesura del documento saranno posizionati all'interno della stessa cartella.

\subsubsection{Struttura dei documenti}
Ogni documento avrà una pagina di intestazione e una pagina con l'elenco delle versioni, laddove il versionamento del documento è ammesso.
Sarà inserita una pagina con l'indice, a meno di indicazione contraria nella sezione \ref{documentazione-documenti}.

\paragraph{Pagina di intestazione}
La pagina di intestazione contiene:
\begin{itemize}
\item logo e nome del gruppo;
\item nome del progetto e azienda cliente;
\item contatto;
\item titolo del documento;
\item informazioni sul documento:
\begin{itemize}
	\item elenco dei redattori;
	\item elenco dei revisori;
	\item elenco dei responsabili;
	\item versione, laddove prevista;
	\item destinazione d'uso;
\end{itemize}
\item breve descrizione.
\end{itemize}

La destinazione d'uso può essere ``interno'' o ``esterno''.

Ogni documento userà per la prima pagina il template \texttt{docs/template/front\_page.tex}.

Eventuali risorse che dovessero servire per il template, saranno nella stessa cartella.

\paragraph{Elenco delle versioni}
L'elenco delle versioni è una tabella con le seguenti colonne:
\begin{itemize}
\item versione;
\item data;
\item persona;
\item attività;
\item descrizione.
\end{itemize}

Con persona si intende il membro del gruppo che svolge l'attività.

Le possibili attività su un documento sono ``Redazione'', ``Revisione'' e ``Approvazione''.

Le righe della tabella sono in ordine cronologico inverso.

\paragraph{Stile}
Ogni documento dovrà includere il file \texttt{docs/template/style.tex} che contiene indicazioni di stile.



\subsubsection{Convenzioni utilizzate}
\paragraph{Riferimenti a file e cartelle}
Per indicare il nome di un file o di una cartella si utilizza il  \texttt{testo monospaziato}.

Il nome di una cartella termina sempre con \texttt{/}.

\paragraph{Stringhe e nomi}
Le stringhe vengono scritte tra doppi apici `` ''.

Per indicare un parametro in un nome o una stringa si usa del testo tra parentesi angolari $<$ $>$. Il parametro non può contenere spazi.

Per indicare una parte di nome o stringa opzionale si usa del testo tra parentesi quadre [ ].

\paragraph{Riferimenti tra documenti}
Per riferirsi a sezioni all'interno dello stesso documento si usa il comando \verb-\ref-.

Per riferirsi a sezioni di un altro documento si usa il nome di quella sezione scritto in corsivo.

\subsubsection{Documenti}\label{documentazione-documenti}

\paragraph{Norme di progetto}
\subparagraph{Scopo}
Lo scopo delle norme di progetto è definire procedure, strumenti e criteri di qualità al fine di stabilire un Way of Working. Ogni membro del gruppo sarà tenuto a rispettare le indicazioni lì presentate.
\subparagraph{Titolo}
Il titolo del documento è ``Norme di progetto".
\subparagraph{Nome del file}
Il file sarà chiamato \texttt{norme\_di\_progetto.tex}.

\paragraph{Verbali}
\subparagraph{Scopo}
Lo scopo dei verbali è tenere traccia del dialogo interno ed esterno al gruppo e delle eventuali decisioni prese.
\subparagraph{Titolo}
Ogni verbale sarà titolato ``Verbale del $<$data$>$[ con $<$esterni$>$]", dove $<$data$>$ indica la data dell'incontro nel formato europeo. In caso all'incontro siano presenti persone esterne al gruppo, verrà usata anche la dicitura ``con $<$esterni$>$", dove $<$esterni$>$ indica i partecipanti alla riunione separati da virgole. Gli esterni possono essere identificati collettivamente tramite il nome dell'azienda o ente che rappresentano, tramite nome e cognome o tramite titolo e cognome.

Esempi:
\begin{itemize}
\item ``Verbale del 09/11/2021'';
\item ``Verbale del 28/10/2021 con SyncLab'';
\item ``Verbale del 25/12/2021 con prof. Cardin''.
\end{itemize}
\subparagraph{Nome dei file}
I file saranno chiamati \texttt{$<$data$>$\_$<$tipo$>$.tex}, dove \texttt{$<$data$>$} è la data dell'incontro in formato \texttt{$<$yyyy$>$\_$<$mm$>$\_$<$dd$>$} e \texttt{$<$tipo$>$} è \texttt{I} se non sono presenti persone esterne al gruppo, \texttt{E} altrimenti.
\subparagraph{Indice}
Non è presente una pagina con l'indice.
\subparagraph{Struttura}
Ogni verbale deve riportare:
\begin{itemize}
\item data, ora e durata;
\item luogo;
\item partecipanti;
\item ordine del giorno;
\item riassunto dei contenuti;
\item eventuali impegni assunti.
\end{itemize}
\subparagraph{Stesura}\label{documentazione-documenti-verbali-stesura}
All'inizio di ogni riunione verrà nominato un partecipante che redigerà il verbale e uno che lo approverà.

Chi redige il verbale ha 24 ore per completare il lavoro. Chi approva il verbale ha 24 ore per approvarlo o segnalare eventuali correzioni da apportare.

Nel secondo caso, il redattore avrà ulteriori 24 ore per adeguarsi alle indicazioni.

\paragraph{Piano di Qualifica}\label{documentazione-documenti-piano-di-qualifica}
\subparagraph{Scopo}
Lo scopo del \textit{Piano di Qualifica} è definire le metriche e i requisiti minimi di qualità per l'approvazione dei prodotti del progetto.
\subparagraph{Titolo}
Il titolo del documento è ``Piano di Qualifica".
\subparagraph{Nome del file}
Il file sarà chiamato \texttt{piano\_di\_qualifica.tex}.

\paragraph{Piano di Progetto}\label{piano-di-progetto}
\subparagraph{Scopo}
Lo scopo del \textit{Piano di Progetto} è definire gli obiettivi da raggiungere, stabilendo le tempistiche e chi se ne occupa e tiene traccia del loro progresso, valutando i costi sostenuti rispetto a quelli preventivati.
\subparagraph{Titolo}
Il titolo del documento è ``Piano di Progetto".
\subparagraph{Nome del file}
Il file sarà chiamato \texttt{piano\_di\_progetto.tex}.

\paragraph{Analisi dei requisiti}\label{analisi-dei-requisiti}
\subparagraph{Scopo}
Lo scopo dell'\textit{Analisi dei requisiti} è raccogliere i risultati dell’attività di analisi dei requisiti. Esso contiene quindi la descrizione dei casi d’uso del prodotto software da sviluppare, ed i requisiti suddivisi per tipologia.
\subparagraph{Titolo}
Il titolo del documento è ``Analisi dei requisiti".
\subparagraph{Nome del file}
Il file sarà chiamato \texttt{analisi\_dei\_requisiti.tex}.

\paragraph{Analisi delle tecnologie}\label{analisi-delle-tecnologie}
\subparagraph{Scopo}
Lo scopo dell'\textit{Analisi delle tecnologie} è valutare le tecnologie disponibili alla realizzazione del progetto. Esso contiene quindi le metriche di valutazione e l'analisi delle tecnologie in base a tali metriche.
\subparagraph{Titolo}
Il titolo del documento è ``Analisi delle tecnologie".
\subparagraph{Nome del file}
Il file sarà chiamato \texttt{analisi\_delle\_tecnologie.tex}.

\subsection{Verifica e Validazione}\label{verifica-e-validazione}
Ogni prodotto per essere accettato nel branch \texttt{main} deve rispettare i requisiti minimi definiti dal \textit{Piano di Qualifica}. 

A tal fine, verrà predisposto un insieme di test per rendere automatica, in toto o in parte, la verifica e la validazione dei prodotti.

\subsection{Configurazione}

\subsubsection{Repository GitHub}\label{configurazione-github}
Per il versionamento del progetto si è scelto di utilizzare git su piattaforma GitHub.

La repository \url{https://github.com/iota97/WinningSoftwareSolution} conterrà tutti i file del progetto.

\paragraph{Comandi utili}
Alcuni comandi utili sono:
\begin{itemize}
\item \textbf{Sincronizzazione con repository remota:} \texttt{git pull};
\item \textbf{Creazione di un nuovo branch}: \texttt{git branch $<$nome\_branch$>$};
\item \textbf{Passaggio a un branch specificata:} \texttt{git checkout $<$nome\_branch$>$};
\item \textbf{Aggiunta delle modifiche alla stage area:} \texttt{git add .};
\item \textbf{Creazione del commit con le modifiche:} \texttt{git commit};
\item \textbf{Push sul remote di un nuovo branch:} \texttt{git push --set-upstream origin $<$nome\_branch$>$};
\item \textbf{Push su un branch già esistente:} \texttt{git push}.
\end{itemize}

\section{(vecchio)Workflow di lavoro}
\subsection{Versionamento}

Ogni modifica alla repository dovrà seguire i seguenti passaggi:
\begin{itemize}
\item modifica del codice su repository locale;
\item modifica su un branch collettivo;
\item pull request tramite piattaforma web GitHub per merge sul branch \texttt{main};
\item review della pull request;
\item merge sul branch \texttt{main}.
\end{itemize}

Un branch collettivo è un branch su cui operano uno o più membri del gruppo e ha come scopo l'elaborazione di un prodotto o di un processo di progetto. Il branch si chiamerà come il prodotto o il processo relativo.

Le modifiche su un branch collettivo avverranno tramite merge di un branch personale o con commit diretto. I membri del gruppo che lavorano sul branch concordano una modalità.

È auspicabile che i branch personali non compaiano nella repository remota.

Prima di effettuare un push alla repository remota o un merge è necessario fare uno squash dei commit superflui.

Non appena chi lavora su un branch collettivo ritenga che ci sia materiale pronto per una revisione, apre una pull request su GitHub.

La review dovrà essere fatta dal verificatore. Il verificatore potrà accettare la pull request.

Una pull request può essere accettata solo se tutti i prodotti coinvolti superano i requisiti minimi definiti nel \textit{Piano di Qualifica}.

Quando il verificatore accetta un pull request, dovrà eliminare il branch relativo a quest'ultima.

Non è consentito fare il caricamento diretto di file tramite interfaccia web o desktop.

Non è consentito forzare i push sulla repository remota.

\subsection{Sviluppo Software}\label{norme-sw}
Per lo sviluppo del codice valgono le seguenti regole:
\begin{itemize}
	\item chi modifica il SW nella stessa PR modifica la documentazione: diagramma delle classi, di sequenza e lista metodi;
    \item eseguire i test automatici prima di creare la PR (no regression);
    \item se si inserisce codice assicurarsi che sia coperto da test;
    \item non lasciare nulla di implicito, sia per visibilità che per tipi.
\end{itemize}

\section{(vecchio)Tracciamento dei progressi}
Per tracciare l'avanzamento del progetto, oltre alla dashboard nel \textit{Piano di Progetto}, verrà utilizzato il sistema di issues e di milestone di GitHub.

Il \textit{Piano di Progetto} definisce quali sono le milestone.

Per ogni milestone verranno create delle issues, in accordo con il \textit{Piano di Progetto}.

Gli issues sono obiettivi intermedi, necessari a raggiungere la milestone corrispondente.

Ogni issue è legato a una milestone.

Un issue può essere diviso in più parti utilizzando una task list.

Ogni sotto-task può essere trasformata in un issue. Quest'ultimo dovrà essere collegato alla stessa milestone dell'issue padre.

\subsection{Revisioni di avanzamento interne}\label{revisioni-di-avanzamento-interne}
Al fine di verificare periodicamente l'effettivo avanzamento delle varie task e meglio coordinare il gruppo è previsto di trovarsi sul canale Discord del gruppo settimanalmente.\\

Questo incontro avverrà ogni \textbf{Mercoledì alle 15.00}.\\

In caso un componente non possa presentarsi dovrà presentare un resoconto scritto del suo stato di avanzamento ed eventuali criticità riscontrate.\\

Questo resoconto va consegnato \textbf{prima del meeting} in modo che sia possibile discutere con i presenti anche della parte di lavoro svolta dagli assenti.
Gli assenti rimarranno aggiornati sullo stato di avanzamento consultando i verbali redatti in occasione di ogni incontro.

\end{document}