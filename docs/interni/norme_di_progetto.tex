\documentclass[a4paper,12pt]{article}

\begin{document}

\section{Introduzone}
\subsection{Scopo del documento}
Lo scopo del presente documento è definire procedure, strumenti e criteri di qualità al fine di stabilire un Way of Working. Ogni membro del gruppo sarà tenuto a rispettare le indicazioni qui presentate.

\section{Processi di supporto}

\subsection{Documentazione}

\subsubsection{Strumenti utilizzati}
Per la stesura dei documenti verrà utilizzato il linguaggio \LaTeX.

\subsubsection{Versionamento}
Ogni documento è univocamente identificabile dal titolo e dalla versione. La versione è definita da tre numeri nel formato \texttt{$<$x$>$.$<$y$>$.$<$z$>$}

\subsubsection{Struttura dei documenti}

\subsubsection{Organizzazione dei file}
I file della documentazione saranno tutti denominati \texttt{$<$nome$>$.tex}, dove \texttt{$<$nome$>$} è il nome del file come definito nella sezione \ref{documentazione-documenti}.

Tutta la documentazione sarà contenuta all'interno della cartella \texttt{docs}. La cartella è a sua volta divisa in due sottocartelle \texttt{interni} ed \texttt{esterni}, che conterranno rispettivamente la documentazione interna al gruppo e quella da condividere con gli altri stakeholders.

Tutti i verbali saranno conservati nella sottocartella \texttt{interni/verbali}.


\subsubsection{Documenti}\label{documentazione-documenti}

\paragraph{Norme di progetto}
\subparagraph{Scopo}
Lo scopo delle norme di progetto è definire procedure, strumenti e criteri di qualità al fine di stabilire un Way of Working. Ogni membro del gruppo sarà tenuto a rispettare le indicazioni lì presentate.
\subparagraph{Titolo}
Il titolo del documento è ``Norme di progetto"
\subparagraph{Nome del file}
Il file sarà chiamato \texttt{norme\_di\_progetto.tex}.

\paragraph{Verbali}
\subparagraph{Scopo}
Lo scopo dei verbali è tenere traccia del dialogo interno ed esterno al gruppo e delle eventuali decisioni prese.
\subparagraph{Titolo}
Ogni verbale sarà titolato ``Verbale del $<$data$>$", dove $<$data$>$ indica la data dell'incontro nel formato europeo. In caso all'incontro siano presenti persone esterne al gruppo verà aggiunta la dicitura ``[con $<$esterni$>$]", dove $<$esterni$>$ indica i partecipanti alla riunione separati da virgole. Gli esterni possono essere identificati collettivamente tramite il nome dell'azienda o ente che rappresentano, tramite nome e cognome o tramite titolo e cognome.\\
Esempi:
\begin{itemize}
\item Verbale del 09/11/2021
\item Verbale del 28/10/2021 con SyncLab
\item Verbale del 25/12/2021 con prof. Cardin
\end{itemize}
\subparagraph{Nome dei file}
I file saranno chiamati \texttt{$<$data$>$\_$<$tipo$>$.tex}, dove \texttt{$<$data$>$} è la data dell'incontro in formato \texttt{$<$yyyy$>$\_$<$mm$>$\_$<$dd$>$} e \texttt{$<$tipo$>$} è \texttt{I} se non sono presenti persone esterne al gruppo, \texttt{E} altrimenti.
\subparagraph{Struttura}
Ogni verbale deve riportare:
\begin{itemize}
\item data, ora e durata;
\item luogo;
\item partecipanti;
\item ordine del giorno;
\item riassunto dei contenuti;
\item eventuali impegni assunti.
\end{itemize}

\subsection{Configurazione}

\subsubsection{Repository GitHub}

% Ogni altro file generato dal compilatore \LaTeX non dovrà essere considerato. 


\end{document}