\documentclass[a4paper,12pt]{article}

\begin{document}

%\input{../../template/front_page}

\section{Introduzone}
\subsection{Scopo del documento}
Lo scopo del presente documento è definire procedure, strumenti e criteri di qualità al fine di stabilire un Way of Working. Ogni membro del gruppo sarà tenuto a rispettare le indicazioni qui presentate.

\section{Processi di supporto}

\subsection{Documentazione}

\subsubsection{Strumenti utilizzati}
Per la stesura dei documenti verrà utilizzato il linguaggio \LaTeX.

\subsubsection{Ciclo di vita}
Ogni documento passa per le seguenti fasi di vita:
\begin{itemize}
\item \textbf{Pianificazione:} il documento viene discusso e delineato per sommi capi sulla base delle necessità a cui deve rispondere;
\item \textbf{Stesura:} un membro del gruppo, nel ruolo di redattore, scrive fisicamente il documento;
\item \textbf{Revisione:} un membro del gruppo, nel ruolo di revisore, controlla che non siano presenti errori grammaticali e che sia aderente alle norme di progetto;
\item \textbf{Approvazione:} un membro del gruppo, nel ruolo di responsabile, verifica che il contenuto del documento sia corretto e coerente con lo scopo dello stesso.
\end{itemize}

Non è possibile per un membro del gruppo coprire più ruoli nella gestione dello stesso documento.

Una volta che un documento è revisionato o approvato può comunque tornare in fase di stesura sulla base delle esigenze del momento.

\subsubsection{Versionamento}
Ogni documento è univocamente identificabile dal titolo e dalla versione. La versione è definita da tre numeri nel formato \texttt{$<$x$>$.$<$y$>$.$<$z$>$}.

La prima versione è la \texttt{0.0.0}

Il numero \texttt{$<$z$>$} viene modificato ogni volta che viene effettuata una stesura del documento. \\
Il numero \texttt{$<$y$>$} viene modificato ogni volta che viene effettuata una revisione del documento. \\
Il numero \texttt{$<$x$>$} viene modificato ogni volta che viene effettuata un'approvazione del documento.

Se il numero \texttt{$<$x$>$} viene modificato, \texttt{$<$y$>$} e \texttt{$<$z$>$} tornano a 0. \\
Se il numero \texttt{$<$y$>$} viene modificato, \texttt{$<$z$>$} torna a 0.

I verbali sono esclusi dal sistema di versionamento poiché non possono essere modificati nel tempo.

\subsubsection{Organizzazione dei file}
Tutta la documentazione sarà contenuta all'interno della cartella \texttt{docs/}. La cartella è a sua volta divisa in due sottocartelle \texttt{docs/interni/} e \texttt{docs/esterni/}, che conterranno rispettivamente la documentazione interna al gruppo e quella da condividere con gli altri stakeholders.

Tutti i verbali saranno conservati nella sottocartella \texttt{interni/verbali/}.

Per ogni documento sarà presente una cartella che sarà chiamata \texttt{$<$nome$>$/}, dove \texttt{$<$nome$>$} è il nome del documento come specificato nella sezione \ref{documentazione-documenti}.

Il file principale di ogni documento sarà all'interno della rispettiva cartella e sarà chiamato \texttt{$<$nome$>$.tex}, dove \texttt{$<$nome$>$} è il nome del file come definito nella sezione \ref{documentazione-documenti}.

Eventuali altri file utilizzati nella stesura del documento saranno posizionati all'interno della stessa cartella.

\subsubsection{Struttura dei documenti}
Ogni documento avrà una pagina di intestazione e una pagina con l'elenco delle versioni, laddove il versionamento del documento è ammesso.
Sarà inserita una pagina con l'indice, a meno di indicazione contraria nella sezione \ref{documentazione-documenti}.

\paragraph{Pagina di intestazione}
La pagina di intestazione contiene:
\begin{itemize}
\item logo e nome del gruppo;
\item nome del progetto;
\item contatto;
\item titolo del documento;
\item informazioni sul documento:
\begin{itemize}
	\item elenco dei redattori;
	\item elenco dei revisori;
	\item elenco dei responsabili;
	\item versione, laddove prevista;
	\item destinazione d'uso interna o esterna;
\end{itemize}
\item breve descrizione.
\end{itemize}

Ogni documento userà per la prima pagina il template \texttt{docs/template/front\_page.tex}.\\
Eventuali risorse che dovessero servire per il template, saranno nella stessa cartella.

\paragraph{Elenco delle versioni}
L'elenco delle versioni è una tabella con le seguenti colonne:
\begin{itemize}
\item versione;
\item data;
\item persona;
\item attività;
\item descrizione.
\end{itemize}

Con persona si intende il membro del gruppo che svolge l'attività.

Le possibili attività su un documento sono la stesura, la revisione e la approvazione.

Le righe della tabella sono in ordine cronologico.

\subsubsection{Convenzioni utilizzate}
Per indicare il nome di un file o di una cartella si utilizza il  \texttt{testo monospaziato}.

Il nome di una cartella termina sempre con \texttt{/}.

Le stringhe vengono scritte tra doppi apici `` ''.

Per indicare un parametro in un nome o una stringa si usa del testo tra parentesi angolari $<$ $>$. Il parametro non può contenere spazi.

Per indicare una parte di nome o stringa opzionale si usa del testo tra parentesi quadre [ ].

Negli elenchi puntati o numerati ogni item termina con un punto e virgola, salvo l'ultimo che termina con un punto.

Se la voce di un elenco descrive un concetto, un termine o un oggetto, allora esso va scritto in grassetto seguito da due punti. 

Ogni frase termina con un punto.

\subsubsection{Documenti}\label{documentazione-documenti}

\paragraph{Norme di progetto}
\subparagraph{Scopo}
Lo scopo delle norme di progetto è definire procedure, strumenti e criteri di qualità al fine di stabilire un Way of Working. Ogni membro del gruppo sarà tenuto a rispettare le indicazioni lì presentate.
\subparagraph{Titolo}
Il titolo del documento è ``Norme di progetto".
\subparagraph{Nome del file}
Il file sarà chiamato \texttt{norme\_di\_progetto.tex}.

\paragraph{Verbali}
\subparagraph{Scopo}
Lo scopo dei verbali è tenere traccia del dialogo interno ed esterno al gruppo e delle eventuali decisioni prese.
\subparagraph{Titolo}
Ogni verbale sarà titolato ``Verbale del $<$data$>$[ con $<$esterni$>$]", dove $<$data$>$ indica la data dell'incontro nel formato europeo. In caso all'incontro siano presenti persone esterne al gruppo, verrà usata anche la dicitura ``con $<$esterni$>$", dove $<$esterni$>$ indica i partecipanti alla riunione separati da virgole. Gli esterni possono essere identificati collettivamente tramite il nome dell'azienda o ente che rappresentano, tramite nome e cognome o tramite titolo e cognome.\\
Esempi:
\begin{itemize}
\item ``Verbale del 09/11/2021'';
\item ``Verbale del 28/10/2021 con SyncLab'';
\item ``Verbale del 25/12/2021 con prof. Cardin''.
\end{itemize}
\subparagraph{Nome dei file}
I file saranno chiamati \texttt{$<$data$>$\_$<$tipo$>$.tex}, dove \texttt{$<$data$>$} è la data dell'incontro in formato \texttt{$<$yyyy$>$\_$<$mm$>$\_$<$dd$>$} e \texttt{$<$tipo$>$} è \texttt{I} se non sono presenti persone esterne al gruppo, \texttt{E} altrimenti.
\subparagraph{Indice}
Non è presente una pagina con l'indice.
\subparagraph{Struttura}
Ogni verbale deve riportare:
\begin{itemize}
\item data, ora e durata;
\item luogo;
\item partecipanti;
\item ordine del giorno;
\item riassunto dei contenuti;
\item eventuali impegni assunti.
\end{itemize}

\subsection{Configurazione}

\subsubsection{Repository GitHub}

Ogni componente del gruppo clonera il repo in locale:\\
	\texttt{git clone https://github.com/iota97/WinningSoftwareSolution}\\\\

Workflow per la modifica del repo:
\begin{itemize}
\item Per creare una modifica al repo prima sincronizzare all'ultima versione del remote\\
	\texttt{git pull};
\item Selezionare la branch main\\
	\texttt{git checkout main};
\item Creare una nuova branch nel formato $<$iniziali\_nome\_cognome$>$-$<$nome\_branch$>$\\
	\texttt{git branch gc-istruzioni-git};
\item Passare alla branch specificata\\
	\texttt{git checkout gc-istruzioni-git};
\item Creare le modifiche necessarie;
\item Aggiungere le modifiche alla stage area\\
	\texttt{git add .};
\item Creare il commit con le modifiche\\
	\texttt{git commit};
\item Pushare sul remote la nuova branch\\
	\texttt{git push --set-upstream origin gc-istruzioni-git};
\item Da GitHub creare la PR;
\item Se necessarie ulteriori modifiche modificare i file e pushare i cambiamenti\\
	\texttt{git push}.
\end{itemize}

\textbf{NB}: Prima di pushare usare \texttt{git status} per verificare di essere nella branch corretta etc...\\\\
Una persona diversa da quella che ha creato la PR si occuperà della review.\\
Tramite l'interfaccia web di Github commenterà se sono necessarie modifiche; una volta approvata procederà al merge.\\\\
Una volta che la PR è stata accettata il creatore eliminera la branch non più necessaria (GitHub mette il tasto a disposizione sulla pagina della PR una volta mergiata).


% Ogni altro file generato dal compilatore \LaTeX non dovrà essere considerato. 


\end{document}