\documentclass[a4paper,12pt]{article}
\author{}
\date{}

\begin{document}
\title{Verbale del 28/10/2021 con SyncLab}
\maketitle

\section{Informazioni}
\textbf{Data}: 28/10/2021\\
\textbf{Durata}: 45 minuti\\
\textbf{Luogo}: Aula virtuale \textit{Google Meet}\\\\

\textbf{Partecipanti}:
\begin{itemize}
	\item Andrea Volpe
	\item Matteo Galvagni
	\item Federico Marchi
	\item Raffaele Oliviero
	\item Albero Nicoletti
	\item Fabio Pallaro
\end{itemize}

\section{Ordine del giorno}
\begin{enumerate}
    \item Presentazioni.
    \item Riassunto delle richieste del proponente.
    \item Domande da parte del gruppo al proponente.
    \item Varie ed eventuali
\end{enumerate}

\section{Svolgimento}

\subsection{Presentazioni}
I membri del gruppo presenti al meeting si sono rapidamente presentati, così come il rappresentante dell'azienda \textit{Synclab} Fabio Pallaro.

\subsection{Riassunto delle richieste del proponente}
È stato fatto un breve riassunto delle richieste relative al progetto e delle modalità di svolgimento, al fine di inquadrare al meglio le aspettative del proponente.

\subsection{Domande da parte del gruppo al proponente}
Sono state richieste diverse delucidazioni circa i punti non ancora chiari delle richieste del proponente.

\begin{itemize}
	\item \textbf{Estensione del sito web di E-commerce.} 
	È stato fatto notare che non è richiesto affatto un sito di E-commerce completo. Ciò che è richiesto è un'applicazione web utile alle procedure di pagamento di un ipotetico E-commerce, assieme ad una pagina ove sia possibile consultare lo stato di pagamenti/ordini.
	\item \textbf{Scelta delle tecnologie utilizzabili.} 
	Il proponente, seppur avanzando qualche consiglio circa le tecnologie da utilizzare (presenti nel capitolato), lascia la quasi totale libertà di scelta per quanto riguarda le tecnologie da usare, ed anzi comunica che parte integrante del progetto sarà la ricerca di nuove e/o più adatte tecnologie allo svolgimento del progetto, accompagnata da relativa documentazione.
	\item \textbf{Utilizzo di un codice QR per la conferma ricevuta di un pacco.}
	Alle perplessità sollevate circa la possibilità di truffe tramite un meccanismo di sblocco dei fondi con codice QR è stato comunicato che, seppur la ricerca di soluzioni alternative è ben vista e di interesse per l'azienda, è considerata come elemento facoltativo in quanto l'interesse principale dell'azienda è quello di sviluppare un metodo di pagamento e non di occuparsi di possibili attori malevoli durante la spedizione o la ricezione di un pacco. È invece fondamentale garantire la sicurezza del pagamento contro eventuali malintenzionati.
	\item \textbf{Varie ed eventuali.}
	Alla richiesta da parte del proponente del perchè questo capitolato sia risultato di interesse, il gruppo ha esposto le proprie personali motivazioni, tra cui un forte interesse verso le tecnologie da utilizzare e una propensione alla parte relativa alla ricerca e allo studio di nuove tecnologie.
\end{itemize}

\section{Conclusioni}
Le delucidazioni da parte del proponente sono state molto utili per capire più a fondo quali fossero le reali richieste e necessità dell'azienda, nonchè a rendere il gruppo più sicuro sulla scelta del capitolato.

\end{document}
